\chapter{Design sicuro} 

In questo capitolo sono esposte le scelte utilizzate per la realizzazione di un design sicuro, in modo tale da garantire elevati livelli di \textit{perfomance, usability e acceptability}, rendendo, però, il sistema sicuro da attacchi esterni.

Nelle successive Sezioni sono riportate le scelte effettuate nei seguenti campi:
\begin{itemize}
    \item \textbf{Architettura}, nella quale saranno trattate le scelte architetturali utilizzate;
    \item \textbf{Design degli asset}, nella quale saranano trattati i design utilizzati per modellare gli asset;
    \item \textbf{Scelte tecnologiche}. nella quale saranno trattate le scelte tecnologiche adottate;
    \item \textbf{Modellazione di una unità mediante Markov Chain}, nella quale sarà espostata la modellazione di una unità mediante Markov Chain.
\end{itemize}


\section{Architettura}

Il sistema progettato utilizza una blockchain per il calcolo e la validazione dei valori ottenuti dall'oracolo della rete bayesiana.

La rete bayesiana è composta da due livelli di: 

\begin{itemize}
    \item \textbf{livello dei fatti}, nel quale sono presenti elementi non conosciuti;
    \item \textbf{livello delle prove}, nel quale sono presenti elementi conoscibili.
\end{itemize}

Tutti gli elementi presenti nella rete bayesiana sono rappresentati da variabili casuali booleane.
Inoltre, è possibile stimare la probabilità di avvenimento degli elementi presenti nel livello dei fatti mediante il processo di \textbf{inferenza diagnostica}, basandosi, quindi, sui valori booleani delle prove osservate.

L'architettura scelta per l'oracolo bayesiano è divisa in una parte \textit{on-chain} ed una parte \textit{off-chain}; il sistema off-chain contiene le probabilità a priori degli elementi presenti nel livello dei fatti; inoltre, nella parte off-chain è presente la tabella delle probabilità condizionate degli elementi presenti nel livello dei fatti, in funzione delle prove osservate.

La parte on-chain, invece, fornisce le probabilità a priori degli elementi presenti nel livello dei fatti, se non sono stato osservate delle prove.

Inoltre, calcola la probabilità a posteriori degli elementi presenti nel livello dei fatti, mediante l'inferenza diagnostica, in base alle prove osservate.


\section{Design degli asset}

 In questa sezione, riportiamo la fase di design degli asset. Essa consiste nello scegliere
 alcune linee guida proposte da OWASP, Saltzer \& Schroeder e Sommerville, le quali verranno
 illustrate nelle successive sottosezioni.
 
 \subsection{Sommerville}
 \subsection{OWASP}
 \subsection{Saltzer \& Schroeder}



\section{Scelte tecnologiche}





