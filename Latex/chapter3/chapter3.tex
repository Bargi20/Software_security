\chapter{Design sicuro} 

In questo capitolo sono esposte le scelte utilizzate per la realizzazione di un design sicuro, in modo tale da garantire elevati livelli di \textit{perfomance, usability e acceptability}, rendendo, però, il sistema sicuro da attacchi esterni.

Nelle successive Sezioni sono riportate le scelte effettuate nei seguenti campi:
\begin{itemize}
    \item \textbf{Architettura}, nella quale saranno trattate le scelte architetturali utilizzate;
    \item \textbf{Design degli asset}, nella quale saranano trattati i design utilizzati per modellare gli asset;
    \item \textbf{Scelte tecnologiche}. nella quale saranno trattate le scelte tecnologiche adottate;
    \item \textbf{Modellazione di una unità mediante Markov Chain}, nella quale sarà espostata la modellazione di una unità mediante Markov Chain.
\end{itemize}


\section{Architettura}

Il sistema progettato utilizza una blockchain per il calcolo e la validazione dei valori ottenuti dall'oracolo della rete bayesiana.

La rete bayesiana è composta da due livelli di: 

\begin{itemize}
    \item \textbf{livello dei fatti}, nel quale sono presenti elementi non conosciuti;
    \item \textbf{livello delle prove}, nel quale sono presenti elementi conoscibili.
\end{itemize}

Tutti gli elementi presenti nella rete bayesiana sono rappresentati da variabili casuali booleane.
Inoltre, è possibile stimare la probabilità di avvenimento degli elementi presenti nel livello dei fatti mediante il processo di \textbf{inferenza diagnostica}, basandosi, quindi, sui valori booleani delle prove osservate.

L'architettura scelta per l'oracolo bayesiano è divisa in una parte \textit{on-chain} ed una parte \textit{off-chain}; il sistema off-chain contiene le probabilità a priori degli elementi presenti nel livello dei fatti; inoltre, nella parte off-chain è presente la tabella delle probabilità condizionate degli elementi presenti nel livello dei fatti, in funzione delle prove osservate.

La parte on-chain, invece, fornisce le probabilità a priori degli elementi presenti nel livello dei fatti, se non sono stato osservate delle prove.

Inoltre, calcola la probabilità a posteriori degli elementi presenti nel livello dei fatti, mediante l'inferenza diagnostica, in base alle prove osservate.


\section{Design degli asset}

 In questa sezione, riportiamo la fase di design degli asset. Essa consiste nello scegliere alcune linee guida proposte da OWASP, Saltzer \& Schroeder Sommerville, le quali verranno illustrate nelle successive sottosezioni.
 
 Si precisa che alcune linee guida sono comuni tra i vari modelli e sono state selezionate una sola volta.

\subsection{OWASP}

Le linee guida scelte tra quelle proposte da \textit{OWASP} sono state scelte le seguenti:

\begin{itemize}
    \item \textbf{Principle of Least privilege}, che consiste nel fornire agli account i privilegi minini richiesti per eseguire tutte le azioni possibili;
    \item \textbf{Fail securely}, che consiste nel permettere al sistema di affrontare dei fallimenti in maniera sicura;
    \item \textbf{Separation of duties}, che consiste nel separare le funzionalità permesse alle varie tipologie di utenti;
    \item \textbf{Fix security issues correctly}, che consiste nel correggere problematiche di sicurezza correttammente, evitando di peggiorare altre funzionalità non coinvolte.
\end{itemize}

\subsection{Sommerville}

Le linee guida scelte tra quelle proposte da \textit{Sommerville} sono state scelte le seguenti:

\begin{itemize}
    \item \textbf{Balance security and usability}, che consiste nel trovare un equilibrio tra la sicurezza e l'usabilità del sistema;
    \item \textbf{Log user actions}, che consiste nel registrare mediante un \textit{log} tutte le attività effettuate all'interno del sistema;
    \item \textbf{Use redundancy and diversity to reduce risk}, che consiste nell'utilizzare ridondanza e diversità per ridurre i rischi di attacchi esterni;
    \item \textbf{Specify the format of all system inputs}, che consiste nello specificare il formato dei valori in output del sistema.
\end{itemize}

\subsection{Saltzer \& Schroeder}

Le linee guida scelte tra quelle proposte da \textit{Saltzer \& Schroeder} sono state scelte le seguenti:

\begin{itemize}
    \item \textbf{KISS Principle}, che consiste nel realizzare il sistema di difesa nella maniera più semplice possibile;
    \item \textbf{Complete mediation}, che consiste nel controllare ogni tentativo di accesso al sistema prima di fornire all'utente la possibilità di eseguire qualsiasi azione;
    \item \textbf{Psychological acceptability}, che consiste nel rendere l'interazione tra il sistema e dell'utente il più intuitiva possibile.
\end{itemize}

\section{Scelte tecnologiche}





