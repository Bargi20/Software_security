\chapter{Deployment} 

In questo capitolo verrà descritto il processo di Deployment del nostro software Ledger Logistic. In particolare, viene illustrato, tramite una guida, come installare, configurare e avviare l'applicazione web. Ledger Logistic è un'applicazione web che interagisce con una rete blockchain privata di ETH, progettata per garantire sicurezza, trasparenza e tracciabilità nelle operazioni di logistica. L'obiettivo di questa sezione è quello di riportare in modo chiaro e dettagliato tutti i passaggi necessari per mettere in esecuzione l'applicazione in un ambiente di produzione o di test. Verranno presentati i prerequisiti, i passaggi e le configurazioni necessarie per l'installazione e l'avvio dell'applicazione web, nonché le considerazioni finali per garantire un funzionamento ottimale del sistema.

\section{Prerequisiti} 
Prima di procedere con l'installazione dell'applicazione web Ledger Logistic, è necessario assicurarsi di rispettare i seguenti prerequisiti:

\begin{itemize}
    \item \textbf{Python 3.10+} : assicurarsi di avere Python 3.10 o una versione successiva installata sul sistema, per la corretta esecuzione dell'applicazione web.
    \item \textbf{Node.js e npm} : installare Node.js (version 24.x) e npm la 11.6.2 (Node Package Manager) per gestire le dipendenze JavaScript necessarie per l'interfaccia web.
    \item \textbf{Hardhat} : installare Hardhat, un ambiente di sviluppo per la compilazione, il deployment e il testing di smart contract su Ethereum.
    \item \textbf{Git} : installare Git per la possibilità di clonare il repository del progetto Ledger Logistic.
    \item \textbf{Docker e Docker Compose} : installare Docker e Docker Compose per la gestione dei container necessari per eseguire la blockchain privata con algoritmo QBFT. 
    \item \textbf{MetaMask} : avere un account MetaMask configurato per test e per interagire con la rete blockchain privata.
    \item \textbf{Supabase} : avere un account Supabase per la corretta gestione del database PostgreSQL. 
\end{itemize}

\section{Installazione e configurazione}
Il primo procedimento da eseguire dopo aver soddisfatto i prerequisiti è quello di clonare il repository del progetto da GitHub in locale. Quindi creiamo una nuova cartella sul nostro dispositivo e, tramite terminale, ci spostiamo al suo interno per digitare il seguente comando: 

\begin{terminal}
git clone https://github.com/username/software_security.git
cd software_security
\end{terminal}

In seguito, bisogna installare le dipendenze e le librerie necessarie per l'esecuzione dell'applicazione web. 
\begin{terminal}
cd app 
python -m venv venv
source venv/bin/activate    # UTENTE MAC/LINUX
venv\\Scripts\\activate       # UTENTE WINDOWS
pip install -r requirements.txt
\end{terminal}

Dopo aver installato le dipendenze, a questo punto bisogna passare alla directory software\_security (ovvero la root del progetto) per installare le dipendenze necessarie per l'utilizzo degli smart contract. 

\begin{terminal}
cd ..
npm install
\end{terminal}

Infine, è necessario avviare i container Docker per inizializzare la rete blockchain privata. Per fare ciò, basta scrivere questo comando nella root del progetto: 
\begin{terminal}
docker-compose up -d
\end{terminal}

\section{Configurazione dopo l'avvio del sistema}

Dopo aver inizializzato tutto il necessario per l'applicazione web, bisogna configurare corettamente MetaMask per interagire con la rete blockchain. Aggiungendo la rete personalizzata con i seguenti parametri: 
\begin{itemize}
    \item \textbf{Nome della rete} : ExpressChain
    \item \textbf{Nuovo URL RPC} : http://localhost:8545
    \item \textbf{ID catena} : 1337
    \item \textbf{Simbolo valuta} : ETH
\end{itemize}

