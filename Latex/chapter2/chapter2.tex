\chapter{Valutazione del rischio}

\hbadness=10000

\section{Diagrammi i*}

In questa sezione verranno riportati i diagrammi i* realizzati per la modellazione del sistema e degli attori coinvolti, partendo da una visione generale senza sistema fino ad arrivare alla modellazione completa con oracolo bayesiano. 

\subsection{SD/SR ruoli senza sistema}

Il seguente diagramma rappresenta gli attori principali coinvolti nel sistema, senza considerare il sistema stesso, con le relative dipendenze. Sono stati identificati i seguenti attori principali:

\begin{itemize}
	\item \textbf{Cliente}: questo attore rappresenta il cliente finale che utilizza il sistema per effettuare acquisti online. L'azione di effettuare l'ordine richiede l'inserimento dei dati personali e di spedizione. Per effettuare il pagamento, dovrà inserire i propri dati della carta di credito, i quali saranno gestiti e validati dall'attore "Gestore circuito di pagamento". I dati del cliente comprendono i dati anagrafici, l'indirizzo di consegna ed informazioni aggiuntive come il numero di telefono, l'email ed il citofono.
	
	\item \textbf{Corriere}: questo attore rappresenta il corriere responsabile della consegna degli ordini ai clienti. Riceve le informazioni di spedizione dall'azienda di spedizione per effettuare le consegne.
	
	\item \textbf{Azienda di spedizione}: questo attore rappresenta l'azienda di spedizione che usa il sistema per gestire le consegne degli ordini. Raccoglie le informazioni degli ordini dai clienti e le fornisce al corriere. Inoltre, esso ha il compito di verificare eventuali reclami riguardo le spedizioni, la verifica di consegne in ritardo e della ricezione dei pagamenti comunicando con l'attore "Gestore circuito di pagamento".
	
	\item \textbf{Gestore circuito di pagamento}: questo attore rappresenta il gestore del circuito di pagamento che si occupa di validare i pagamenti effettuati dai clienti. Riceve i dati della carta di credito dal cliente e li elabora per autorizzare o rifiutare il pagamento. I dati della carta di credito comprendono il numero della carta, la data di scadenza, il codice CVV ed il saldo.
\end{itemize}

% Diagramma i* attori senza sistema
\begin{figure}[ht!]
	\centering
	\includegraphics[width=1\columnwidth]{chapter2/immagini/1\_diagramma\_attori\_no\_sistema.png}
\end{figure}

\clearpage

\subsection{SD/SR ruoli con sistema}

Il seguente diagramma aggiunge l'attore sistema oltre agli attori precedenti.
\begin{itemize}
	\item \textbf{Sistema}: questo attore rappresenta il sistema software che gestisce gli ordini, i pagamenti, le spedizioni e l'accesso a sé stesso. Esso interagisce con tutti gli altri attori per garantire il corretto funzionamento del processo di effettuazione degli ordini.
\end{itemize}

% Diagramma i* attori con sistema
\begin{figure}[ht!]
	\centering
	\includegraphics[width=0.8\columnwidth]{chapter2/immagini/2\_diagramma\_con\_sistema.png}
\end{figure}

\clearpage

\subsection{SD/SR sistema e attaccanti con alberi di attacco}
% Diagramma i* attori con alberi di attacco
\begin{figure}[ht!]	\centering
	\includegraphics[width=0.9\columnwidth]{chapter2/immagini/3\_diagramma\_con\_alberi\_di\_attacco.png}
\end{figure}
\clearpage
\subsection{SD/SR sistema e oracolo bayesiano}
% Diagramma i* attori con oracolo
\begin{figure}[ht!]
	\centering
	\includegraphics[width=1\columnwidth]{chapter2/immagini/4\_diagramma\_con\_oracolo.png}
\end{figure}
% Prima pagina: sezione + PDF in landscape



\newpage
\pdfpagewidth=297mm
\pdfpageheight=210mm
\newgeometry{margin=2cm} % solo margini

\includepdf[
  pages=1,
  fitpaper=true,
  scale=0.85,        
  pagecommand={\section{Tabella Dual Stride} %qui si può scrivere
  }
]{chapter2/dual-stride.pdf}

\restoregeometry
\pdfpagewidth=\paperwidth
\pdfpageheight=\paperheight

\newpage
\pdfpagewidth=297mm
\pdfpageheight=210mm
\newgeometry{margin=2cm} % solo margini

\includepdf[pages=2-,fitpaper=true,pagecommand={}]{chapter2/dual-stride.pdf}

\restoregeometry
\pdfpagewidth=\paperwidth
\pdfpageheight=\paperheight

\section{Abuse Case}
In questa sezione, vengono riportati gli schemi di Jacobson relativi agli abuse case. In particolare si analizzerà ogni tipologia di attacco individuato nella tabella Dual Stride.


%%%%%%%%%% Authentication Abuse (CAPEC 114) %%%%%%%%%%%%%%%
\renewcommand{\arraystretch}{1.5} % aumenta l'altezza delle righe del 50%
\begin{table}[ht!]
	\hfuzz=12pt
	\centering
	\resizebox{\textwidth}{!}{
	{\footnotesize
		\begin{tabular}{|l |l| l|}
			\hline
			\begin{minipage}[t]{3cm}\textbf{Case Type}\end{minipage} &
			\begin{minipage}[t]{6cm}\textbf{Abuse Case}\end{minipage} &
			\begin{minipage}[t]{5cm}\textbf{Case ID} AC-01 \end{minipage} \\ \hline
			
			\begin{minipage}[t]{3cm}\textbf{Case Name}\vspace{0.5em}\end{minipage} &
			\multicolumn{2}{|l|}{
				\begin{minipage}[t]{12cm}
					Authentication Abuse (CAPEC 114) 
				\vspace{0.5em}\end{minipage}
			} \\
			\hline
			
			\begin{minipage}[t]{3cm}\textbf{Actors}\vspace{0.5em}\end{minipage} &
			\multicolumn{2}{|l|}{
				\begin{minipage}[t]{12cm}
					Sistema, Attaccante
				\vspace{0.5em}\end{minipage}
			} \\
			\hline
			
			\begin{minipage}[t]{3cm}\textbf{Description}\vspace{0.5em}\end{minipage} &
			\multicolumn{2}{|l|}{
				\begin{minipage}[t]{12cm}
					Un aggressore ottiene l'accesso non autorizzato a un'applicazione, un servizio o un dispositivo conoscendo le debolezze intrinseche di un meccanismo di autenticazione o sfruttando una falla nell'implementazione dello schema di autenticazione. In un attacco di questo tipo, un meccanismo di autenticazione funziona, ma una sequenza di eventi attentamente controllata fa sì che il meccanismo conceda l'accesso all'aggressore.
				\vspace{0.5em}\end{minipage}
			} \\
			\hline
			
			\begin{minipage}[t]{3cm}\textbf{Data}\vspace{0.5em}\end{minipage} &
			\multicolumn{2}{|l|}{
				\begin{minipage}[t]{12cm}
					Dati utente di sistema
				\vspace{0.5em}\end{minipage}
			} \\
			\hline
			
			\begin{minipage}[t]{3cm}\textbf{Stimulus and Preconditions}\vspace{0.5em}\end{minipage} &
			\multicolumn{2}{|l|}{
				\begin{minipage}[t]{12cm}
					\begin{itemize}
						\item Un meccanismo o sottosistema di autenticazione che implementa una qualche forma di autenticazione, come password, certificati di sicurezza, ecc., che presenta qualche difetto/vulnerabilità;
						\item Un'applicazione client, un accesso da riga di comando a un file binario o un linguaggio di programmazione in grado di interagire con il meccanismo di autenticazione.
					\end{itemize}
				\vspace{0.5em}\end{minipage}
			} \\
			\hline
			
			\begin{minipage}[t]{3cm}\textbf{Attack Flow 1}\vspace{0.5em}\end{minipage} &
			\multicolumn{2}{|l|}{
				\begin{minipage}[t]{12cm} 
					Un attaccante sfrutta una debolezza nel meccanismo di autenticazione, permettendogli di eseguire azioni illegalmente. Per esempio, dopo aver bypassato l'autenticazione, potrebbe effettuare azioni come visionare i dati contenuti nel sistema o azioni di modifica dei dati.
				\vspace{0.5em}\end{minipage}
			} \\
			\hline
			
			\begin{minipage}[t]{3cm}\textbf{Attack Flow 2}\vspace{0.5em}\end{minipage} &
			\multicolumn{2}{|l|}{
				\begin{minipage}[t]{12cm} 
					/
				\vspace{0.5em}\end{minipage}
			} \\
			\hline
			
			\begin{minipage}[t]{3cm}\textbf{Attack Flow 3}\vspace{0.5em}\end{minipage} &
			\multicolumn{2}{|l|}{
				\begin{minipage}[t]{12cm} 
					/
				\vspace{0.5em}\end{minipage}
			} \\
			\hline
			
			\begin{minipage}[t]{3cm}\textbf{Response and Postconditions}\vspace{0.5em}\end{minipage} &
			\multicolumn{2}{|l|}{
				\begin{minipage}[t]{12cm} 
					L’aggressore ottiene l’accesso non autorizzato ed effettua azioni malevoli.
				\vspace{0.5em}\end{minipage}
			} \\
			\hline
			
			\begin{minipage}[t]{3cm}\textbf{Non Functional Requirements}\vspace{0.5em}\end{minipage} &
		\multicolumn{2}{|l|}{
			\begin{minipage}[t]{12cm}
				Garantire che tutti i meccanismi di autenticazione siano sicuri e a più fattori, includendo protezioni contro l’elevazione dei privilegi non autorizzata. Effettuare audit regolari per rilevare configurazioni errate o credenziali
				compromesse.
			\vspace{0.5em}\end{minipage}
		} \\
			\hline
			
			\begin{minipage}[t]{3cm}\textbf{Mitigations}\vspace{0.5em}\end{minipage} &
			\multicolumn{2}{|l|}{
				\begin{minipage}[t]{12cm}
					\begin{itemize}
						\item Utilizzare autenticazione MFA;
					\end{itemize}
				\vspace{0.5em}\end{minipage}
			} \\
			\hline
			
			\begin{minipage}[t]{3cm}\textbf{Comments}\vspace{0.5em}\end{minipage} &
			\multicolumn{2}{|l|}{
				\begin{minipage}[t]{12cm}
					Gli attacchi sui sistemi di autenticazione sono una delle cause principali di compromissioni. La combinazione di MFA, auditing continuo e l'uso di password robuste può ridurre significativamente il rischio.
				\vspace{0.5em}\end{minipage}
			} \\
			\hline
			
		\end{tabular}
	}}

\end{table} \clearpage

%%%%%%%%%%%%%%%%%%% Privilege Escalation (CAPEC 233) %%%%%%%%%%%%%%%%%%%%%%%%%

\begin{table}[ht!]
	\hfuzz=12pt
	\centering
	\resizebox{\textwidth}{!}{
	{\footnotesize
		\begin{tabular}{|l |l| l|}
			\hline
			\begin{minipage}[t]{3cm}\textbf{Case Type}\end{minipage} &
			\begin{minipage}[t]{6cm}\textbf{Abuse Case}\end{minipage} &
			\begin{minipage}[t]{5cm}\textbf{Case ID} AC-02 \end{minipage} \\ \hline
			
			\begin{minipage}[t]{3cm}\textbf{Case Name}\vspace{0.5em}\end{minipage} &
			\multicolumn{2}{|l|}{
				\begin{minipage}[t]{12cm}
					Privilege Escalation (CAPEC 233) 
				\vspace{0.5em}\end{minipage}
			} \\
			\hline
			
			\begin{minipage}[t]{3cm}\textbf{Actors}\vspace{0.5em}\end{minipage} &
			\multicolumn{2}{|l|}{
				\begin{minipage}[t]{12cm}
					Sistema, Attaccante
				\vspace{0.5em}\end{minipage}
			} \\
			\hline
			
			\begin{minipage}[t]{3cm}\textbf{Description}\vspace{0.5em}\end{minipage} &
			\multicolumn{2}{|l|}{
				\begin{minipage}[t]{12cm}
					Un avversario sfrutta una debolezza che gli consente di elevare i propri privilegi ed eseguire azioni che non dovrebbe essere autorizzato a eseguire.				
					\vspace{0.5em}\end{minipage}
			} \\
			\hline
			
			\begin{minipage}[t]{3cm}\textbf{Data}\vspace{0.5em}\end{minipage} &
			\multicolumn{2}{|l|}{
				\begin{minipage}[t]{12cm}
					Dati utente di sistema, Dati cliente, Dati pagamento, Dati corriere.
				\vspace{0.5em}\end{minipage}
			} \\
			\hline
			
			\begin{minipage}[t]{3cm}\textbf{Stimulus and Preconditions}\vspace{0.5em}\end{minipage} &
			\multicolumn{2}{|l|}{
				\begin{minipage}[t]{12cm}
					\begin{itemize}
        				\item Il sistema contiene un meccanismo di controllo dei privilegi mal configurato o vulnerabile
        				\item L'attaccante ha un accesso limitato al sistema.
        				\item L'attaccante conosce o riesce a dedurre una vulnerabilità sfruttabile nel controllo dei privilegi.
      				\end{itemize}
				\vspace{0.5em}\end{minipage}
			} \\
			\hline
			
			\begin{minipage}[t]{3cm}\textbf{Attack Flow 1}\vspace{0.5em}\end{minipage} &
			\multicolumn{2}{|l|}{
				\begin{minipage}[t]{12cm} 
					Gli attaccanti possono eseguire il caching sudo e/o utilizzare il file "sudoers" per elevare i propri privilegi. Ciò permette di eseguire comandi al posto di altri utenti o per generare processi con privilegi più elevati.
				\vspace{0.5em}\end{minipage}
			} \\
			\hline
			
			\begin{minipage}[t]{3cm}\textbf{Attack Flow 2}\vspace{0.5em}\end{minipage} &
			\multicolumn{2}{|l|}{
				\begin{minipage}[t]{12cm} 
					L’attaccante sfrutta la configurazione errata del meccanismo di controllo degli account utente, consentendo l’esecuzione di codice arbitrario con privilegi elevati senza richiedere autorizzazioni
				\vspace{0.5em}\end{minipage}
			} \\
			\hline
			
			\begin{minipage}[t]{3cm}\textbf{Attack Flow 3}\vspace{0.5em}\end{minipage} &
			\multicolumn{2}{|l|}{
				\begin{minipage}[t]{12cm} 
					/
				\vspace{0.5em}\end{minipage}
			} \\
			\hline
			
			\begin{minipage}[t]{3cm}\textbf{Response and Postconditions}\vspace{0.5em}\end{minipage} &
			\multicolumn{2}{|l|}{
				\begin{minipage}[t]{12cm} 
					I privilegi aumentati vengono utilizzati per eseguire operazioni non autorizzate.
				\vspace{0.5em}\end{minipage}
			} \\
			\hline
			
			\begin{minipage}[t]{3cm}\textbf{Non Functional Requirements}\vspace{0.5em}\end{minipage} &
		\multicolumn{2}{|l|}{
			\begin{minipage}[t]{12cm}
				Deve essere garantita una gestione dei privilegi sicura, proteggendo i meccanismi di escalation e prevenendo configurazioni usate. Monitoraggio continuo e auditing dei privilegi devono essere implementati per rilevare tentativi di escalation non autorizzati
			\vspace{0.5em}\end{minipage}
		} \\
			\hline
			
			\begin{minipage}[t]{3cm}\textbf{Mitigations}\vspace{0.5em}\end{minipage} &
			\multicolumn{2}{|l|}{
				\begin{minipage}[t]{12cm}
					\begin{itemize}
        				\item Applicare MFA per azioni sensibili;
        				\item Controllare periodicamente file di configurazione;
        				\item Effettuare audit periodici del sistema;
        				\item Mantenere aggiornati i programmi, le librerie e framework in uso;
      				\end{itemize}
				\vspace{0.5em}\end{minipage}
			} \\
			\hline
			
			\begin{minipage}[t]{3cm}\textbf{Comments}\vspace{0.5em}\end{minipage} &
			\multicolumn{2}{|l|}{
				\begin{minipage}[t]{12cm}
					È fondamentale combinare l'autenticazione MFA, monitorare continuamente il sistema, mantenere aggiornate le varie tecnologie in uso e gestire rigorosamente i privilegi e i file di configurazione per prevenire vulnerabilità di questo tipo.
				\vspace{0.5em}\end{minipage}
			} \\
			\hline
			
		\end{tabular}
	}}
\end{table} \clearpage

%%%%%%%%%%%%%%%%%% Targeted Malware (CAPEC 542) %%%%%%%%%%%%%%%%%%%%%%

\begin{table}[ht!]
	\hfuzz=12pt
	\centering
	\resizebox{\textwidth}{!}{
	{\footnotesize
		\begin{tabular}{|l |l| l|}
			\hline
			\begin{minipage}[t]{3cm}\textbf{Case Type}\end{minipage} &
			\begin{minipage}[t]{6cm}\textbf{Abuse Case}\end{minipage} &
			\begin{minipage}[t]{5cm}\textbf{Case ID} AC-03 \end{minipage} \\ \hline
			
			\begin{minipage}[t]{3cm}\textbf{Case Name}\vspace{0.5em}\end{minipage} &
			\multicolumn{2}{|l|}{
				\begin{minipage}[t]{12cm}
					Targeted Malware (CAPEC 542) 
				\vspace{0.5em}\end{minipage}
			} \\
			\hline
			
			\begin{minipage}[t]{3cm}\textbf{Actors}\vspace{0.5em}\end{minipage} &
			\multicolumn{2}{|l|}{
				\begin{minipage}[t]{12cm}
					Cliente, Corriere, Sistema, Attaccante
				\vspace{0.5em}\end{minipage}
			} \\
			\hline
			
			\begin{minipage}[t]{3cm}\textbf{Description}\vspace{0.5em}\end{minipage} &
			\multicolumn{2}{|l|}{
				\begin{minipage}[t]{12cm}
					Un avversario sviluppa un malware mirato che sfrutta una vulnerabilità nota in un ambiente informatico organizzativo. Il malware creato per questi attacchi si basa specificamente sulle informazioni raccolte sull'ambiente tecnologico. L'esecuzione con successo del malware consente a un avversario di ottenere un'ampia varietà di impatti tecnici negativi	
				\vspace{0.5em}\end{minipage}
			} \\
			\hline
			
			\begin{minipage}[t]{3cm}\textbf{Data}\vspace{0.5em}\end{minipage} &
			\multicolumn{2}{|l|}{
				\begin{minipage}[t]{12cm}
					Dati utente di sistema, Dati cliente, Dati pagamento, Dati corriere.
				\vspace{0.5em}\end{minipage}
			} \\
			\hline
			
			\begin{minipage}[t]{3cm}\textbf{Stimulus and Preconditions}\vspace{0.5em}\end{minipage} &
			\multicolumn{2}{|l|}{
				\begin{minipage}[t]{12cm}
					\begin{itemize}
        				\item L'attaccante deve raccogliere informazioni sull'ambiente target.
        				\item L'attaccante deve provare modi di social engineering come il phishing per far eseguire il malware sul sistema. 
        				\item Identificare vulnerabilità e creare malware ad hoc per sfruttarle.
      				\end{itemize}
				\vspace{0.5em}\end{minipage}
			} \\
			\hline
			
			\begin{minipage}[t]{3cm}\textbf{Attack Flow 1}\vspace{0.5em}\end{minipage} &
			\multicolumn{2}{|l|}{
				\begin{minipage}[t]{12cm} 
					L'attaccante analizza il sistema cercando di ottenere informazioni utili per lo sviluppo di un malware ed il suo deploy. Successivamente, dopo lo sviluppo del malware, l'attaccante tenta il deploy di quest'ultimo o tramite tecniche di social engineering come il phishing, andando ad inviare email o messaggi fasulli agli utenti interessati.
				\vspace{0.5em}\end{minipage}
			} \\
			\hline
			
			\begin{minipage}[t]{3cm}\textbf{Attack Flow 2}\vspace{0.5em}\end{minipage} &
			\multicolumn{2}{|l|}{
				\begin{minipage}[t]{12cm} 
					L'attaccante, sempre dopo aver analizzato il sistema e sviluppato un malware, sfrutta una o più vulnerabilità riscontrate nel sistema per effettuare il deploy del malware.
				\vspace{0.5em}\end{minipage}
			} \\
			\hline
			
			\begin{minipage}[t]{3cm}\textbf{Attack Flow 3}\vspace{0.5em}\end{minipage} &
			\multicolumn{2}{|l|}{
				\begin{minipage}[t]{12cm} 
					/
				\vspace{0.5em}\end{minipage}
			} \\
			\hline
			
			\begin{minipage}[t]{3cm}\textbf{Response and Postconditions}\vspace{0.5em}\end{minipage} &
			\multicolumn{2}{|l|}{
				\begin{minipage}[t]{12cm} 
					L'aggressore riesce ad eseguire malware mirati contro il sistema.
				\vspace{0.5em}\end{minipage}
			} \\
			\hline
			
			\begin{minipage}[t]{3cm}\textbf{Non Functional Requirements}\vspace{0.5em}\end{minipage} &
		\multicolumn{2}{|l|}{
			\begin{minipage}[t]{12cm}
				Garantire che il sistema effettui un monitoraggio avanzato per rilevare le minacce.
			\vspace{0.5em}\end{minipage}
				} \\
			\hline
			
			\begin{minipage}[t]{3cm}\textbf{Mitigations}\vspace{0.5em}\end{minipage} &
			\multicolumn{2}{|l|}{
				\begin{minipage}[t]{12cm}
					\begin{itemize}
        				\item Mantenere aggiornati sistemi e software;
        				\item Implementare strumenti di rilevamento come antivirus e firewall;
        				\item Effettuare degli audit del sistema andando ad analizzare manualmente processi e servizi in esecuzione.
      				\end{itemize}
				\vspace{0.5em}\end{minipage}
			} \\
			\hline
			
			\begin{minipage}[t]{3cm}\textbf{Comments}\vspace{0.5em}\end{minipage} &
			\multicolumn{2}{|l|}{
				\begin{minipage}[t]{12cm}
					Gli avversari spesso utilizzano tecniche di offuscamento quando sviluppano malware allo scopo di evitare il rilevamento o impedire al bersaglio di decodificare e comprendere un campione di malware catturato. Alcune di queste tecniche includono, ma non sono limitate a, il riempimento binario, l'imballaggio del software, la rimozione di simboli e stringhe da un payload e l'utilizzo di una risoluzione API dinamica.
				\vspace{0.5em}\end{minipage}
			} \\
			\hline
			
		\end{tabular}
	}}
\end{table} \clearpage

%%%%%%%%%%%%%%%% Phishing (CAPEC 98) %%%%%%%%%%%%%%%%%%%


\begin{table}[ht!]
	\hfuzz=12pt
	\centering
	\resizebox{\textwidth}{!}{
	{\footnotesize
		\begin{tabular}{|l |l| l|}
			\hline
			\begin{minipage}[t]{3cm}\textbf{Case Type}\end{minipage} &
			\begin{minipage}[t]{6cm}\textbf{Abuse Case}\end{minipage} &
			\begin{minipage}[t]{5cm}\textbf{Case ID} AC-04 \end{minipage} \\ \hline
			
			\begin{minipage}[t]{3cm}\textbf{Case Name}\vspace{0.5em}\end{minipage} &
			\multicolumn{2}{|l|}{
				\begin{minipage}[t]{12cm}
					Phishing (CAPEC 98) 
				\vspace{0.5em}\end{minipage}
			} \\
			\hline
			
			\begin{minipage}[t]{3cm}\textbf{Actors}\vspace{0.5em}\end{minipage} &
			\multicolumn{2}{|l|}{
				\begin{minipage}[t]{12cm}
					Cliente, Sistema, Attaccante
				\vspace{0.5em}\end{minipage}
			} \\
			\hline
			
			\begin{minipage}[t]{3cm}\textbf{Description}\vspace{0.5em}\end{minipage} &
			\multicolumn{2}{|l|}{
				\begin{minipage}[t]{12cm}
					Il phishing è una tecnica di ingegneria sociale in cui un utente malintenzionato si maschera da entità legittima con la quale la vittima potrebbe fare affari al fine di indurre l'utente a rivelare alcune informazioni riservate (molto spesso credenziali di autenticazione) che possono essere successivamente utilizzate da un malintenzionato. Il phishing è essenzialmente una forma di raccolta di informazioni o "pesca" per informazioni	
				\vspace{0.5em}\end{minipage}
			} \\
			\hline
			
			\begin{minipage}[t]{3cm}\textbf{Data}\vspace{0.5em}\end{minipage} &
			\multicolumn{2}{|l|}{
				\begin{minipage}[t]{12cm}
					Dati utente di sistema, Dati cliente, Dati pagamento, Dati corriere.
				\vspace{0.5em}\end{minipage}
			} \\
			\hline
			
			\begin{minipage}[t]{3cm}\textbf{Stimulus and Preconditions}\vspace{0.5em}\end{minipage} &
			\multicolumn{2}{|l|}{
				\begin{minipage}[t]{12cm}
					\begin{itemize}
        				\item Presenza di una vulnerabilità nota nell’ambiente target, non ancora corretta o mitigata. 
        				\item Un aggressore deve avere un modo per entrare in contatto con la vittima (per esempio tramite e-mail);
        				\item Capacità dell’attaccante di raccogliere informazioni tramite ricognizione (OSINT, scansioni passive o altre tecniche non invasive)
        				\item L'aggressore deve ottenere la fiducia della vittima, inducendola con l'inganno a compiere determinate azioni.
        				\item Il servizio ingannevole deve assomigliare il più possibile a quello reale.
      				\end{itemize}
				\vspace{0.5em}\end{minipage}
			} \\
			\hline
			
			\begin{minipage}[t]{3cm}\textbf{Attack Flow 1}\vspace{0.5em}\end{minipage} &
			\multicolumn{2}{|l|}{
				\begin{minipage}[t]{12cm} 
					Un aggressore invia un'e-mail alla vittima malevola per indurre l'utente a cliccare sul link incluso nell'e-mail (che indirizza la vittima al sito web dell'aggressore) e ad accedere. La chiave è far credere alla vittima che l'e-mail provenga da un'entità legittima e che il sito web a cui rimanda l'URL nell'e-mail sia il sito web legittimo. Un invito all'azione deve solitamente suonare legittimo e sufficientemente urgente da indurre l'utente ad agire.
				\vspace{0.5em}\end{minipage}
			} \\
			\hline
			
			\begin{minipage}[t]{3cm}\textbf{Attack Flow 2}\vspace{0.5em}\end{minipage} &
			\multicolumn{2}{|l|}{
				\begin{minipage}[t]{12cm} 
					Una volta che l'aggressore ottiene alcune informazioni sensibili tramite phishing (credenziali di accesso, dati della carta di credito, ecc.), può sfruttare queste informazioni. Ad esempio, può utilizzare le credenziali di accesso della vittima per accedere al suo conto bancario e trasferire denaro su un conto a sua scelta.
				\vspace{0.5em}\end{minipage}
			} \\
			\hline
			
			\begin{minipage}[t]{3cm}\textbf{Attack Flow 3}\vspace{0.5em}\end{minipage} &
			\multicolumn{2}{|l|}{
				\begin{minipage}[t]{12cm} 
					Un aggressore crea un sito web che assomiglia molto al sito web che sta cercando di impersonare. Tale sito web in genere include un modulo di accesso in cui la vittima deve inserire le proprie credenziali di autenticazione.
				\vspace{0.5em}\end{minipage}
			} \\
			\hline
			
			\begin{minipage}[t]{3cm}\textbf{Response and Postconditions}\vspace{0.5em}\end{minipage} &
			\multicolumn{2}{|l|}{
				\begin{minipage}[t]{12cm} 
					L'aggressore riesce ad ottenere le informazioni riservate.
				\vspace{0.5em}\end{minipage}
			} \\
			\hline
			
			\begin{minipage}[t]{3cm}\textbf{Non Functional Requirements}\vspace{0.5em}\end{minipage} &
		\multicolumn{2}{|l|}{
			\begin{minipage}[t]{12cm}
				Garantire l'implementazione di filtri avanzati, il monitoraggio e l'analisi delle attività anomale.
			\vspace{0.5em}\end{minipage}
				} \\
			\hline
			
			\begin{minipage}[t]{3cm}\textbf{Mitigations}\vspace{0.5em}\end{minipage} &
			\multicolumn{2}{|l|}{
				\begin{minipage}[t]{12cm}
					Non seguire alcun link che si riceve all'interno delle e-mail e non inserire credenziali di accesso su alcun sito web proveniente da e-mail sospette.
				\vspace{0.5em}\end{minipage}
			} \\
			\hline
			
			\begin{minipage}[t]{3cm}\textbf{Comments}\vspace{0.5em}\end{minipage} &
			\multicolumn{2}{|l|}{
				\begin{minipage}[t]{12cm}
					Questo CAPEC descrive un attacco in cui un avversario sviluppa malware su misura per sfruttare debolezze note dell’ambiente target.
				\vspace{0.5em}\end{minipage}
			} \\
			\hline
			
		\end{tabular}
	}}
\end{table} \clearpage

%%%%%%%%%%%%%% Adversary in the middle(AiTM) (CAPEC 94) %%%%%%%%%%%%%%%

% per tabelle troppo "lunghe" che strabordano in basso, allargare dimensione totale della
% tabella e usare hspace per ricentrarlo

\begin{table}[ht!]
\hfuzz=5pt % sposta la tabella verso sinistra
\centering
\resizebox{\textwidth}{!}{
{\footnotesize
\begin{tabular}{|l|l|l|}
\hline
\begin{minipage}[t]{4.5cm}\textbf{Case Type}\end{minipage} &
\begin{minipage}[t]{7cm}\textbf{Abuse Case}\end{minipage} &
\begin{minipage}[t]{4.5cm}\textbf{Case ID} AC-05\end{minipage} \\ \hline

\begin{minipage}[t]{4.5cm}\textbf{Case Name}\vspace{0.5em}\end{minipage} &
\multicolumn{2}{|l|}{
\begin{minipage}[t]{11.5cm}
Adversary in the middle (AiTM) (CAPEC 94)
\vspace{0.5em}\end{minipage}} \\ \hline

\begin{minipage}[t]{4.5cm}\textbf{Actors}\vspace{0.5em}\end{minipage} &
\multicolumn{2}{|l|}{
\begin{minipage}[t]{11.5cm}
Cliente, Sistema, Attaccante
\vspace{0.5em}\end{minipage}} \\ \hline

\begin{minipage}[t]{4.5cm}\textbf{Description}\vspace{0.5em}\end{minipage} &
\multicolumn{2}{|l|}{
\begin{minipage}[t]{11.5cm}
Ogni volta che un componente tenta di comunicare con l'altro, i dati fluiscono prima attraverso l'avversario, che può osservarli o alterarli, prima di essere trasmessi al destinatario previsto come se non fossero mai stati osservati. Questa interposizione è trasparente lasciando i due componenti compromessi inconsapevoli della potenziale corruzione o perdita delle loro comunicazioni. Il potenziale di questi attacchi produce un'implicita mancanza di fiducia nella comunicazione o nell'identificazione tra due componenti.
\vspace{0.5em}\end{minipage}} \\ \hline

\begin{minipage}[t]{4.5cm}\textbf{Data}\vspace{0.5em}\end{minipage} &
\multicolumn{2}{|l|}{
\begin{minipage}[t]{11.5cm}
Dati utente di sistema, Dati cliente, Dati pagamento, Dati corriere.
\vspace{0.5em}\end{minipage}} \\ \hline

\begin{minipage}[t]{4.5cm}\textbf{Stimulus and Preconditions}\vspace{0.5em}\end{minipage} &
\multicolumn{2}{|l|}{
\begin{minipage}[t]{11.5cm}
\begin{itemize}
\item Ci sono due componenti che comunicano tra loro;
\item Un utente malintenzionato è in grado di identificare la natura e il meccanismo di comunicazione tra i due componenti bersaglio;
\item Un utente malintenzionato può origliare la comunicazione tra i componenti bersaglio;
\item Una forte autenticazione reciproca non viene utilizzata tra i due componenti bersaglio;
\item La comunicazione avviene in chiaro (non crittografato) o con crittografia insufficiente e falsificabile.
\end{itemize}
\vspace{0.5em}\end{minipage}} \\ \hline

\begin{minipage}[t]{4.5cm}\textbf{Attack Flow 1}\vspace{0.5em}\end{minipage} &
\multicolumn{2}{|l|}{
\begin{minipage}[t]{11.5cm}
L'attaccante si interpone nella rete con uno spyware installato sul sistema. Ciò permette di registrare tutto il traffico che transita nella rete. Se il traffico non è criptato, l'attaccante può leggere tutte le comunicazioni in chiaro, altrimenti deve decriptarle.
\vspace{0.5em}\end{minipage}} \\ \hline

\begin{minipage}[t]{4.5cm}\textbf{Attack Flow 2}\vspace{0.5em}\end{minipage} &
\multicolumn{2}{|l|}{
\begin{minipage}[t]{11.5cm}
L'attaccante può "avvelenare" la cache ARP (Address Resolution Protocol) per posizionarsi tra le comunicazioni di due o più dispositivi in rete.
\vspace{0.5em}\end{minipage}} \\ \hline

\begin{minipage}[t]{4.5cm}\textbf{Attack Flow 3}\vspace{0.5em}\end{minipage} &
\multicolumn{2}{|l|}{
\begin{minipage}[t]{11.5cm}
L'attaccante può reindirizzare il traffico di rete verso sistemi di sua proprietà falsificando il traffico DHCP e comportandosi come server DHCP dannoso. Raggiungendo la posizione di "avversario nel mezzo" (AiTM), gli aggressori possono raccogliere le comunicazioni di rete, comprese le credenziali in transito tramite protocolli non sicuri.
\vspace{0.5em}\end{minipage}} \\ \hline

\begin{minipage}[t]{4.5cm}\textbf{Response and Postconditions}\vspace{0.5em}\end{minipage} &
\multicolumn{2}{|l|}{
\begin{minipage}[t]{11.5cm}
L’aggressore riesce ad inserirsi nel canale di comunicazione tra due o più componenti.
\vspace{0.5em}\end{minipage}} \\ \hline

\begin{minipage}[t]{4.5cm}\textbf{Non Functional Requirements}\vspace{0.5em}\end{minipage} &
\multicolumn{2}{|l|}{
\begin{minipage}[t]{11.5cm}
\begin{itemize}
\item Comunicazioni sicure tramite TLS 1.3, SSH, SSL.
\item Gestione sicura di chiavi, certificati, handshake.
\item Integrità dei messaggi tramite MAC o firme digitali.
\end{itemize}
\vspace{0.5em}\end{minipage}} \\ \hline

\begin{minipage}[t]{4.5cm}\textbf{Mitigations}\vspace{0.5em}\end{minipage} &
\multicolumn{2}{|l|}{
\begin{minipage}[t]{11.5cm}
\begin{itemize}
\item Chiavi pubbliche firmate da CA attendibili.
\item Crittografia del traffico (SSL/TLS/SSH).
\item Autenticazione reciproca forte.
\item Scambio sicuro delle chiavi pubbliche.
\end{itemize}
\vspace{0.5em}\end{minipage}} \\ \hline

\begin{minipage}[t]{4.5cm}\textbf{Comments}\vspace{0.5em}\end{minipage} &
\multicolumn{2}{|l|}{
\begin{minipage}[t]{11.5cm}
Il rischio di manipolazione delle transazioni tramite API è elevato quando la protezione non è adeguata, mediante attacchi AiTM. Questi attacchi differiscono dagli attacchi di sniffing perché modificano i messaggi prima che raggiungano il destinatario previsto.
\vspace{0.5em}\end{minipage}} \\ \hline

\end{tabular}
}}
\end{table} \clearpage


%%%%%%%Reusing session ID (CAPEC 60)%%%%%%%%%%%%%%%%%%%%%

\begin{table}[ht!]
	\resizebox{\textwidth}{!}{
	\centering
	{\footnotesize
		\begin{tabular}{|l |l| l|}
			\hline
			\begin{minipage}[t]{3cm}\textbf{Case Type}\end{minipage} &
			\begin{minipage}[t]{6cm}\textbf{Abuse Case}\end{minipage} &
			\begin{minipage}[t]{5cm}\textbf{Case ID} AC-06 \end{minipage} \\ \hline
			
			\begin{minipage}[t]{3cm}\textbf{Case Name}\vspace{0.5em}\end{minipage} &
			\multicolumn{2}{|l|}{
				\begin{minipage}[t]{12cm}
					Reusing session ID (CAPEC 60) 
				\vspace{0.5em}\end{minipage}
			} \\
			\hline
			
			\begin{minipage}[t]{3cm}\textbf{Actors}\vspace{0.5em}\end{minipage} &
			\multicolumn{2}{|l|}{
				\begin{minipage}[t]{12cm}
					Utente, Sistema, Attaccante
				\vspace{0.5em}\end{minipage}
			} \\
			\hline
			
			\begin{minipage}[t]{3cm}\textbf{Description}\vspace{0.5em}\end{minipage} &
			\multicolumn{2}{|l|}{
				\begin{minipage}[t]{12cm}
					Questo attacco mira al riutilizzo di un ID di sessione valido per falsificare il sistema di destinazione al fine di ottenere privilegi. L'attaccante cerca di riutilizzare un ID di sessione rubato utilizzato in precedenza durante una transazione per eseguire lo spoofing e il dirottamento della sessione.				
				\vspace{0.5em}\end{minipage}
			} \\
			\hline
			
			\begin{minipage}[t]{3cm}\textbf{Data}\vspace{0.5em}\end{minipage} &
			\multicolumn{2}{|l|}{
				\begin{minipage}[t]{12cm}
					Dati utente di sistema.
				\vspace{0.5em}\end{minipage}
			} \\
			\hline
			
			\begin{minipage}[t]{3cm}\textbf{Stimulus and Preconditions}\vspace{0.5em}\end{minipage} &
			\multicolumn{2}{|l|}{
				\begin{minipage}[t]{12cm}
					\begin{itemize}
        				\item L'host di destinazione utilizza gli ID di sessione/session token per tenere traccia degli utenti;
        				\item Gli ID di sessione/session token vengono utilizzati per controllare l'accesso alle risorse;
        				\item Gli ID di sessione/session token utilizzati dall'host di destinazione non sono ben protetti dal furto di sessione.
      				\end{itemize}
				\vspace{0.5em}\end{minipage}
			} \\
			\hline
			
			\begin{minipage}[t]{3cm}\textbf{Attack Flow 1}\vspace{0.5em}\end{minipage} &
			\multicolumn{2}{|l|}{
				\begin{minipage}[t]{12cm} 
					L'aggressore interagisce con l'host di destinazione e scopre che gli ID di sessione/token di sessione vengono utilizzati per autenticare gli utenti. Successivamente ruba un ID di sessione/token di sessione da un utente valido e lo usa per eseguire azioni per suo conto.
				\vspace{0.5em}\end{minipage}
			} \\
			\hline
			
			\begin{minipage}[t]{3cm}\textbf{Attack Flow 2}\vspace{0.5em}\end{minipage} &
			\multicolumn{2}{|l|}{
				\begin{minipage}[t]{12cm} 
				L'aggressore tenta di utilizzare l'ID di sessione rubato per ottenere l'accesso al sistema con i privilegi del proprietario originale dell'ID di sessione.	
				\vspace{0.5em}\end{minipage}
			} \\
			\hline
			
			\begin{minipage}[t]{3cm}\textbf{Attack Flow 3}\vspace{0.5em}\end{minipage} &
			\multicolumn{2}{|l|}{
				\begin{minipage}[t]{12cm} 
						/
					\vspace{0.5em}\end{minipage}
				} \\
			\hline
			
			\begin{minipage}[t]{3cm}\textbf{Response and Postconditions}\vspace{0.5em}\end{minipage} &
			\multicolumn{2}{|l|}{
				\begin{minipage}[t]{12cm} 
					L'aggressore riesce ad utilizzare lo stesso ID di sessione di un altro utente loggato nel sistema.				\vspace{0.5em}\end{minipage}
					} \\
			\hline
			
			\begin{minipage}[t]{3cm}\textbf{Non Functional Requirements}\vspace{0.5em}\end{minipage} &
		\multicolumn{2}{|l|}{
			\begin{minipage}[t]{12cm}
				Garantire una gestione sicura delle sessioni, implementando tecniche specifiche per evitare il furto di sessione.			
			\vspace{0.5em}\end{minipage}
				} \\
			\hline
			
			\begin{minipage}[t]{3cm}\textbf{Mitigations}\vspace{0.5em}\end{minipage} &
			\multicolumn{2}{|l|}{
				\begin{minipage}[t]{12cm}
					\begin{itemize}
            			\item Invalidare sempre un ID di sessione dopo il logout dell'utente.

            			\item Impostare un timeout di sessione per gli ID di sessione.

            			\item Non codificare l'ID di invio della sessione con il metodo GET, altrimenti l'ID della sessione verrà copiato nell'URL. In generale, evitare di scrivere gli ID di sessione negli URL. Gli URL possono accedere ai file di registro, che sono vulnerabili a un utente malintenzionato.

            			\item Crittografare i dati della sessione associati all'ID della sessione.

            			\item Usare l'autenticazione a più fattori.
        			\end{itemize}				
				\vspace{0.5em}\end{minipage}
			} \\
			\hline
			
			\begin{minipage}[t]{3cm}\textbf{Comments}\vspace{0.5em}\end{minipage} &
			\multicolumn{2}{|l|}{
				\begin{minipage}[t]{12cm}
					Questo attacco descrive un attacco in cui l’avversario riutilizza o intercetta un ID di sessione valido per assumere l’identità dell’utente legittimo, sfruttando debolezze nei meccanismi di gestione e protezione delle sessioni.
				\vspace{0.5em}\end{minipage}
			} \\
			\hline
			
		\end{tabular}
		}
	}
\end{table} \clearpage

%%%%%%%%%%%%% Password brute forcing (CAPEC 49)%%%%%%%%%%%%%%%%%%%%%%%%%%%%%%%

\begin{table}[ht!]
	\hfuzz=12pt
	\centering
	\resizebox{\textwidth}{!}{
	{\footnotesize
			\begin{tabular}{|l |l| l|}
			\hline
			\begin{minipage}[t]{3cm}\textbf{Case Type}\end{minipage} &
			\begin{minipage}[t]{6cm}\textbf{Abuse Case}\end{minipage} &
			\begin{minipage}[t]{5cm}\textbf{Case ID} AC-07 \end{minipage} \\ \hline
			
			\begin{minipage}[t]{3cm}\textbf{Case Name}\vspace{0.5em}\end{minipage} &
			\multicolumn{2}{|l|}{
				\begin{minipage}[t]{12cm}
					Password brute forcing (CAPEC 49) 
				\vspace{0.5em}\end{minipage}
			} \\
			\hline
			
			\begin{minipage}[t]{3cm}\textbf{Actors}\vspace{0.5em}\end{minipage} &
			\multicolumn{2}{|l|}{
				\begin{minipage}[t]{12cm}
					Sistema, Attaccante
				\vspace{0.5em}\end{minipage}
			} \\
			\hline
			
			\begin{minipage}[t]{3cm}\textbf{Description}\vspace{0.5em}\end{minipage} &
			\multicolumn{2}{|l|}{
				\begin{minipage}[t]{12cm}
					Un avversario prova ogni possibile valore per una password finché non ci riesce. Un attacco brute force passerà in rassegna tutte le password possibili, dato l'alfabeto utilizzato (lettere minuscole, lettere maiuscole, numeri, simboli, ecc.) e la lunghezza massima della password.				
				\vspace{0.5em}\end{minipage}
			} \\
			\hline
			
			\begin{minipage}[t]{3cm}\textbf{Data}\vspace{0.5em}\end{minipage} &
			\multicolumn{2}{|l|}{
				\begin{minipage}[t]{12cm}
					Dati utente di sistema.
				\vspace{0.5em}\end{minipage}
			} \\
			\hline
			
			\begin{minipage}[t]{3cm}\textbf{Stimulus and Preconditions}\vspace{0.5em}\end{minipage} &
			\multicolumn{2}{|l|}{
				\begin{minipage}[t]{12cm}
					\begin{itemize}
        				\item L'host di destinazione utilizza gli ID di sessione/session token per tenere traccia degli utenti;
        				\item Gli ID di sessione/session token vengono utilizzati per controllare l'accesso alle risorse;
        				\item Gli ID di sessione/session token utilizzati dall'host di destinazione non sono ben protetti dal furto di sessione.
      				\end{itemize}
				\vspace{0.5em}\end{minipage}
			} \\
			\hline
			
			\begin{minipage}[t]{3cm}\textbf{Attack Flow 1}\vspace{0.5em}\end{minipage} &
			\multicolumn{2}{|l|}{
				\begin{minipage}[t]{12cm} 
					L'aggressore interagisce con l'host di destinazione e scopre che gli ID di sessione/token di sessione vengono utilizzati per autenticare gli utenti. Successivamente ruba un ID di sessione/token di sessione da un utente valido e lo usa per eseguire azioni per suo conto.
				\vspace{0.5em}\end{minipage}
			} \\
			\hline
			
			\begin{minipage}[t]{3cm}\textbf{Attack Flow 2}\vspace{0.5em}\end{minipage} &
			\multicolumn{2}{|l|}{
				\begin{minipage}[t]{12cm} 
					L'aggressore tenta di utilizzare l'ID di sessione rubato per ottenere l'accesso al sistema con i privilegi del proprietario originale dell'ID di sessione.				
				\vspace{0.5em}\end{minipage}
			} \\
			\hline
			
			\begin{minipage}[t]{3cm}\textbf{Attack Flow 3}\vspace{0.5em}\end{minipage} &
			\multicolumn{2}{|l|}{
				\begin{minipage}[t]{12cm} 
						/
					\vspace{0.5em}\end{minipage}
				} \\
			\hline
			
			\begin{minipage}[t]{3cm}\textbf{Response and Postconditions}\vspace{0.5em}\end{minipage} &
			\multicolumn{2}{|l|}{
				\begin{minipage}[t]{12cm} 
					L'aggressore riesce ad utilizzare lo stesso ID di sessione di un altro utente loggato nel sistema.				
				\vspace{0.5em}\end{minipage}
					} \\
			\hline
			
			\begin{minipage}[t]{3cm}\textbf{Non Functional Requirements}\vspace{0.5em}\end{minipage} &
		\multicolumn{2}{|l|}{
			\begin{minipage}[t]{12cm}
				Garantire una gestione sicura delle sessioni, implementando tecniche specifiche per evitare il furto di sessione.			
			\vspace{0.5em}\end{minipage}
				} \\
			\hline
			
			\begin{minipage}[t]{3cm}\textbf{Mitigations}\vspace{0.5em}\end{minipage} &
			\multicolumn{2}{|l|}{
				\begin{minipage}[t]{12cm}
					\begin{itemize}
            			\item Invalidare sempre un ID di sessione dopo il logout dell'utente.

            			\item Impostare un timeout di sessione per gli ID di sessione.

            			\item Non codificare l'ID di invio della sessione con il metodo GET, altrimenti l'ID della sessione verrà copiato nell'URL. In generale, evitare di scrivere gli ID di sessione negli URL. Gli URL possono accedere ai file di registro, che sono vulnerabili a un utente malintenzionato.

            			\item Crittografare i dati della sessione associati all'ID della sessione.

            			\item Usare l'autenticazione a più fattori.
        			\end{itemize}				
				\vspace{0.5em}\end{minipage}
			} \\
			\hline
			
			\begin{minipage}[t]{3cm}\textbf{Comments}\vspace{0.5em}\end{minipage} &
			\multicolumn{2}{|l|}{
				\begin{minipage}[t]{12cm}
					Questo attacco descrive un attacco in cui l’avversario riutilizza o intercetta un ID di sessione valido per assumere l’identità dell’utente legittimo, sfruttando debolezze nei meccanismi di gestione e protezione delle sessioni.
				\vspace{0.5em}\end{minipage}
			} \\
			\hline
			
		\end{tabular}
	}}
\end{table} \clearpage
%%%%%%%%%%%%%%%%%% Try common or default usernames and passwords (CAPEC 70) %%%%%%%%%%%%%%%%%%%%%%%%%%

\begin{table}[ht!]
	\hfuzz=12pt
	\centering
	\resizebox{\textwidth}{!}{
	{\footnotesize
		\begin{tabular}{|l |l| l|}
			\hline
			\begin{minipage}[t]{3cm}\textbf{Case Type}\end{minipage} &
			\begin{minipage}[t]{6cm}\textbf{Abuse Case}\end{minipage} &
			\begin{minipage}[t]{5cm}\textbf{Case ID} AC-08 \end{minipage} \\ \hline
			
			\begin{minipage}[t]{3cm}\textbf{Case Name}\vspace{0.5em}\end{minipage} &
			\multicolumn{2}{|l|}{
				\begin{minipage}[t]{12cm}
					Try common or default usernames and passwords (CAPEC 70) 
				\vspace{0.5em}\end{minipage}
			} \\
			\hline
			
			\begin{minipage}[t]{3cm}\textbf{Actors}\vspace{0.5em}\end{minipage} &
			\multicolumn{2}{|l|}{
				\begin{minipage}[t]{12cm}
					Sistema, Attaccante
				\vspace{0.5em}\end{minipage}
			} \\
			\hline
			
			\begin{minipage}[t]{3cm}\textbf{Description}\vspace{0.5em}\end{minipage} &
			\multicolumn{2}{|l|}{
				\begin{minipage}[t]{12cm}
					Un avversario può provare alcuni nomi utente e password comuni o predefiniti per ottenere l'accesso al sistema ed eseguire azioni non autorizzate. Un avversario può provare una forza bruta intelligente usando password vuote, credenziali predefinite del fornitore note, nonché un dizionario di nomi utente e password comuni. Molti prodotti del fornitore sono preconfigurati con nomi utente e password predefiniti (e quindi ben noti) che dovrebbero essere eliminati prima dell'utilizzo in un ambiente di produzione. È un errore comune dimenticare di rimuovere queste credenziali di accesso predefinite. Un altro problema è che gli utenti sceglierebbero password molto semplici (comuni) (ad esempio "12345" o "password") che rendono più facile per l'attaccante ottenere l'accesso al sistema rispetto all'utilizzo di un attacco di forza bruta o anche di un attacco dizionario utilizzando un dizionario completo.				
				\vspace{0.5em}\end{minipage}
			} \\
			\hline
			
			\begin{minipage}[t]{3cm}\textbf{Data}\vspace{0.5em}\end{minipage} &
			\multicolumn{2}{|l|}{
				\begin{minipage}[t]{12cm}
					Dati utente di sistema.
				\vspace{0.5em}\end{minipage}
			} \\
			\hline
			
			\begin{minipage}[t]{3cm}\textbf{Stimulus and Preconditions}\vspace{0.5em}\end{minipage} &
			\multicolumn{2}{|l|}{
				\begin{minipage}[t]{12cm}
					\begin{itemize}
        				\item Il sistema utilizza l'autenticazione basata su password a un fattore. L'avversario ha i mezzi per interagire con il sistema;
        				\item Elenco specifico della tecnologia o del fornitore di nomi utente e password predefiniti.
      				\end{itemize}
				\vspace{0.5em}\end{minipage}
			} \\
			\hline
			
			\begin{minipage}[t]{3cm}\textbf{Attack Flow 1}\vspace{0.5em}\end{minipage} &
			\multicolumn{2}{|l|}{
				\begin{minipage}[t]{12cm} 
					Un utente imposta la propria password su "password" o la lascia intenzionalmente vuota. Se il sistema non dispone di un'applicazione della forza della password rispetto a una politica di password valida, questa password potrebbe essere ammessa. L'attaccante tenta di utilizzare una serie di nomi utente e password comuni, tra cui "password" ed ottiene l'accesso al sistema.
				\vspace{0.5em}\end{minipage}
			} \\
			\hline
			
			\begin{minipage}[t]{3cm}\textbf{Attack Flow 2}\vspace{0.5em}\end{minipage} &
			\multicolumn{2}{|l|}{
				\begin{minipage}[t]{12cm} 
					L'attaccante analizza il sistema per identificare le credenziali predefinite del fornitore. Successivamente, l'attaccante tenta di utilizzare queste credenziali per ottenere l'accesso al sistema.				
				\vspace{0.5em}\end{minipage}
			} \\
			\hline
			
			\begin{minipage}[t]{3cm}\textbf{Attack Flow 3}\vspace{0.5em}\end{minipage} &
			\multicolumn{2}{|l|}{
				\begin{minipage}[t]{12cm} 
						/
					\vspace{0.5em}\end{minipage}
				} \\
			\hline
			
			\begin{minipage}[t]{3cm}\textbf{Response and Postconditions}\vspace{0.5em}\end{minipage} &
			\multicolumn{2}{|l|}{
				\begin{minipage}[t]{12cm} 
					L'aggressore riesce ad ottenere mediante la prova di password comuni entrando nel sistema.				
				\vspace{0.5em}\end{minipage}
					} \\
			\hline
			
			\begin{minipage}[t]{3cm}\textbf{Non Functional Requirements}\vspace{0.5em}\end{minipage} &
		\multicolumn{2}{|l|}{
			\begin{minipage}[t]{12cm}
				Controllare l'input della password e dello username nella registrazione di un utente, obbligandolo a non utilizzare password o username comuni.			
			\vspace{0.5em}\end{minipage}
				} \\
			\hline
			
			\begin{minipage}[t]{3cm}\textbf{Mitigations}\vspace{0.5em}\end{minipage} &
			\multicolumn{2}{|l|}{
				\begin{minipage}[t]{12cm}
					\begin{itemize}
            			 \item Elimina tutte le credenziali predefinite dell'account che possono essere inserite dal fornitore del prodotto.

        \item Implementare un meccanismo di limitazione delle password. Questo meccanismo dovrebbe prendere in considerazione sia l'indirizzo IP che il nome di accesso dell'utente.

        \item Metti insieme una politica di password forte e assicurati che tutte le password create dagli utenti siano conformi. In alternativa, genera automaticamente password complesse per gli utenti.

        \item Le password devono essere riciclate per prevenire l'invecchiamento, cioè ogni tanto deve essere scelta una nuova password.
        			\end{itemize}				
				\vspace{0.5em}\end{minipage}
			} \\
			\hline
			
			\begin{minipage}[t]{3cm}\textbf{Comments}\vspace{0.5em}\end{minipage} &
			\multicolumn{2}{|l|}{
				\begin{minipage}[t]{12cm}
					Un utente imposta la propria password su "123" o lascia intenzionalmente la sua password vuota. Se il sistema non ha il controllo della password rispetto a una solida politica di password, questa password può essere ammessa. Password come questi due esempi sono due password semplici e comuni che possono essere facilmente indovinate dall'avversario.
				\vspace{0.5em}\end{minipage}
			} \\
			\hline
			
		\end{tabular}
	}}
\end{table} \clearpage


%%%%%%%%%%%%%%%%%% Dictionary-based password attack (CAPEC 16) %%%%%%%%%%%%%%%%%%%%%%%%%%

\begin{table}[ht!]
	\hfuzz=12pt
	\centering
	{\footnotesize
		\begin{tabular}{|l |l| l|}
			\hline
			\begin{minipage}[t]{3cm}\textbf{Case Type}\end{minipage} &
			\begin{minipage}[t]{6cm}\textbf{Abuse Case}\end{minipage} &
			\begin{minipage}[t]{5cm}\textbf{Case ID} AC-09 \end{minipage} \\ \hline
			
			\begin{minipage}[t]{3cm}\textbf{Case Name}\vspace{0.5em}\end{minipage} &
			\multicolumn{2}{|l|}{
				\begin{minipage}[t]{12cm}
					Dictionary-based password attack (CAPEC 16) 
				\vspace{0.5em}\end{minipage}
			} \\
			\hline
			
			\begin{minipage}[t]{3cm}\textbf{Actors}\vspace{0.5em}\end{minipage} &
			\multicolumn{2}{|l|}{
				\begin{minipage}[t]{12cm}
					Sistema, Attaccante
				\vspace{0.5em}\end{minipage}
			} \\
			\hline
			
			\begin{minipage}[t]{3cm}\textbf{Description}\vspace{0.5em}\end{minipage} &
			\multicolumn{2}{|l|}{
				\begin{minipage}[t]{12cm}
					Un utente malintenzionato prova ciascuna delle parole in un dizionario come password per ottenere l'accesso al sistema tramite l'account di un utente. Se la password scelta dall'utente era una parola all'interno del dizionario, questo attacco avrà successo (in assenza di altre mitigazioni). Questa è un'istanza specifica del modello di attacco di forzatura bruta della password.

    Gli attacchi del dizionario differiscono da attacchi simili come Password Spraying (CAPEC-565) e Credential Stuffing (CAPEC-600), poiché sfruttano combinazioni sconosciute di nome utente/password e non si preoccupano di indurre blocchi dell'account.				
				\vspace{0.5em}\end{minipage}
			} \\
			\hline
			
			\begin{minipage}[t]{3cm}\textbf{Data}\vspace{0.5em}\end{minipage} &
			\multicolumn{2}{|l|}{
				\begin{minipage}[t]{12cm}
				Dati utente di sistema.
				\vspace{0.5em}\end{minipage}
			} \\
			\hline
			
			\begin{minipage}[t]{3cm}\textbf{Stimulus and Preconditions}\vspace{0.5em}\end{minipage} &
			\multicolumn{2}{|l|}{
				\begin{minipage}[t]{12cm}
					\begin{itemize}
        \item Il sistema utilizza l'autenticazione basata su password a un fattore;

        \item Il sistema non ha una politica di password solida che viene applicata;

        \item Il sistema non implementa un meccanismo di limitazione della password efficace;
        
      \end{itemize}
				\vspace{0.5em}\end{minipage}
			} \\
			\hline
			
			\begin{minipage}[t]{3cm}\textbf{Attack Flow 1}\vspace{0.5em}\end{minipage} &
			\multicolumn{2}{|l|}{
				\begin{minipage}[t]{12cm} 
					Determinare i criteri di password dell'applicazione/sistema di destinazione, come ad esempio la lunghezza consentita e il formato richiesto
				\vspace{0.5em}\end{minipage}
			} \\
			\hline
			
			\begin{minipage}[t]{3cm}\textbf{Attack Flow 2}\vspace{0.5em}\end{minipage} &
			\multicolumn{2}{|l|}{
				\begin{minipage}[t]{12cm} 
					Determinare i nomi utente di cui decifrare le password, ad esempio attraverso lo sniffing della rete, interrogando il sistema oppure dal file system			
				\vspace{0.5em}\end{minipage}
			} \\
			\hline
			
			\begin{minipage}[t]{3cm}\textbf{Attack Flow 3}\vspace{0.5em}\end{minipage} &
			\multicolumn{2}{|l|}{
				\begin{minipage}[t]{12cm} 
						Utilizza uno strumento di deciframento delle password che sfrutti il dizionario per fornire le password al sistema e verificare se funzionano.
					\vspace{0.5em}\end{minipage}
				} \\
			\hline
			
			\begin{minipage}[t]{3cm}\textbf{Response and Postconditions}\vspace{0.5em}\end{minipage} &
			\multicolumn{2}{|l|}{
				\begin{minipage}[t]{12cm} 
						L’aggressore ottiene l’accesso non autorizzato ed effettua azioni
    					malevole.			
				\vspace{0.5em}\end{minipage}
					} \\
			\hline
			
			\begin{minipage}[t]{3cm}\textbf{Non Functional Requirements}\vspace{0.5em}\end{minipage} &
		\multicolumn{2}{|l|}{
			\begin{minipage}[t]{12cm}
						Controllo da parte del sistema di obbligare l'utente a non utilizzare come password parole prese dal dizionario, effettuando un controllo di tutte le parole comuni.
			\vspace{0.5em}\end{minipage}
				} \\
			\hline
			
			\begin{minipage}[t]{3cm}\textbf{Mitigations}\vspace{0.5em}\end{minipage} &
			\multicolumn{2}{|l|}{
				\begin{minipage}[t]{12cm}
						\begin{itemize}
        \item Crea una politica di password complessa e assicurati che il tuo sistema applichi questa politica;

        \item Implementare un meccanismo intelligente di limitazione delle password. È necessario fare attenzione a garantire che questi meccanismi non consentano eccessivamente gli attacchi di blocco dell'account come CAPEC-2;

        \item Sfrutta l'autenticazione a più fattori per tutti i servizi di autenticazione.
      \end{itemize}			
				\vspace{0.5em}\end{minipage}
			} \\
			\hline
			
			\begin{minipage}[t]{3cm}\textbf{Comments}\vspace{0.5em}\end{minipage} &
			\multicolumn{2}{|l|}{
				\begin{minipage}[t]{12cm}
					Gli attacchi all’autenticazione sono una delle cause principali di compro
    missioni. La combinazione di MFA, auditing continuo e training degli utenti può ridurre significativamente il rischio.
				\vspace{0.5em}\end{minipage}
			} \\
			\hline
		\end{tabular}
	}
\end{table} \clearpage


%%%%%%%%%%%%%%%%%% Carry-Off GPS Attack (CAPEC 628) %%%%%%%%%%%%%%%%%%%%%%%%%%

\begin{table}[ht!]
	\hfuzz=20pt
    \centering
	\resizebox{\textwidth}{!}{
    {\footnotesize
    \begin{tabular}{|l|l|l|}
        \hline
        \begin{minipage}[t]{3cm}\textbf{Case Type}\end{minipage} &
        \begin{minipage}[t]{7cm}\textbf{Abuse Case}\end{minipage} &
        \begin{minipage}[t]{7cm}\textbf{Case ID} AC-10\end{minipage} \\ \hline
        
        \begin{minipage}[t]{3cm}\textbf{Case Name}\end{minipage} &
        \multicolumn{2}{|l|}{%
            \begin{minipage}[t]{14cm}
            Carry-Off GPS Attack (CAPEC 628)
            \end{minipage}%
        } \\ \hline
        
        \begin{minipage}[t]{3cm}\textbf{Actors}\end{minipage} &
        \multicolumn{2}{|l|}{%
            \begin{minipage}[t]{14cm}
            Sistema, Attaccante
            \end{minipage}%
        } \\ \hline
        
        \begin{minipage}[t]{3cm}\textbf{Description}\end{minipage} &
        \multicolumn{2}{|l|}{%
            \begin{minipage}[t]{14cm}
            Una forma comune di attacco di spoofing GPS, comunemente definito attacco carry-off, inizia con la trasmissione da parte di un avversario di segnali sincronizzati con i segnali autentici rilevati dal ricevitore bersaglio. La potenza dei segnali contraffatti viene quindi gradualmente aumentata e allontanata dai segnali autentici. Col tempo, l'avversario può allontanare il bersaglio dalla destinazione prevista e dirigerlo verso una posizione scelta dall'avversario.
            \vspace{0.5em}\end{minipage}%
        } \\ \hline
        
        \begin{minipage}[t]{3cm}\textbf{Data}\end{minipage} &
        \multicolumn{2}{|l|}{%
            \begin{minipage}[t]{14cm}
            	Dati GPS.
            \end{minipage}%
        } \\ \hline
        
        \begin{minipage}[t]{3cm}\textbf{Stimulus and Preconditions}\vspace{0.5em}\end{minipage} &
        \multicolumn{2}{|l|}{%
            \begin{minipage}[t]{14cm}
            Per eseguire operazioni critiche, il bersaglio deve poter contare su un segnale GPS valido.
            \end{minipage}%
        } \\ \hline
        
        \begin{minipage}[t]{3cm}\textbf{Attack Flow 1}\end{minipage} &
        \multicolumn{2}{|l|}{%
            \begin{minipage}[t]{14cm}
            L’avversario crea una copia quasi indistinguibile del segnale GPS reale, mantenendo una potenza quasi identica per evitare rilevazioni. Dopo aver ottenuto il controllo del segnale, modifica la posizione in modo estremamente graduale, così che la deviazione sembri naturale e non venga percepita come anomala.
            \vspace{0.5em}\end{minipage}%
        } \\ \hline
        
        \begin{minipage}[t]{3cm}\textbf{Attack Flow 2}\end{minipage} &
        \multicolumn{2}{|l|}{%
            \begin{minipage}[t]{14cm}
            L’attaccante replica i segnali GPS autentici del bersaglio, li sincronizza perfettamente e inizia a trasmetterli con una potenza leggermente superiore. Man mano che il ricevitore inizia a fidarsi dei segnali falsi, l’avversario introduce un lento spostamento della posizione stimata, trascinando il bersaglio lontano dal percorso reale verso una destinazione scelta.
            \vspace{0.5em}\end{minipage}%
        } \\ \hline
        
        \begin{minipage}[t]{3cm}\textbf{Attack Flow 3}\end{minipage} &
        \multicolumn{2}{|l|}{%
            \begin{minipage}[t]{14cm} /
            \end{minipage}%
        } \\ \hline
        
        \begin{minipage}[t]{3cm}\textbf{Response and Postconditions}\end{minipage} &
        \multicolumn{2}{|l|}{%
            \begin{minipage}[t]{14cm}
            \begin{itemize}
                \item Il sistema nota incoerenze o potenza anomala nel segnale e sospetta spoofing.
                \item Il ricevitore ignora il segnale sospetto e passa a una modalità di navigazione sicura.
                \item Il segnale falso viene ignorato, la posizione rimane corretta e la navigazione resta affidabile.
                \item Il sistema accetta il segnale spoofato, la posizione diventa errata e il bersaglio viene deviato verso una destinazione scelta dall’attaccante.
            \end{itemize}
            \vspace{0.5em}\end{minipage}%
        } \\ \hline
        
        \begin{minipage}[t]{3cm}\textbf{Non Functional Requirements}\vspace{0.5em}\end{minipage} &
        \multicolumn{2}{|l|}{%
            \begin{minipage}[t]{14cm} 
				\begin{itemize}
					\item Il sistema deve mantenere una posizione coerente anche in presenza di segnali GPS anomali o degradati.
					\item Il ricevitore deve essere in grado di continuare a funzionare correttamente anche se il GPS viene spoofato o manipolato.
					\item Il sistema deve reagire rapidamente quando il segnale GPS mostra comportamenti sospetti, riducendo al minimo la deviazione potenziale.
					\item Il sistema deve garantire continuità operativa tramite sensori ridondanti quando il GPS è inattendibile.
					\item La piattaforma deve verificare che le informazioni di navigazione siano coerenti e non violate, prevenendo modifiche silenziose alla posizione.
				\end{itemize}
            \vspace*{0.5em}\end{minipage}%
        } \\ \hline
        
        \begin{minipage}[t]{3cm}\textbf{Mitigations}\end{minipage} &
        \multicolumn{2}{|l|}{%
            \begin{minipage}[t]{14cm} 
				\begin{itemize}
					\item Integrare dati non‑GPS per verificare e correggere la posizione quando il segnale appare sospetto.
					\item Utilizzare segnali autenticati.
					\item Escludere automaticamente il GPS dalla navigazione quando rilevato come inattendibile.
					\item Passare a una modalità di guida/navigazione degradata ma affidabile quando il GPS risulta compromesso.
				\end{itemize}
            \vspace*{0.5em}\end{minipage}%
        } \\ \hline
        
        \begin{minipage}[t]{3cm}\textbf{Comments}\end{minipage} &
        \multicolumn{2}{|l|}{%
            \begin{minipage}[t]{14cm} 
				Un attacco "proof-of-concept" è stato eseguito con successo nel giugno 2013, quando lo yacht di lusso "White Rose" è stato deviato con segnali GPS falsificati da Monaco all'isola di Rodi da un gruppo di studenti di ingegneria aerospaziale della Cockrell School of Engineering dell'Università del Texas ad Austin. Gli studenti erano a bordo dello yacht e hanno permesso alle loro apparecchiature di spoofing di sopraffare gradualmente la potenza del segnale dei satelliti della costellazione GPS effettiva, alterando la rotta dello yacht.
            \vspace*{0.5em}\end{minipage}%
        } \\ \hline
    \end{tabular}
    }}
\end{table} \clearpage

%%%%%%%%%%%%%%%%%% Credential stuffing (CAPEC 600)  %%%%%%%%%%%%%%%%%%%%%%%%%%
\begin{table}[ht!]

	\hfuzz=42pt % sposta la tabella verso sinistra
	\centering
	\resizebox{\textwidth}{!}{
	{\footnotesize
		\begin{tabular}{|l |l| l|}
			\hline
			\begin{minipage}[t]{2cm}\textbf{Case Type}\end{minipage} &
			\begin{minipage}[t]{12cm}\textbf{Abuse Case}\end{minipage} &
			\begin{minipage}[t]{4cm}\textbf{Case ID} AC-11 \end{minipage} \\ \hline
			
			\begin{minipage}[t]{2cm}\textbf{Case Name}\vspace{0.5em}\end{minipage} &
			\multicolumn{2}{|l|}{
				\begin{minipage}[t]{17cm}
					Credential stuffing (CAPEC 600)
				\vspace{0.5em}\end{minipage}
			} \\
			\hline
			
			\begin{minipage}[t]{2cm}\textbf{Actors}\vspace{0.5em}\end{minipage} &
			\multicolumn{2}{|l|}{
				\begin{minipage}[t]{17cm}
					 Sistema, Attaccante
				\vspace{0.5em}\end{minipage}
			} \\
			\hline
			
			\begin{minipage}[t]{2cm}\textbf{Description}\vspace{0.5em}\end{minipage} &
			\multicolumn{2}{|l|}{
				\begin{minipage}[t]{17cm}
	Un avversario prova combinazioni di nome utente/password note su diversi sistemi, applicazioni o servizi per ottenere un accesso autenticato aggiuntivo. Gli attacchi di Credential Stuffing si basano sul fatto che molti utenti sfruttano la stessa combinazione di nome utente/password per più sistemi, applicazioni e servizi.				\vspace{0.5em}\end{minipage}
			} \\
			\hline
			
			\begin{minipage}[t]{2cm}\textbf{Data}\vspace{0.5em}\end{minipage} &
			\multicolumn{2}{|l|}{
				\begin{minipage}[t]{17cm}
					Dati utente di sistema.
				\vspace{0.5em}\end{minipage}
			} \\
			\hline
			
			\begin{minipage}[t]{2cm}\textbf{Stimulus and Preconditions}\vspace{0.5em}\end{minipage} &
			\multicolumn{2}{|l|}{
				\begin{minipage}[t]{17cm}
					\begin{itemize}
        				\item Il sistema/applicazione utilizza l'autenticazione basata su password a un fattore, SSO e/o l'autenticazione basata su cloud;

        \item Il sistema/applicazione non ha una solida politica di password che viene applicata;

        \item Il sistema/applicazione non implementa un meccanismo di limitazione della password efficace;

        \item L'avversario possiede un elenco di account utente noti e password corrispondenti che potrebbero esistere sul bersaglio;
      	
		\end{itemize}
				\vspace{0.5em}\end{minipage}
			} \\
			\hline
			
			\begin{minipage}[t]{2cm}\textbf{Attack Flow 1}\vspace{0.5em}\end{minipage} &
				\multicolumn{2}{|l|}{
				\begin{minipage}[t]{17cm} 
				L'avversario deve ottenere credenziali note per poter accedere al sistema, all'applicazione o al servizio di destinazione.				\vspace{0.5em}\end{minipage}
				}\\
			\hline
			
			\begin{minipage}[t]{2cm}\textbf{Attack Flow 2}\vspace{0.5em}\end{minipage} &
			\multicolumn{2}{|l|}{
				\begin{minipage}[t]{17cm} 
					Determinare i criteri delle password del sistema/applicazione di destinazione per determinare se le credenziali note rientrano nei criteri specificati.				
					\vspace{0.5em}\end{minipage}
			} \\
			\hline
			
			\begin{minipage}[t]{2cm}\textbf{Attack Flow 3}\vspace{0.5em}\end{minipage} &
			\multicolumn{2}{|l|}{
				\begin{minipage}[t]{17cm} 
						Prova ogni combinazione nome utente/password finché la destinazione non concede l'accesso.
					\vspace{0.5em}\end{minipage}
				} \\
			\hline
			
			\begin{minipage}[t]{2cm}
\textbf{Response and Postconditions}\vspace{0.5em}\end{minipage} &
			\multicolumn{2}{|l|}{
				\begin{minipage}[t]{17cm} 
					L’aggressore ottiene l’accesso non autorizzato ed effettua azioni malevole.				\vspace{0.5em}\end{minipage}
					} \\
			\hline
			
			\begin{minipage}[t]{2cm}
\textbf{Non Functional Requirements}\vspace{0.5em}\end{minipage} &
		\multicolumn{2}{|l|}{
			\begin{minipage}[t]{17cm}
				Garantire il monitoraggio dei registri di sistema e l'implementazione dell'autenticazione a più fattori.			
			\vspace{0.5em}\end{minipage}
				} \\
			\hline
			
			\begin{minipage}[t]{2cm}\textbf{Mitigations}\vspace{0.5em}\end{minipage} &
			\multicolumn{2}{|l|}{
				\begin{minipage}[t]{17cm}
					\begin{itemize}
            			\item Utilizzare l'autenticazione a più fattori per tutti i servizi di autenticazione e prima di concedere a un'entità l'accesso alla rete del dominio.

        \item Crea una politica di password complessa e assicurarsi che il sistema applichi questa politica;

        \item Assicurarsi che gli utenti non riutilizzino combinazioni nome utente/password per più sistemi, applicazioni o servizi.

        \item Non riutilizzare le credenziali dell'account amministratore locale su tutti i sistemi;

        \item Nega l'uso remoto delle credenziali di amministratore locali per accedere ai sistemi di dominio;

        \item Non consentire agli account di essere un amministratore locale su più di un sistema;

        \item Implementare un meccanismo intelligente di limitazione delle password. È necessario fare attenzione a garantire che questi meccanismi non consentano eccessivamente gli attacchi di blocco dell'account come CAPEC-2;

        \item Monitorare i registri di sistema e di dominio per l'accesso anomalo alle credenziali.
        			\end{itemize}				
				\vspace{0.5em}\end{minipage}
			} \\
			\hline
			
			\begin{minipage}[t]{2cm}\textbf{Comments}\vspace{0.5em}\end{minipage} &
			\multicolumn{2}{|l|}{
				\begin{minipage}[t]{17cm}
					Un utente sfrutta la password "Password123" per una manciata di accessi alle applicazioni. Un avversario ottiene la combinazione nome utente/password di una vittima da una violazione di un'applicazione di social media ed esegue un attacco Credential Stuffing contro più applicazioni bancarie e di carte di credito. Poiché l'utente sfrutta le stesse credenziali per l'accesso al proprio conto bancario, l'avversario si autentica con successo sul conto bancario dell'utente e trasferisce denaro su un conto offshore.
				\vspace{0.5em}\end{minipage}
			} \\
			\hline
			
		\end{tabular}
		}}
\end{table} \clearpage

%%%%%%%%%%%%%%%%%% Traffic Injection (CAPEC 594) %%%%%%%%%%%%%%%%%%%%%%%%%%
\begin{table}[ht]
	\hfuzz=12pt
	\resizebox{\textwidth}{!}{
	\centering
	{\footnotesize
		\begin{tabular}{|l |l| l|}
			\hline
			\begin{minipage}[t]{3cm}\textbf{Case Type}\end{minipage} &
			\begin{minipage}[t]{6cm}\textbf{Abuse Case}\end{minipage} &
			\begin{minipage}[t]{5cm}\textbf{Case ID} AC-12 \end{minipage} \\ \hline
			
			\begin{minipage}[t]{3cm}\textbf{Case Name}\vspace{0.5em}\end{minipage} &
			\multicolumn{2}{|l|}{
				\begin{minipage}[t]{12cm}
					Traffic Injection (CAPEC 594)
				\vspace{0.5em}\end{minipage}
			} \\
			\hline
			
			\begin{minipage}[t]{3cm}\textbf{Actors}\vspace{0.5em}\end{minipage} &
			\multicolumn{2}{|l|}{
				\begin{minipage}[t]{12cm}
					 Sistema, Attaccante
				\vspace{0.5em}\end{minipage}
			} \\
			\hline
			\begin{minipage}[t]{3cm}\textbf{Description}\vspace{0.5em}\end{minipage} &
			\multicolumn{2}{|l|}{
				\begin{minipage}[t]{11.5cm}
	Un avversario inietta traffico nella connessione di rete del bersaglio. L'avversario è quindi in grado di degradare o interrompere la connessione e potenzialmente modificarne il contenuto. Non si tratta di un attacco flooding, poiché l'avversario non si concentra sull'esaurimento delle risorse. Piuttosto, sta elaborando un input specifico per influenzare il sistema in un modo particolare.		\vspace{0.5em}\end{minipage}
			} \\
			\hline
			
			\begin{minipage}[t]{3cm}\textbf{Data}\vspace{0.5em}\end{minipage} &
			\multicolumn{2}{|l|}{
				\begin{minipage}[t]{12cm}
				Dati utente di sistema, Dati cliente, Dati pagamento, Dati corriere, Dati GPS.
				\vspace{0.5em}\end{minipage}
			} \\
			\hline
			
			\begin{minipage}[t]{3cm}\textbf{Stimulus and Preconditions}\vspace{0.5em}\end{minipage} &
			\multicolumn{2}{|l|}{
				\begin{minipage}[t]{11.5cm}
					\begin{itemize}
        				\item L'applicazione di destinazione deve sfruttare un canale di comunicazione aperto.
        \item Il canale su cui comunica il bersaglio deve essere vulnerabile all'intercettazione.
      				\end{itemize}
				\vspace{0.5em}\end{minipage}
			} \\
			\hline
			
			\begin{minipage}[t]{3cm}\textbf{Attack Flow 1}\vspace{0.5em}\end{minipage} &
				\multicolumn{2}{|l|}{
				\begin{minipage}[t]{11.5cm}
					L'attaccante intercetta il traffico di rete tra due endpoint comunicanti. Utilizzando tecniche come la spoofing degli indirizzi IP o ARP, l'attaccante si posiziona come intermediario nella comunicazione. Successivamente, l'attaccante inietta pacchetti di dati malevoli o manipolati nel flusso di traffico esistente, alterando il contenuto delle comunicazioni tra gli endpoint.
				\vspace{0.5em}\end{minipage}
			} 
			\\
			\hline
			
			\begin{minipage}[t]{3cm}\textbf{Attack Flow 2}\vspace{0.5em}\end{minipage} &
			\multicolumn{2}{|l|}{
				\begin{minipage}[t]{11.5cm} 
					L'attaccante monitora il traffico di rete per identificare pacchetti specifici da iniettare. Utilizzando strumenti di analisi del traffico, l'attaccante seleziona i momenti opportuni per inserire i pacchetti malevoli, cercando di evitare rilevamenti e garantire che i pacchetti iniettati vengano accettati dagli endpoint di destinazione.		
					\vspace{0.5em}\end{minipage}
			} \\
			\hline
			
			\begin{minipage}[t]{3cm}\textbf{Attack Flow 3}\vspace{0.5em}\end{minipage} &
			\multicolumn{2}{|l|}{
				\begin{minipage}[t]{12cm} 
						/
					\vspace{0.5em}\end{minipage}
				} \\
			\hline
			
			\begin{minipage}[t]{3cm}\textbf{Response and Postconditions}\vspace{0.5em}\end{minipage} &
			\multicolumn{2}{|l|}{
				\begin{minipage}[t]{11.5cm} 
					 \begin{itemize}
        \item L’attacco altera il traffico di rete, causando risposte errate o blocchi che rendono il servizio non affidabile.
        \item L’attaccante modifica o inserisce dati nel traffico, compromettendo l’accuratezza e la coerenza delle informazioni scambiate. 
      \end{itemize}				\vspace{0.5em}\end{minipage}
					} \\
			\hline
			
			\begin{minipage}[t]{3cm}\textbf{Non Functional Requirements}\vspace{0.5em}\end{minipage} &
		\multicolumn{2}{|l|}{
			\begin{minipage}[t]{11.5cm}
				Garantire la protezione dell’integrità del traffico tramite autenticazione delle comunicazioni e il rilevamento di pacchetti anomali o non autorizzati.		
			\vspace{0.5em}\end{minipage}
				} \\
			\hline
			
			\begin{minipage}[t]{3cm}\textbf{Mitigations}\vspace{0.5em}\end{minipage} &
			\multicolumn{2}{|l|}{
				\begin{minipage}[t]{11.5cm}
					\begin{itemize}
            			 \item Usare TLS per tutte le comunicazioni di rete, così da proteggere confidenzialità e integrità, rendendo più difficile l’iniezione di pacchetti malevoli.
        \item Verificare l’identità degli endpoint di rete tramite protocolli sicuri (TLS, IPsec) per assicurarsi che i dati provengano da fonti legittime.
        \item Implementare ingress filtering per bloccare pacchetti con indirizzi IP falsificati, riducendo la possibilità di spoofing e injection.
        \item Deployare sistemi di rilevamento/prevenzione delle intrusioni per analizzare il traffico di rete e bloccare pattern sospetti o iniettati.
        \item Configurare policy di rete che negano per default il traffico non autorizzato, consentendo solo le comunicazioni esplicitamente permesse.
        \item In ambienti aziendali, usare la decrittazione controllata del TLS presso punti di sicurezza per permettere l’analisi del traffico cifrato senza compromettere la sicurezza.
        \item Usare firewall e limiti di traffico per evitare sovraccarichi o pacchetti in eccesso.
        			\end{itemize}				
				\vspace{0.5em}\end{minipage}
			} \\
			\hline
			
			\begin{minipage}[t]{3cm}\textbf{Comments}\vspace{0.5em}\end{minipage} &
			\multicolumn{2}{|l|}{
				\begin{minipage}[t]{12cm}
					/
				\vspace{0.5em}\end{minipage}
			} \\
			\hline
			
		\end{tabular}
		}}
\end{table} \clearpage

%%%%%%%%%%%%%%%%%% Counterfeit GPS Signals (CAPEC 627) %%%%%%%%%%%%%%%%%%%%%%%%%%
\begin{table}[ht!]
	    \resizebox{\textwidth}{!}{%
		\centering
	{\footnotesize
		\begin{tabular}{|l |l| l|}
			\hline
			\begin{minipage}[t]{3cm}\textbf{Case Type}\end{minipage} &
			\begin{minipage}[t]{6cm}\textbf{Abuse Case}\end{minipage} &
			\begin{minipage}[t]{7cm}\textbf{Case ID} AC-13 \end{minipage} \\ \hline
			
			\begin{minipage}[t]{3cm}\textbf{Case Name}\vspace{0.5em}\end{minipage} &
			\multicolumn{2}{|l|}{
				\begin{minipage}[t]{13cm}
					Counterfeit GPS Signals (CAPEC 627)
				\vspace{0.5em}\end{minipage}
			} \\
			\hline
			
			\begin{minipage}[t]{3cm}\textbf{Actors}\vspace{0.5em}\end{minipage} &
			\multicolumn{2}{|l|}{
				\begin{minipage}[t]{13cm}
					 Sistema, Attaccante
				\vspace{0.5em}\end{minipage}
			} \\
			\hline
			\begin{minipage}[t]{3cm}\textbf{Description}\vspace{0.5em}\end{minipage} &
			\multicolumn{2}{|l|}{
				\begin{minipage}[t]{16.5cm}
	Un avversario tenta di ingannare un ricevitore GPS trasmettendo segnali GPS contraffatti, strutturati in modo da assomigliare a un insieme di segnali GPS normali. Questi segnali falsificati possono essere strutturati in modo tale da indurre il ricevitore a stimare la propria posizione in un luogo diverso da quello in cui si trova effettivamente, oppure a considerarla localizzata in un luogo diverso ma in un momento diverso, come determinato dall'avversario.		\vspace{0.5em}\end{minipage}
			} \\
			\hline
			
			\begin{minipage}[t]{3cm}\textbf{Data}\vspace{0.5em}\end{minipage} &
			\multicolumn{2}{|l|}{
				\begin{minipage}[t]{13cm}
					Dati GPS.
				\vspace{0.5em}\end{minipage}
			} \\
			\hline
			
			\begin{minipage}[t]{3cm}\textbf{Stimulus and Preconditions}\vspace{0.5em}\end{minipage} &
			\multicolumn{2}{|l|}{
				\begin{minipage}[t]{16.5cm}
					Per eseguire operazioni critiche, il bersaglio deve poter contare su un segnale GPS valido.
				\vspace{0.5em}\end{minipage}
			} \\
			\hline
			
			\begin{minipage}[t]{3cm}\textbf{Attack Flow 1}\vspace{0.5em}\end{minipage} &
				\begin{minipage}[t]{4.5cm} 
				 Defacement: Internal Defacement	(T1491.001)	
				\vspace{0.5em}\end{minipage} &
				\begin{minipage}[t]{10cm}
				Un avversario può modificare o rovinare i sistemi interni di un’organizzazione per intimidire o confondere gli utenti, mettendo in dubbio l’affidabilità dell’infrastruttura. Questo può includere la modifica di pagine web interne, dei messaggi di accesso ai server o perfino dei computer degli utenti, ad esempio cambiando lo sfondo del desktop. A volte vengono usate immagini offensive o inquietanti per creare disagio o spingere gli utenti a seguire falsi messaggi o istruzioni.	In genere questo tipo di deturpamento avviene dopo che l’attaccante ha già ottenuto accesso ai sistemi e raggiunto altri obiettivi più importanti, perché rende evidente la sua presenza.		\vspace{0.5em}\end{minipage}
			\\
			\hline
			
			\begin{minipage}[t]{3cm}\textbf{Attack Flow 2}\vspace{0.5em}\end{minipage} &
				\begin{minipage}[t]{4cm} 
				Defacement: External Defacement	(T1491.002)	
				\vspace{0.5em}\end{minipage} &
				\begin{minipage}[t]{10cm}
				Un avversario può deturpare sistemi esterni di un’organizzazione per inviare messaggi, intimidire o ingannare l’azienda e i suoi utenti. Questo tipo di attacco può far perdere fiducia nei servizi online e far dubitare dell’integrità del sistema. I siti web pubblici sono i bersagli più comuni: spesso vengono modificati da attaccanti per diffondere propaganda o messaggi politici. L' external defacement esterno può essere usato come reazione a decisioni prese da un’organizzazione o da un governo, oppure per attirare attenzione e generare ulteriori eventi.
				\vspace{0.5em}\end{minipage}
			\\
			\hline
			\begin{minipage}[t]{3cm}\textbf{Attack Flow 3}\vspace{0.5em}\end{minipage} &
			\multicolumn{2}{|l|}{
				\begin{minipage}[t]{13cm} 
						/
					\vspace{0.5em}\end{minipage}
				} \\
			\hline
			
			\begin{minipage}[t]{3cm}\textbf{Response and Postconditions}\vspace{0.5em}\end{minipage} &
			\multicolumn{2}{|l|}{
				\begin{minipage}[t]{13cm} 
					 
        Il sistema riceve segnali GPS falsi, che lo inducono a calcolare una posizione errata o un orario sbagliato, comportando errori di navigazioni e di localizzazione.
      			
	  \vspace{0.5em}\end{minipage}	
	} \\
			\hline
			
			\begin{minipage}[t]{3cm}\textbf{Non Functional Requirements}\vspace{0.5em}\end{minipage} &
		\multicolumn{2}{|l|}{
			\begin{minipage}[t]{16.5cm}
				Il sistema deve essere in grado di mantenere la propria affidabilità anche se il GPS diventa non attendibile, reagendo rapidamente a variazioni sospette della posizione e deve continuare a funzionare grazie a fonti alternative, evitando improvvisi malfunzionamenti.
			\vspace{0.5em}\end{minipage}
				} \\
			\hline
			
			\begin{minipage}[t]{3cm}\textbf{Mitigations}\vspace{0.5em}\end{minipage} &
			\multicolumn{2}{|l|}{
				\begin{minipage}[t]{16.5cm}
					\begin{itemize}
            			 \item Riconoscere quando un segnale GPS non è autentico e nel ridurre la dipendenza dal GPS stesso.
						\item È utile confrontare continuamente la posizione ricevuta con altre fonti, come sensori interni, mappe, altri sistemi satellitari o dati provenienti da reti terrestri.
						\item La coerenza del movimento deve essere monitorata: cambiamenti troppo rapidi o improvvisi sono indizi di spoofing.
						\item A livello hardware si possono usare ricevitori avanzati e antenne progettate per filtrare segnali sospetti.
        			\end{itemize}				
				\vspace{0.5em}\end{minipage}
			} \\
			\hline
			
			\begin{minipage}[t]{3cm}\textbf{Comments}\vspace{0.5em}\end{minipage} &
			\multicolumn{2}{|l|}{
				\begin{minipage}[t]{16.5cm}
					Un attaccante invia segnali GPS falsi per far credere a un sistema di trovarsi in un’altra posizione o in un altro momento. Un aspetto importante è che lo spoofing può restare invisibile se il dispositivo non controlla l’integrità del segnale. Ad esempio, un drone potrebbe essere indotto a deviare rotta verso un’area non sicura.
				\vspace{0.5em}\end{minipage}
			} \\
			\hline
			
		\end{tabular}
		}
		}
\end{table} \clearpage



%%%%%%%%%%%%%%%%%% Transaction or Event Tampering via Application API Manipulation (CAPEC 385) %%%%%%%%%%%%%%%%%%%%%%%%%%
\begin{table}[ht!]
	
	    \resizebox{\textwidth}{!}{%
		\centering
	{\footnotesize
		\begin{tabular}{|l |l| l|}
			\hline
			\begin{minipage}[t]{3cm}\textbf{Case Type}\end{minipage} &
			\begin{minipage}[t]{6cm}\textbf{Abuse Case}\end{minipage} &
			\begin{minipage}[t]{7cm}\textbf{Case ID} AC-14 \end{minipage} \\ \hline
			
			\begin{minipage}[t]{3cm}\textbf{Case Name}\vspace{0.5em}\end{minipage} &
			\multicolumn{2}{|l|}{
				\begin{minipage}[t]{13cm}
					Transaction or Event Tampering via Application API Manipulation (CAPEC 385)
				\vspace{0.5em}\end{minipage}
			} \\
			\hline
			
			\begin{minipage}[t]{3cm}\textbf{Actors}\vspace{0.5em}\end{minipage} &
			\multicolumn{2}{|l|}{
				\begin{minipage}[t]{13cm}
					 Sistema, Attaccante
				\vspace{0.5em}\end{minipage}
			} \\
			\hline
			\begin{minipage}[t]{3cm}\textbf{Description}\vspace{0.5em}\end{minipage} &
			\multicolumn{2}{|l|}{
				\begin{minipage}[t]{16.5cm}
	Un aggressore ospita o si unisce a un evento o a una transazione all'interno di un framework applicativo per modificare il contenuto dei messaggi o degli elementi scambiati. L'esecuzione di questo attacco consente all'aggressore di manipolare il contenuto in modo tale da produrre messaggi o contenuti apparentemente autentici, ma che potrebbero contenere link ingannevoli, sostituire un elemento o un altro, falsificare un elemento esistente ed effettuare uno scambio falso, o altrimenti modificare gli importi o l'identità di ciò che viene scambiato. Le tecniche richiedono l'utilizzo di software specializzati che consentono all'aggressore di stabilire comunicazioni di tipo man-in-the-middle tra il browser web e il sistema remoto al fine di modificare il contenuto di vari elementi dell'applicazione. Spesso, gli elementi scambiati nel gioco possono essere monetizzati tramite vendite di monete, dollari virtuali, ecc. Lo scopo dell'attacco è truffare la vittima intrappolando i pacchetti di dati coinvolti nello scambio e alterando l'integrità del processo di trasferimento.
		\vspace{0.5em}\end{minipage}
			} \\
			\hline
			
			\begin{minipage}[t]{3cm}\textbf{Data}\vspace{0.5em}\end{minipage} &
			\multicolumn{2}{|l|}{
				\begin{minipage}[t]{13cm}
					Dati utente di sistema, Dati cliente, Dati pagamento, Dati corriere.
				\vspace{0.5em}\end{minipage}
			} \\
			\hline
			
			\begin{minipage}[t]{3cm}\textbf{Stimulus and Preconditions}\vspace{0.5em}\end{minipage} &
			\multicolumn{2}{|l|}{
				\begin{minipage}[t]{13cm}
					\begin{itemize}
						\item Il software mirato utilizza le API del framework applicativo;
						\item Un programma software che consente l'uso di comunicazioni Adversary-in-the-Middle ( CAPEC-94 ) tra client e server, come un proxy man-in-the-middle.
					\end{itemize}
				\vspace{0.5em}\end{minipage}
			} \\
			\hline
			
			\begin{minipage}[t]{3cm}\textbf{Attack Flow 1}\vspace{0.5em}\end{minipage} &
				\begin{minipage}[t]{4.5cm} 
				 Process Injection	(T1055.001)	
				\vspace{0.5em}\end{minipage} &
				\begin{minipage}[t]{10cm}
				Gli aggressori possono iniettare librerie a collegamento dinamico (DLL) nei processi per eludere le difese basate sui processi e, eventualmente, elevare i privilegi. L'iniezione di DLL è un metodo per eseguire codice arbitrario nello spazio degli indirizzi di un processo attivo separato.

				L'iniezione di DLL viene comunemente eseguita scrivendo il percorso di una DLL nello spazio di indirizzamento virtuale del processo di destinazione prima di caricare la DLL invocando un nuovo thread. La scrittura può essere eseguita con chiamate API Windows native come VirtualAllocExe WriteProcessMemory, quindi invocata con CreateRemoteThread. \vspace{0.5em}\end{minipage}
			\\
			\hline
			
			\begin{minipage}[t]{3cm}\textbf{Attack Flow 2}\vspace{0.5em}\end{minipage} &
				\begin{minipage}[t]{4cm} 
				Defacement: External Defacement	(T1055.002)	
				\vspace{0.5em}\end{minipage} &
				\begin{minipage}[t]{10cm}
				Gli avversari possono iniettare file eseguibili portatili (PE) nei processi per eludere le difese basate sui processi e, eventualmente, elevare i privilegi. L'iniezione di PE è un metodo per eseguire codice arbitrario nello spazio degli indirizzi di un processo attivo separato.

				L'iniezione PE viene comunemente eseguita copiando il codice  nello spazio di indirizzamento virtuale del processo di destinazione prima di invocarlo tramite un nuovo thread. La scrittura può essere eseguita con chiamate API Windows native come VirtualAllocExe WriteProcessMemory, quindi invocata con CreateRemoteThread codice aggiuntivo. Lo spostamento del codice iniettato introduce il requisito aggiuntivo di funzionalità per rimappare i riferimenti di memoria.
				\vspace{0.5em}\end{minipage}
			\\
			\hline
			\begin{minipage}[t]{3cm}\textbf{Attack Flow 3}\vspace{0.5em}\end{minipage} &
			\multicolumn{2}{|l|}{
				\begin{minipage}[t]{13cm} 
						/
					\vspace{0.5em}\end{minipage}
				} \\
			\hline
			
			\begin{minipage}[t]{3cm}\textbf{Response and Postconditions}\vspace{0.5em}\end{minipage} &
			\multicolumn{2}{|l|}{
				\begin{minipage}[t]{13cm} 
					 Il sistema riceve messaggi falsificati, che potrebbero contenere link ingannevoli o informazioni false, modificando dati nel DataBase e causando, così, disagi al sistema. 		\vspace{0.5em}\end{minipage}	
					} \\
			\hline
			
			\begin{minipage}[t]{3cm}\textbf{Non Functional Requirements}\vspace{0.5em}\end{minipage} &
		\multicolumn{2}{|l|}{
			\begin{minipage}[t]{13cm}
				Il sistema deve validare l'integrità del payload dei messaggi delle API, applicando una validazione rigorosa dei parametri e rifiutando campi non previsti all'interno delle richieste API.
			\vspace{0.5em}\end{minipage}
				} \\
			\hline
			
			\begin{minipage}[t]{3cm}\textbf{Mitigations}\vspace{0.5em}\end{minipage} &
			\multicolumn{2}{|l|}{
				\begin{minipage}[t]{13cm}
					\begin{itemize}
            			 \item Validazione dei parametri delle API;
						\item Utilizzo di un "Schema Validation", rifiutando le richieste con uno schema strutturale differente dallo schema previsto;
						\item Implementazione di meccanismi di autenticazione e autorizzazione robusti per limitare l'accesso alle API solo agli utenti e ai sistemi autorizzati.
        			\end{itemize}				
				\vspace{0.5em}\end{minipage}
			} \\
			\hline
			
			\begin{minipage}[t]{3cm}\textbf{Comments}\vspace{0.5em}\end{minipage} &
			\multicolumn{2}{|l|}{
				\begin{minipage}[t]{13cm}
					/
				\vspace{0.5em}\end{minipage}
			} \\
			\hline
			
		\end{tabular}
		}
		}
\end{table} \clearpage




%%%%%%%%%%%%%%%%%% Server Side Request Forgery (CAPEC 664)  %%%%%%%%%%%%%%%%%%%%%%%%%%
\begin{table}[ht!]
	    \resizebox{\textwidth}{!}{%
		\centering
	{\footnotesize
		\begin{tabular}{|l |l| l|}
			\hline
			\begin{minipage}[t]{4cm}\textbf{Case Type}\end{minipage} &
			\begin{minipage}[t]{7cm}\textbf{Abuse Case}\end{minipage} &
			\begin{minipage}[t]{8cm}\textbf{Case ID} AC-15 \end{minipage} \\ \hline
			
			\begin{minipage}[t]{4cm}\textbf{Case Name}\vspace{0.5em}\end{minipage} &
			\multicolumn{2}{|l|}{
				\begin{minipage}[t]{13cm}
					Server Side Request Forgery (CAPEC 664) 	
									\vspace{0.5em}\end{minipage}
			} \\
			\hline
			
			\begin{minipage}[t]{4cm}\textbf{Actors}\vspace{0.5em}\end{minipage} &
			\multicolumn{2}{|l|}{
				\begin{minipage}[t]{13cm}
					 Sistema, Attaccante
				\vspace{0.5em}\end{minipage}
			} \\
			\hline
			\begin{minipage}[t]{4cm}\textbf{Description}\vspace{0.5em}\end{minipage} &
			\multicolumn{2}{|l|}{
				\begin{minipage}[t]{18cm}
	Un avversario sfrutta una convalida di input non corretta inviando input creati in modo dannoso a un'applicazione target in esecuzione su un server, con l'obiettivo di costringere il server a effettuare una richiesta a se stesso, ai servizi web in esecuzione nella rete interna del server o a terze parti esterne. In caso di successo, la richiesta dell'avversario verrà effettuata con il livello di privilegio del server, aggirandone i controlli di autenticazione. Ciò consente all'avversario di accedere a dati sensibili, eseguire comandi sulla rete del server ed effettuare richieste esterne con l'identità rubata del server. Gli attacchi Server Side Request Forgery differiscono dagli attacchi Cross Site Request Forgery in quanto prendono di mira il server stesso, mentre gli attacchi CSRF sfruttano un meccanismo di autenticazione utente non sicuro per eseguire azioni non autorizzate per conto dell'utente.

		\vspace{0.5em}\end{minipage}
			} \\
			\hline
			
			\begin{minipage}[t]{4cm}\textbf{Data}\vspace{0.5em}\end{minipage} &
			\multicolumn{2}{|l|}{
				\begin{minipage}[t]{13cm}
					Dati utente di sistema, Dati cliente, Dati pagamento, Dati corriere.
				\vspace{0.5em}\end{minipage}
			} \\
			\hline
			
			\begin{minipage}[t]{4cm}\textbf{Stimulus and Preconditions}\vspace{0.5em}\end{minipage} &
			\multicolumn{2}{|l|}{
				\begin{minipage}[t]{18cm}
					\begin{itemize}
						\item Il server deve eseguire un'applicazione web che elabora le richieste HTTP;
						\item L'avversario dovrà rilevare la vulnerabilità tramite un servizio intermediario o specificare URL creati in modo dannoso e analizzare la risposta del server;
						\item L'avversario dovrà accedere alle risorse interne, estrarre informazioni o sfruttare i servizi in esecuzione sul server per eseguire azioni non autorizzate, come attraversare la rete locale o instradare un attacco DDoS TCP riflesso attraverso di essi.
					\end{itemize}
				\vspace{0.5em}\end{minipage}
			} \\
			\hline
			
			\begin{minipage}[t]{4cm}\textbf{Attack Flow 1}\vspace{0.5em}\end{minipage} &
				\begin{minipage}[t]{5.5cm} 
				 Abuse Elevation Control Mechanism: Sudo and Sudo Caching	(T1548.003)	
				\vspace{0.5em}\end{minipage} &
				\begin{minipage}[t]{11cm}
				Gli avversari possono eseguire il caching sudo e/o utilizzare il file sudoers per elevare i privilegi. Gli avversari possono farlo per eseguire comandi al posto di altri utenti o per generare processi con privilegi più elevati.
\vspace{0.5em}\end{minipage}
			\\
			\hline
			
			\begin{minipage}[t]{4cm}\textbf{Attack Flow 2}\vspace{0.5em}\end{minipage} &
				\begin{minipage}[t]{5cm} 
				Abuse Elevation Control Mechanism: Elevated Execution with Prompt	(T1548.004)	
				\vspace{0.5em}\end{minipage} &
				\begin{minipage}[t]{11cm}
				Gli avversari potrebbero sfruttare l'AuthorizationExecuteWithPrivilegesAPI per aumentare i privilegi richiedendo all'utente le credenziali. Lo scopo di questa API è fornire agli sviluppatori di applicazioni un modo semplice per eseguire operazioni con privilegi di root, come l'installazione o l'aggiornamento di applicazioni. Questa API non convalida che il programma che richiede i privilegi di root provenga da una fonte attendibile o sia stato modificato in modo dannoso.
				\vspace{0.5em}\end{minipage}
			\\
			\hline
			\begin{minipage}[t]{4cm}\textbf{Attack Flow 3}\vspace{0.5em}\end{minipage} &
			\begin{minipage}[t]{5cm} 
				Abuse Elevation Control Mechanism: TCC Manipulation	(T1548.006)	
				\vspace{0.5em}\end{minipage} &
				\begin{minipage}[t]{11cm}
				Gli aggressori possono manipolare o abusare del servizio o del database Transparency, Consent \& Control (TCC) per concedere autorizzazioni elevate agli eseguibili dannosi. TCC è un meccanismo di controllo Privacy \& Security di macOS utilizzato per determinare se il processo in esecuzione ha l'autorizzazione ad accedere ai dati o ai servizi protetti da TCC, come la condivisione dello schermo, la fotocamera, il microfono o l'accesso completo al disco (FDA).
				\vspace{0.5em}\end{minipage} \\
			\hline
			
			\begin{minipage}[t]{4cm}\textbf{Response and Postconditions}\vspace{0.5em}\end{minipage} &
			\multicolumn{2}{|l|}{
				\begin{minipage}[t]{18cm}
					 Il sistema effettua una richiesta a se stesso o a servizi esterni, utilizzando il livello di privilegio del server, consentendo all'attaccante di accedere a dati sensibili o eseguire comandi non autorizzati.
					 		\vspace{0.5em}\end{minipage}	
					} \\
			\hline
			
			\begin{minipage}[t]{4cm}
				\textbf{Non Functional Requirements}\vspace{0.5em}\end{minipage} &
		\multicolumn{2}{|l|}{
			\begin{minipage}[t]{18cm}
				Il sistema deve convalidare l'input ricevuto, impedendo di effettuare richieste non autorizzate utilizzando i privilegi del server.
			\vspace{0.5em}\end{minipage}
				} \\
			\hline
			
			\begin{minipage}[t]{4cm}\textbf{Mitigations}\vspace{0.5em}\end{minipage} &
			\multicolumn{2}{|l|}{
				\begin{minipage}[t]{18cm}
					\begin{itemize}
            			 \item La prima linea d'azione per mitigare questa vulnerabilità è la gestione sicura delle richieste in arrivo. Questo può essere fatto tramite la convalida degli URL;
            			 \item Più avanti nel flusso del processo, un altro modo per proteggere il server è esaminare la risposta e verificare che sia come previsto prima dell'invio;
            			 \item Un'altra efficace misura di sicurezza è quella di consentire l'accesso al nome DNS o all'indirizzo IP di ogni servizio che l'applicazione web deve utilizzare. In questo modo, il server non può effettuare richieste esterne a servizi arbitrari;
            			 \item Richiedere l'autenticazione per i servizi locali aggiunge un ulteriore livello di sicurezza tra l'avversario e i servizi interni in esecuzione sul server. Imponendo l'autenticazione locale, un avversario non potrà accedere a tutti i servizi interni solo tramite l'accesso al server.
        			\end{itemize}				
				\vspace{0.5em}\end{minipage}
			} \\
			\hline
			
			\begin{minipage}[t]{4cm}\textbf{Comments}\vspace{0.5em}\end{minipage} &
			\multicolumn{2}{|l|}{
				\begin{minipage}[t]{18cm}
					Un sito di e-commerce consente al cliente di filtrare i risultati per categorie specifiche. Quando il cliente seleziona la categoria desiderata, il negozio online interroga un servizio back-end per recuperare i prodotti richiesti. La richiesta potrebbe essere simile a questa:

					\begin{tcolorbox}[verbatim,
colback=yellow!5!white, colframe=yellow!75!black, title=Richiesta]
\small
POST /prodotto/categoria HTTP/1.0
Content-Type: application/x-www-form-urlencoded
Content-Length: 200

vulnerableService=http://vulnerableshop.net:8080/product \linebreak /category/check?categoryName=someCategory
\end{tcolorbox}



Un utente malintenzionato può modificare l'URL della richiesta in modo che appaia in questo modo:

\begin{tcolorbox}[verbatim,
colback=red!5!white, colframe=red!75!black, title=Richiesta Manipolata]
\small
POST /prodotto/categoria HTTP/1.0
Content-Type: application/x-www-form-urlencoded
Content-Length: 200

vulnerableService=http://localhost/server-status
\end{tcolorbox}

				\vspace{0.5em}\end{minipage}
			} \\
			\hline
			
		\end{tabular}
		}}
\end{table} \clearpage



%%%%%%%%%%%%%%%%%% Input data manipulation (CAPEC 153) %%%%%%%%%%%%%%%%%%%%%%%%%%
\begin{table}[ht!]
	
	    \resizebox{\textwidth}{!}{%
		\centering
	{\footnotesize
		\begin{tabular}{|l |l| l|}
			\hline
			\begin{minipage}[t]{3cm}\textbf{Case Type}\end{minipage} &
			\begin{minipage}[t]{6cm}\textbf{Abuse Case}\end{minipage} &
			\begin{minipage}[t]{7cm}\textbf{Case ID} AC-16 \end{minipage} \\ \hline
			
			\begin{minipage}[t]{3cm}\textbf{Case Name}\vspace{0.5em}\end{minipage} &
			\multicolumn{2}{|l|}{
				\begin{minipage}[t]{13cm}
					 Input data manipulation (CAPEC 153)
				\vspace{0.5em}\end{minipage}
			} \\
			\hline
			
			\begin{minipage}[t]{3cm}\textbf{Actors}\vspace{0.5em}\end{minipage} &
			\multicolumn{2}{|l|}{
				\begin{minipage}[t]{13cm}
					 Sistema, Attaccante
				\vspace{0.5em}\end{minipage}
			} \\
			\hline
			\begin{minipage}[t]{3cm}\textbf{Description}\vspace{0.5em}\end{minipage} &
			\multicolumn{2}{|l|}{
				\begin{minipage}[t]{17.5cm}
	Un aggressore sfrutta una debolezza nella convalida dell'input controllando il formato, la struttura e la composizione dei dati in un'interfaccia di elaborazione dell'input. Fornendo input in un formato non standard o inaspettato, un aggressore può compromettere la sicurezza del bersaglio.
	Ad esempio, l'utilizzo di una codifica dei caratteri diversa potrebbe far sì che un testo pericoloso venga trattato come testo sicuro. In alternativa, l'attaccante può utilizzare determinati flag, come le estensioni dei file, per indurre un'applicazione target a credere che i dati forniti debbano essere gestiti tramite un determinato interprete, quando in realtà i dati non sono del tipo appropriato. Ciò può portare a bypassare i meccanismi di protezione, costringendo il target a utilizzare componenti specifici per l'elaborazione dell'input o, in altro modo, causando una gestione dei dati dell'utente diversa da quella prevista. Questo attacco differisce dalla Manipolazione delle Variabili in quanto la Manipolazione delle Variabili tenta di sovvertire l'elaborazione del target tramite il valore dell'input, mentre la Manipolazione dei Dati di Input cerca di controllare il modo in cui l'input viene elaborato.
		\vspace{0.5em}\end{minipage}
			} \\
			\hline
			
			\begin{minipage}[t]{3cm}\textbf{Data}\vspace{0.5em}\end{minipage} &
			\multicolumn{2}{|l|}{
				\begin{minipage}[t]{13cm}
					Dati utente di sistema, Dati cliente, Dati pagamento, Dati corriere.
				\vspace{0.5em}\end{minipage}
			} \\
			\hline
			
			\begin{minipage}[t]{3cm}\textbf{Stimulus and Preconditions}\vspace{0.5em}\end{minipage} &
			\multicolumn{2}{|l|}{
				\begin{minipage}[t]{13cm}
					Il target deve accettare i dati dell'utente per l'elaborazione e il modo in cui tali dati vengono elaborati deve dipendere da qualche aspetto del formato o dei flag che l'aggressore può controllare.
				\vspace{0.5em}\end{minipage}
			} \\
			\hline
			\begin{minipage}[t]{4cm}\textbf{Attack Flow 1}\vspace{0.5em}\end{minipage} &
			\begin{minipage}[t]{5cm} 
				Exploit Public-Facing Application	(T1190)	
				\vspace{0.5em}\end{minipage} &
				\begin{minipage}[t]{11cm}
				Gli aggressori potrebbero tentare di sfruttare una debolezza in un host o sistema connesso a Internet per accedere inizialmente a una rete. La debolezza del sistema può essere un bug del software, un problema temporaneo o una configurazione errata, mediante l'utilizzo di tecniche come l'iniezione SQL, il cross-site scripting (XSS) o il buffer overflow.
				\vspace{0.5em}\end{minipage} \\
			\hline
			
			
			\begin{minipage}[t]{4cm}\textbf{Attack Flow 2}\vspace{0.5em}\end{minipage} &
			\begin{minipage}[t]{5cm} 
				Server Software Component: Web Shell (T1505.003)	
				\vspace{0.5em}\end{minipage} &
				\begin{minipage}[t]{11cm}
				Gli avversari possono sfruttare le backdoor dei server web con web shell per stabilire un accesso persistente ai sistemi. Una web shell è uno script web che viene installato su un server web accessibile pubblicamente per consentire a un avversario di accedere al server web come gateway in una rete. Una web shell può fornire un set di funzioni da eseguire o un'interfaccia a riga di comando sul sistema che ospita il server web. Le web shell possono essere caricate sul server web tramite vari metodi, tra cui l'upload di file dannosi, lo sfruttamento di vulnerabilità del software del server web o l'abuso di funzionalità legittime del server web.
				\vspace{0.5em}\end{minipage} \\	
			\hline


			\begin{minipage}[t]{3cm}\textbf{Attack Flow 3}\vspace{0.5em}\end{minipage} &
			\multicolumn{2}{|l|}{
				\begin{minipage}[t]{13cm} 
						/
					\vspace{0.5em}\end{minipage}
				} \\
			\hline
			
			\begin{minipage}[t]{3cm}\textbf{Response and Postconditions}\vspace{0.5em}\end{minipage} &
			\multicolumn{2}{|l|}{
				\begin{minipage}[t]{13cm} 
					 Il sistema riceve messaggi ed input inviati dall'aggressore.		\vspace{0.5em}\end{minipage}	
					} \\
			\hline
			
			\begin{minipage}[t]{3cm}\textbf{Non Functional Requirements}\vspace{0.5em}\end{minipage} &
		\multicolumn{2}{|l|}{
			\begin{minipage}[t]{17.5cm}
				Il sistema deve validare il formato, la struttura e la composizione dei dati in un'interfaccia di elaborazione dell'input, rifiutando input in un formato non standard o inaspettato.
			\vspace{0.5em}\end{minipage}
				} \\
			\hline
			
			\begin{minipage}[t]{3cm}\textbf{Mitigations}\vspace{0.5em}\end{minipage} &
			\multicolumn{2}{|l|}{
				\begin{minipage}[t]{13cm}
					\begin{itemize}
            			 \item Convalida rigorosa dell'input;
						\item Utilizzo di librerie sicure per il parsing dei dati;
						\item Limitazione dei valori sensibili modificabili dall’utente.
						\end{itemize}
				\vspace{0.5em}\end{minipage}
			} \\
			\hline
			
			\begin{minipage}[t]{3cm}\textbf{Comments}\vspace{0.5em}\end{minipage} &
			\multicolumn{2}{|l|}{
				\begin{minipage}[t]{13cm}
					/
				\vspace{0.5em}\end{minipage}
			} \\
			\hline
			
		\end{tabular}
		}
		}
\end{table} \clearpage


%%%%%%%%%%%%%%%%%% Fake the Source of Data (CAPEC 194) %%%%%%%%%%%%%%%%%%%%%%%%%%
\begin{table}[ht!]
	
	    \resizebox{\textwidth}{!}{%
		\centering
	{\footnotesize
		\begin{tabular}{|l |l| l|}
			\hline
			\begin{minipage}[t]{3cm}\textbf{Case Type}\end{minipage} &
			\begin{minipage}[t]{6cm}\textbf{Abuse Case}\end{minipage} &
			\begin{minipage}[t]{7cm}\textbf{Case ID} AC-17 \end{minipage} \\ \hline
			
			\begin{minipage}[t]{3cm}\textbf{Case Name}\vspace{0.5em}\end{minipage} &
			\multicolumn{2}{|l|}{
				\begin{minipage}[t]{13cm}
					 Fake the Source of Data (CAPEC 194)
				\vspace{0.5em}\end{minipage}
			} \\
			\hline
			
			\begin{minipage}[t]{3cm}\textbf{Actors}\vspace{0.5em}\end{minipage} &
			\multicolumn{2}{|l|}{
				\begin{minipage}[t]{13cm}
					 Sistema, Attaccante
				\vspace{0.5em}\end{minipage}
			} \\
			\hline
			\begin{minipage}[t]{3cm}\textbf{Description}\vspace{0.5em}\end{minipage} &
			\multicolumn{2}{|l|}{
				\begin{minipage}[t]{17.5cm}
	Un avversario sfrutta un'autenticazione impropria per fornire dati o servizi con un'identità falsificata. Lo scopo dell'utilizzo dell'identità falsificata potrebbe essere quello di impedire la tracciabilità dei dati forniti o di appropriarsi dei diritti concessi a un altro individuo. Una delle forme più semplici di questo attacco sarebbe la creazione di un messaggio di posta elettronica con un campo "Da" modificato per far sembrare che il messaggio sia stato inviato da qualcuno diverso dal mittente effettivo. La radice dell'attacco (in questo caso il sistema di posta elettronica) non riesce ad autenticare correttamente la fonte e questo fa sì che il lettore esegua erroneamente l'azione richiesta. I risultati dell'attacco variano a seconda dei dettagli dell'attacco, ma tra i risultati più comuni figurano l'escalation dei privilegi, l'offuscamento di altri attacchi e la corruzione/manipolazione dei dati.
		\vspace{0.5em}\end{minipage}
			} \\
			\hline
			
			\begin{minipage}[t]{3cm}\textbf{Data}\vspace{0.5em}\end{minipage} &
			\multicolumn{2}{|l|}{
				\begin{minipage}[t]{13cm}
					Dati utente di sistema, Dati cliente, Dati pagamento, Dati corriere, Dati GPS.
				\vspace{0.5em}\end{minipage}
			} \\
			\hline
			
			\begin{minipage}[t]{3cm}\textbf{Stimulus and Preconditions}\vspace{0.5em}\end{minipage} &
			\multicolumn{2}{|l|}{
				\begin{minipage}[t]{17.5cm}
					\begin{itemize}
						\item Questo attacco è applicabile solo quando un'entità vulnerabile associa dati o servizi a un'identità. Senza tale associazione, non ci sarebbe motivo di falsificare la fonte;
						\item Le risorse richieste variano a seconda della natura dell'attacco. Tra gli strumenti di cui un aggressore potrebbe aver bisogno ci sono strumenti per creare pacchetti di rete personalizzati, software client specifici e strumenti per catturare il traffico di rete. Tuttavia, molte varianti di questo attacco non richiedono risorse da parte dell'aggressore.
					\end{itemize}
				\vspace{0.5em}\end{minipage}
			} \\
			\hline
			\begin{minipage}[t]{4cm}\textbf{Attack Flow 1}\vspace{0.5em}\end{minipage} &
			\begin{minipage}[t]{5cm} 
				URL Redirector Abuse	(WASC-38)	
				\vspace{0.5em}\end{minipage} &
				\begin{minipage}[t]{11cm}
				I redirector URL rappresentano una funzionalità comune utilizzata dai siti web per inoltrare una richiesta in arrivo a una risorsa alternativa. Questa operazione può essere eseguita per diversi motivi e spesso viene eseguita per consentire lo spostamento delle risorse all'interno della struttura delle directory ed evitare di interrompere la funzionalità per gli utenti che richiedono la risorsa nella sua posizione precedente. I redirector URL possono anche essere utilizzati per implementare il bilanciamento del carico, sfruttando URL abbreviati o registrando i link in uscita. È quest'ultima implementazione che viene spesso utilizzata negli attacchi di phishing, come descritto nell'esempio seguente. I redirector URL non rappresentano necessariamente una vulnerabilità di sicurezza diretta, ma possono essere sfruttati in modo improprio dagli aggressori che cercano di indurre le vittime a credere di stare navigando verso un sito diverso da quello di destinazione effettivo.
				\vspace{0.5em}\end{minipage} \\
			\hline
			
			
			\begin{minipage}[t]{4cm}\textbf{Attack Flow 2}\vspace{0.5em}\end{minipage} &
			\multicolumn{2}{|l|}{
				\begin{minipage}[t]{13cm} 
						/
					\vspace{0.5em}\end{minipage}
				} \\
			\hline


			\begin{minipage}[t]{3cm}\textbf{Attack Flow 3}\vspace{0.5em}\end{minipage} &
			\multicolumn{2}{|l|}{
				\begin{minipage}[t]{13cm} 
						/
					\vspace{0.5em}\end{minipage}
				} \\
			\hline
			
			\begin{minipage}[t]{3cm}\textbf{Response and Postconditions}\vspace{0.5em}\end{minipage} &
			\multicolumn{2}{|l|}{
				\begin{minipage}[t]{13cm} 
					 Il sistema riceve messaggi ed input inviati dall'aggressore, mediante un fonte falsificata.		\vspace{0.5em}\end{minipage}	
					} \\
			\hline
			
			\begin{minipage}[t]{3cm}\textbf{Non Functional Requirements}\vspace{0.5em}\end{minipage} &
		\multicolumn{2}{|l|}{
			\begin{minipage}[t]{17cm}
				Il sistema deve verificare l’identità della sorgente per ogni dato ricevuto, impedendo di accettare dati provenienti da fonti non autorizzate.
			\vspace{0.5em}\end{minipage}
				} \\
			\hline
			
			\begin{minipage}[t]{3cm}\textbf{Mitigations}\vspace{0.5em}\end{minipage} &
			\multicolumn{2}{|l|}{
				\begin{minipage}[t]{13cm}
					\begin{itemize}
            			 \item Autenticazione forte della sorgente dei dati;
						\item Validazione della provenienza;
						\item Crittografia end-to-end;
						\item Limitazione dei canali di input.
						\end{itemize}
				\vspace{0.5em}\end{minipage}
			} \\
			\hline
			
			\begin{minipage}[t]{3cm}\textbf{Comments}\vspace{0.5em}\end{minipage} &
			\multicolumn{2}{|l|}{
				\begin{minipage}[t]{13cm}
					/
				\vspace{0.5em}\end{minipage}
			} \\
			\hline
			
		\end{tabular}
		}
		}
\end{table} \clearpage


%%%%%%%%%%%%%%%%%% Content Spoofing (CAPEC 148) %%%%%%%%%%%%%%%%%%%%%%%%%%
\begin{table}[ht!]

	    \resizebox{\textwidth}{!}{%
		\centering
		\hspace*{-3cm}
	{\footnotesize
		\begin{tabular}{|l |l| l|}
			\hline
			\begin{minipage}[t]{3.6cm}\textbf{Case Type}\end{minipage} &
			\begin{minipage}[t]{7.2cm}\textbf{Abuse Case}\end{minipage} &
			\begin{minipage}[t]{8.4cm}\textbf{Case ID} AC-18 \end{minipage} \\ \hline
			
			\begin{minipage}[t]{3.6cm}\textbf{Case Name}\vspace{0.5em}\end{minipage} &
			\multicolumn{2}{|l|}{
				\begin{minipage}[t]{15.6cm}
					 Content Spoofing (CAPEC 148)
				\vspace{0.5em}\end{minipage}
			} \\
			\hline
			
			\begin{minipage}[t]{3.6cm}\textbf{Actors}\vspace{0.5em}\end{minipage} &
			\multicolumn{2}{|l|}{
				\begin{minipage}[t]{15.6cm}
					 Sistema, Attaccante
				\vspace{0.5em}\end{minipage}
			} \\
			\hline
			\begin{minipage}[t]{3.6cm}\textbf{Description}\vspace{0.5em}\end{minipage} &
			\multicolumn{2}{|l|}{
				\begin{minipage}[t]{21cm}
	Un aggressore modifica il contenuto per renderlo diverso da ciò che il produttore del contenuto originale intendeva, mantenendo invariata la fonte apparente del contenuto. Il termine "spoofing di contenuto" viene spesso utilizzato per descrivere la modifica di pagine web ospitate da un bersaglio per visualizzare il contenuto dell'aggressore anziché quello del proprietario. Tuttavia, qualsiasi contenuto può essere falsificato, inclusi il contenuto di messaggi di posta elettronica, trasferimenti di file o il contenuto di altri protocolli di comunicazione di rete. Il contenuto può essere modificato all'origine (ad esempio, modificando il file sorgente di una pagina web) o in transito (ad esempio, intercettando e modificando un messaggio tra il mittente e il destinatario). Di solito, l'aggressore tenterà di nascondere il fatto che il contenuto è stato modificato, ma in alcuni casi, come nel caso del defacement di un sito web, questo non è necessario. Il content spoofing può portare all'esposizione a malware, frodi finanziarie (se il contenuto regola transazioni finanziarie), violazioni della privacy e altri effetti indesiderati.
		\vspace{0.5em}\end{minipage}
			} \\
			\hline
			
			\begin{minipage}[t]{3.6cm}\textbf{Data}\vspace{0.5em}\end{minipage} &
			\multicolumn{2}{|l|}{
				\begin{minipage}[t]{15.6cm}
					Dati utente di sistema, Dati cliente, Dati pagamento, Dati corriere, Dati GPS.
				\vspace{0.5em}\end{minipage}
			} \\
			\hline
			
			\begin{minipage}[t]{3.6cm}\textbf{Stimulus and Preconditions}\vspace{0.5em}\end{minipage} &
			\multicolumn{2}{|l|}{
				\begin{minipage}[t]{21cm}
					\begin{itemize}
						\item Il bersaglio deve fornire contenuti, ma non proteggerli adeguatamente dalle modifiche. L'avversario deve avere i mezzi per alterare dati a cui non è autorizzato. Se il contenuto deve essere modificato durante il transito, l'avversario deve essere in grado di intercettare i messaggi presi di mira;
						\item Se il contenuto deve essere modificato durante il transito, l'avversario necessita di uno strumento in grado di intercettare la comunicazione del bersaglio e di generare/creare pacchetti personalizzati per influenzare le comunicazioni;
						\item il contenuto preso di mira viene modificato in modo che tutto o parte di esso venga reindirizzato verso contenuti pubblicati dall'aggressore (ad esempio, immagini e frame nel sito web del bersaglio potrebbero essere modificati per essere caricati da una fonte controllata dall'aggressore). In questi casi, l'aggressore necessita delle risorse necessarie per ospitare il contenuto sostitutivo.
					\end{itemize}
				\vspace{0.5em}\end{minipage}
			} \\
			\hline
			\begin{minipage}[t]{3.6cm}\textbf{Attack Flow 1}\vspace{0.5em}\end{minipage} &
				\begin{minipage}[t]{5cm} 
				 Defacement: Internal Defacement	(T1491.001)	
				\vspace{0.5em}\end{minipage} &
				\begin{minipage}[t]{13.5cm}
				Un avversario può modificare o rovinare i sistemi interni di un’organizzazione per intimidire o confondere gli utenti, mettendo in dubbio l’affidabilità dell’infrastruttura. Questo può includere la modifica di pagine web interne, dei messaggi di accesso ai server o perfino dei computer degli utenti, ad esempio cambiando lo sfondo del desktop. A volte vengono usate immagini offensive o inquietanti per creare disagio o spingere gli utenti a seguire falsi messaggi o istruzioni.	In genere questo tipo di deturpamento avviene dopo che l’attaccante ha già ottenuto accesso ai sistemi e raggiunto altri obiettivi più importanti, perché rende evidente la sua presenza.		\vspace{0.5em}\end{minipage}
			\\
			\hline
			
			\begin{minipage}[t]{3.6cm}\textbf{Attack Flow 2}\vspace{0.5em}\end{minipage} &
				\begin{minipage}[t]{5cm} 
				Defacement: External Defacement	(T1491.002)	
				\vspace{0.5em}\end{minipage} &
				\begin{minipage}[t]{13.5cm}
				Un avversario può deturpare sistemi esterni di un’organizzazione per inviare messaggi, intimidire o ingannare l’azienda e i suoi utenti. Questo tipo di attacco può far perdere fiducia nei servizi online e far dubitare dell’integrità del sistema. I siti web pubblici sono i bersagli più comuni: spesso vengono modificati da attaccanti per diffondere propaganda o messaggi politici. L' external defacement esterno può essere usato come reazione a decisioni prese da un’organizzazione o da un governo, oppure per attirare attenzione e generare ulteriori eventi.
				\vspace{0.5em}\end{minipage}
			\\
			\hline


			\begin{minipage}[t]{3.6cm}\textbf{Attack Flow 3}\vspace{0.5em}\end{minipage} &
				\begin{minipage}[t]{5cm} 
				Content Spoofing (WASC-12)
				\vspace{0.5em}\end{minipage} &
				\begin{minipage}[t]{13.5cm}
				Alcune pagine web vengono servite utilizzando sorgenti di contenuto HTML create dinamicamente. Ad esempio, la posizione sorgente di un frame (\texttt{<frame src="http://foo.example/file.html">}) potrebbe essere specificata tramite un valore di parametro URL (\texttt{http://foo.example/page?frame\_src=http://foo.example \linebreak /file.html}). Un aggressore potrebbe essere in grado di sostituire il valore del parametro \texttt{frame\_src} con \text {frame\_src=http://attacker.example/spoof.html}. A differenza dei redirector, quando la pagina web risultante viene servita, la barra degli indirizzi del browser rimane visibilmente sotto il dominio previsto dall'utente (\texttt{foo.example}), ma i dati esterni (\texttt{attacker.example}) sono nascosti da contenuti legittimi. Link appositamente creati possono essere inviati a un utente tramite e-mail, messaggi istantanei, lasciati su post di bacheche elettroniche o imposti agli utenti tramite un attacco Cross-site Scripting. Se un aggressore induce un utente a visitare una pagina web designata dal suo URL dannoso, l'utente crederà di visualizzare contenuti autentici da una posizione, quando in realtà non è così. Gli utenti si fideranno implicitamente del contenuto contraffatto, poiché la barra degli indirizzi del browser visualizza \texttt{http://foo.example}, quando in realtà il frame HTML sottostante fa riferimento a \texttt{http://attacker.example}.
				\vspace{0.5em}\end{minipage}\\
				\hline

			
			\begin{minipage}[t]{3.6cm}\textbf{Response and Postconditions}\vspace{0.5em}\end{minipage} &
			\multicolumn{2}{|l|}{
				\begin{minipage}[t]{15.6cm} 
					 Il sistema riceve messaggi ed input inviati dall'aggressore, mediante un fonte falsificata.		\vspace{0.5em}\end{minipage}	
					} \\
			\hline
			
			\begin{minipage}[t]{3.6cm}\textbf{Non Functional Requirements}\vspace{0.5em}\end{minipage} &
		\multicolumn{2}{|l|}{
			\begin{minipage}[t]{15.6cm}
				Il sistema deve garantire che i contenuti mostrati all’utente non possano essere alterati da attori non autorizzati.
			\vspace{0.5em}\end{minipage}
				} \\
			\hline
			
			\begin{minipage}[t]{3.6cm}\textbf{Mitigations}\vspace{0.5em}\end{minipage} &
			\multicolumn{2}{|l|}{
				\begin{minipage}[t]{15.6cm}
					\begin{itemize}
            			 \item Validazione e sanificazione dell’input;
						\item Crittografia e integrità dei contenuti;
						\item Prevenzione dello spoofing dell’interfaccia.
						\end{itemize}
				\vspace{0.5em}\end{minipage}
			} \\
			\hline
			
			\begin{minipage}[t]{3.6cm}\textbf{Comments}\vspace{0.5em}\end{minipage} &
			\multicolumn{2}{|l|}{
				\begin{minipage}[t]{15.6cm}
					/
				\vspace{0.5em}\end{minipage}
			} \\
			\hline
			
		\end{tabular}
		}
		}
\end{table} \clearpage

%%%%%%%%%%%%%%%%%% Identity Spoofing                (CAPEC 151) %%%%%%%%%%%%%%%%%%%%%%%%%%
\begin{table}[ht!]
	    \resizebox{\textwidth}{!}{%
		\centering
	{\footnotesize
		\begin{tabular}{|l |l| l|}
			\hline
			\begin{minipage}[t]{3.6cm}\textbf{Case Type}\end{minipage} &
			\begin{minipage}[t]{7.2cm}\textbf{Abuse Case}\end{minipage} &
			\begin{minipage}[t]{8.4cm}\textbf{Case ID} AC-19 \end{minipage} \\ \hline
			
			\begin{minipage}[t]{3.6cm}\textbf{Case Name}\vspace{0.5em}\end{minipage} &
			\multicolumn{2}{|l|}{
				\begin{minipage}[t]{15.6cm}
					Identity Spoofing (CAPEC 151)  
				\vspace{0.5em}\end{minipage}
			} \\
			\hline
			
			\begin{minipage}[t]{3.6cm}\textbf{Actors}\vspace{0.5em}\end{minipage} &
			\multicolumn{2}{|l|}{
				\begin{minipage}[t]{15.6cm}
					 Sistema, Attaccante
				\vspace{0.5em}\end{minipage}
			} \\
			\hline
			\begin{minipage}[t]{3.6cm}\textbf{Description}\vspace{0.5em}\end{minipage} &
			\multicolumn{2}{|l|}{
				\begin{minipage}[t]{21cm}
						L'Identity Spoofing si riferisce all'azione di assumere (ossia, assumere) l'identità di un'altra entità (umana o non umana) e quindi utilizzare tale identità per raggiungere un obiettivo. Un avversario può creare messaggi che sembrano provenire da un principio diverso o utilizzare credenziali di autenticazione rubate/contraffatte.
		\vspace{0.5em}\end{minipage}
			} \\
			\hline
			
			\begin{minipage}[t]{3.6cm}\textbf{Data}\vspace{0.5em}\end{minipage} &
			\multicolumn{2}{|l|}{
				\begin{minipage}[t]{15.6cm}
					Dati utente di sistema.
				\vspace{0.5em}\end{minipage}
			} \\
			\hline
			
			\begin{minipage}[t]{3.6cm}\textbf{Stimulus and Preconditions}\vspace{0.5em}\end{minipage} &
			\multicolumn{2}{|l|}{
				\begin{minipage}[t]{21cm}
					L'identità associata al messaggio o alla risorsa deve essere rimovibile o modificabile in modo non rilevabile.
				\vspace{0.5em}\end{minipage}
			} \\
			\hline
			\begin{minipage}[t]{3.6cm}\textbf{Attack Flow 1}\vspace{0.5em}\end{minipage} & 
			\multicolumn{2}{|l|}{
				\begin{minipage}[t]{21cm} 
						L’avversario ottiene o costruisce credenziali che imitano un’altra identità e le utilizza per inviare richieste o comunicazioni che appaiono legittime. Una volta accettato come entità autorizzata, può accedere a servizi, dati o funzionalità altrimenti non disponibili, manipolando processi o ingannando sistemi che si basano sull’autenticità dell’identità presentata. L’attacco prosegue fino a quando l'attaccante riesce a mantenere viva la falsa identità o fino a quando non ha raggiunto il suo scopo, come l’accesso non autorizzato o l'esfiltrazione di informazioni.
				\vspace{0.5em}\end{minipage}
			}\\
			\hline
			
			\begin{minipage}[t]{3.6cm}\textbf{Attack Flow 2}\vspace{0.5em}\end{minipage} &
			\multicolumn{2}{|l|}{
				\begin{minipage}[t]{17.5cm}
					/
				\vspace{0.5em}\end{minipage}
			}\\
			\hline


			\begin{minipage}[t]{3.6cm}\textbf{Attack Flow 3}\vspace{0.5em}\end{minipage} &
			\multicolumn{2}{|l|}{
				\begin{minipage}[t]{17.5cm}
					/
				\vspace{0.5em}\end{minipage}
			}\\
				\hline

			
			\begin{minipage}[t]{3.6cm}\textbf{Response and Postconditions}\vspace{0.5em}\end{minipage} &
			\multicolumn{2}{|l|}{
				\begin{minipage}[t]{18.6cm} 
					 \begin{itemize}
						\item Il sistema invalida immediatamente la sessione o il token associato all’identità falsificata.
						\item Le credenziali sospette vengono bloccate o revocate e viene richiesta una ri-verifica dell’identità agli utenti coinvolti.
						\item Vengono registrati log dettagliati dell’evento e inviati alert ai responsabili della sicurezza.
						\item Le attività dell’attaccante vengono isolate e analizzate, assicurando che non restino accessi persistenti o credenziali compromesse.
						\item Il sistema ristabilisce la piena integrità delle operazioni, garantendo che la falsa identità non possa essere riutilizzata.
					 \end{itemize}
				\vspace{0.5em}\end{minipage}	
					} \\
			\hline
			
			\begin{minipage}[t]{3.6cm}\textbf{Non Functional Requirements}\vspace{0.5em}\end{minipage} &
		\multicolumn{2}{|l|}{
			\begin{minipage}[t]{19.5cm}
				\begin{itemize}
					\item L’autenticazione deve essere robusta e resistente alla falsificazione, garantendo sicurezza senza compromettere la fruibilità.
					\item I log di monitoraggio devono essere completi, immutabili e sempre disponibili per audit o indagini.
					\item Il sistema deve rilevare attività anomale in tempo reale e rispondere rapidamente a tentativi di impersonificazione.
					\item L’infrastruttura deve mantenere continuità operativa anche durante incidenti di sicurezza o processi di verifica.
				\end{itemize}
			\vspace{0.5em}\end{minipage}
				} \\
			\hline
			
			\begin{minipage}[t]{3.6cm}\textbf{Mitigations}\vspace{0.5em}\end{minipage} &
			\multicolumn{2}{|l|}{
				\begin{minipage}[t]{15.6cm}
					Utilizzare processi di autenticazione robusti (ad esempio, autenticazione a più fattori).
				\vspace{0.5em}\end{minipage}
			} \\
			\hline
			
			\begin{minipage}[t]{3.6cm}\textbf{Comments}\vspace{0.5em}\end{minipage} &
			\multicolumn{2}{|l|}{
				\begin{minipage}[t]{15.6cm}
					/
				\vspace{0.5em}\end{minipage}
			} \\
			\hline
			
		\end{tabular}
		}
		}
\end{table} \clearpage

%%%%%%%%%%%%%%%%%% Log Injection-Tampering-Forging (CAPEC 93) %%%%%%%%%%%%%%%%%%%%%%%%%%
\begin{table}[ht!]
	    \resizebox{\textwidth}{!}{%
		\centering
	{\footnotesize
		\begin{tabular}{|l |l| l|}
			\hline
			\begin{minipage}[t]{3cm}\textbf{Case Type}\end{minipage} &
			\begin{minipage}[t]{6cm}\textbf{Abuse Case}\end{minipage} &
			\begin{minipage}[t]{7cm}\textbf{Case ID} AC-20 \end{minipage} \\ \hline
			
			\begin{minipage}[t]{3cm}\textbf{Case Name}\vspace{0.5em}\end{minipage} &
			\multicolumn{2}{|l|}{
				\begin{minipage}[t]{13cm}
					 Log Injection-Tampering-Forging (CAPEC 93)
				\vspace{0.5em}\end{minipage}
			} \\
			\hline
			
			\begin{minipage}[t]{3cm}\textbf{Actors}\vspace{0.5em}\end{minipage} &
			\multicolumn{2}{|l|}{
				\begin{minipage}[t]{13cm}
					 Sistema, Attaccante
				\vspace{0.5em}\end{minipage}
			} \\
			\hline
			\begin{minipage}[t]{3cm}\textbf{Description}\vspace{0.5em}\end{minipage} &
			\multicolumn{2}{|l|}{
				\begin{minipage}[t]{17.3cm}
	Questo attacco prende di mira i file di registro dell'host di destinazione. L'attaccante inietta, manipola o finge voci di registro dannose nel file di registro, consentendogli di indurre in errore un audit di registro, coprire le tracce di attacco o eseguire altre azioni dannose. L'host di destinazione non controlla correttamente l'accesso al registro. Di conseguenza, i dati contaminati si traducono nei file di registro che portano a un fallimento nella capacità di responsabilità, non ripudio e capacità forense degli incidenti.
		\vspace{0.5em}\end{minipage}
			} \\
			\hline
			
			\begin{minipage}[t]{3cm}\textbf{Data}\vspace{0.5em}\end{minipage} &
			\multicolumn{2}{|l|}{
				\begin{minipage}[t]{13cm}
					Dati utente di sistema, Dati cliente, Dati pagamento, Dati corriere, Dati GPS.
				\vspace{0.5em}\end{minipage}
			} \\
			\hline
			
			\begin{minipage}[t]{3cm}\textbf{Stimulus and Preconditions}\vspace{0.5em}\end{minipage} &
			\multicolumn{2}{|l|}{
				\begin{minipage}[t]{17.3cm}
				L'host di destinazione protegge in modo insufficiente l'accesso ai registri o ai meccanismi di registrazione.				\vspace{0.5em}\end{minipage}
			} \\
			\hline
			\begin{minipage}[t]{4cm}\textbf{Attack Flow 1}\vspace{0.5em}\end{minipage} &
			\begin{minipage}[t]{5cm} \raggedright
				Indicator Removal on Host	(T1070)	
				\vspace{0.5em}\end{minipage} &
				\begin{minipage}[t]{11cm}
					Gli avversari possono eliminare o modificare gli artefatti generati all'interno dei sistemi per rimuovere le prove della loro presenza o ostacolare le difese. Vari artefatti possono essere creati da un avversario o qualcosa che può essere attribuito alle azioni di un avversario. In genere questi artefatti vengono utilizzati come indicatori difensivi relativi agli eventi monitorati, come stringhe da file scaricati, registri generati dalle azioni dell'utente e altri dati analizzati dai difensori. La posizione, il formato e il tipo di artefatto (come il comando o la cronologia di accesso) sono spesso specifici per ogni piattaforma.
					\vspace{0.5em}\end{minipage} \\
			\hline
			
			
			\begin{minipage}[t]{4cm}\textbf{Attack Flow 2}\vspace{0.5em}\end{minipage} &
			\begin{minipage}[t]{5cm} \raggedright
				Impair Defenses: Disable Windows Event Logging (T1562.002)	
				\vspace{0.5em}\end{minipage} &
				\begin{minipage}[t]{11cm}
				Gli avversari disabilitano la registrazione degli eventi Windows per eliminare le tracce delle loro attività, ostacolando i rilevamenti e gli audit di sicurezza. Il servizio EventLog registra attività critiche come accessi e creazioni di processi, essenziali per le indagini forensi.				\vspace{0.5em}\end{minipage} \\	
			\hline


			\begin{minipage}[t]{3cm}\textbf{Attack Flow 3}\vspace{0.5em}\end{minipage} &
			\multicolumn{2}{|l|}{
				\begin{minipage}[t]{13cm} 
						/
					\vspace{0.5em}\end{minipage}
				} \\
			\hline
			
			\begin{minipage}[t]{3cm}\textbf{Response and Postconditions}\vspace{0.5em}\end{minipage} &
			\multicolumn{2}{|l|}{
				\begin{minipage}[t]{17.3cm} 
					 L'attaccante riesce ad eludere i meccanismi di registrazione e monitoraggio, inserendo valori di input falsificati.		\vspace{0.5em}\end{minipage}	
					} \\
			\hline
			
			\begin{minipage}[t]{3cm}\textbf{Non Functional Requirements}\vspace{0.5em}\end{minipage} &
		\multicolumn{2}{|l|}{
			\begin{minipage}[t]{17.3cm}
				Il sistema deve validare e normalizzare il formato dell'input prima della registrazione, l'uso di logging sicuro per prevenire l'iniezione di log e l'implementazione di meccanismi di monitoraggio per rilevare attività sospette nei file di registro.
			\vspace{0.5em}\end{minipage}
				} \\
			\hline
			
			\begin{minipage}[t]{3cm}\textbf{Mitigations}\vspace{0.5em}\end{minipage} &
			\multicolumn{2}{|l|}{
				\begin{minipage}[t]{17.3cm}
					\begin{itemize}
            			 \item Controlla attentamente l'accesso ai file di registro fisici;

						\item Non consentire che i dati contaminati siano scritti nel file di registro senza previa convalida dell'input. Un elenco di permessi può essere utilizzato per convalidare correttamente i dati;

						\item Usa la sincronizzazione per controllare il flusso di esecuzione;

						\item Usa gli strumenti di analisi statica per identificare le vulnerabilità di log forging;

						\item Evita di visualizzare i registri con strumenti che possono interpretare i caratteri di controllo nel file, come le shell della riga di comando.
						\end{itemize}
				\vspace{0.5em}\end{minipage}
			}\\
			\hline
			
			\begin{minipage}[t]{3cm}\textbf{Comments}\vspace{0.5em}\end{minipage} &
			\multicolumn{2}{|l|}{
				\begin{minipage}[t]{13cm}
					/
				\vspace{0.5em}\end{minipage}
			} \\
			\hline
			
		\end{tabular}
		}
		}
\end{table} \clearpage



%%%%%%%%%%%%%%%%%% Clickjacking (CAPEC 103) %%%%%%%%%%%%%%%%%%%%%%%%%%
\begin{table}[ht!]
	\centering
	\resizebox{\textwidth}{!}{%
	{\footnotesize
		\begin{tabular}{|l |l| l|}
			\hline
			\begin{minipage}[t]{3cm}\textbf{Case Type}\end{minipage} &
			\begin{minipage}[t]{10cm}\textbf{Abuse Case}\end{minipage} &
			\begin{minipage}[t]{3.5cm}\textbf{Case ID} AC-21 \end{minipage} \\ \hline
			
			\begin{minipage}[t]{3cm}\textbf{Case Name}\vspace{0.5em}\end{minipage} &
			\multicolumn{2}{|l|}{
				\begin{minipage}[t]{14cm}
					 Clickjacking (CAPEC 103)
				\vspace{0.5em}\end{minipage}
			} \\
			\hline
			
			\begin{minipage}[t]{3cm}\textbf{Actors}\vspace{0.5em}\end{minipage} &
			\multicolumn{2}{|l|}{
				\begin{minipage}[t]{14cm}
					 Sistema, Attaccante, Utente
				\vspace{0.5em}\end{minipage}
			} \\
			\hline
			\begin{minipage}[t]{3cm}\textbf{Description}\vspace{0.5em}\end{minipage} &
			\multicolumn{2}{|l|}{
				\begin{minipage}[t]{14cm}
			Mentre è connessa a qualche sistema di destinazione, la vittima visita il sito dannoso dell'avversario che mostra un'interfaccia utente con cui la vittima desidera interagire. In realtà, la pagina clickjacked ha un livello trasparente sopra l'interfaccia utente visibile con controlli di azione che l'avversario desidera che la vittima esegua. La vittima fa clic su pulsanti o altri elementi dell'interfaccia utente che vede sulla pagina che in realtà attivano i controlli di azione nel livello di sovrapposizione trasparente. A seconda di quale sia quel controllo dell'azione, l'avversario potrebbe aver appena ingannato la vittima nell'esecuzione di alcune funzionalità potenzialmente privilegiate (e certamente indesiderate) nel sistema di destinazione a cui la vittima è autenticata. Il problema di base qui è che c'è una dicotomia tra ciò su cui la vittima pensa di fare clic rispetto a ciò su cui sta effettivamente facendo clic.		\vspace{0.5em}\end{minipage}
			} \\
			\hline
			
			\begin{minipage}[t]{3cm}\textbf{Data}\vspace{0.5em}\end{minipage} &
			\multicolumn{2}{|l|}{
				\begin{minipage}[t]{14cm}
					Dati utente di sistema.
				\vspace{0.5em}\end{minipage}
			} \\
			\hline
			
			\begin{minipage}[t]{3cm}\textbf{Stimulus and Preconditions}\vspace{0.5em}\end{minipage} &
            \multicolumn{2}{|l|}{
                \begin{minipage}[t]{14cm}
                    L'attacco è realizzabile quando una vittima, autenticata con una sessione attiva su un'applicazione web, utilizza un browser moderno che supporta tecnologie come iFrames o JavaScript e mantiene aperta la finestra del sistema di destinazione, esponendosi così all'esecuzione involontaria di azioni sensibili.
                \vspace{0.5em}\end{minipage}
            } \\
            \hline
            
            \begin{minipage}[t]{3cm}\textbf{Attack Flow 1}\vspace{0.5em}\end{minipage} &
			\multicolumn{2}{|l|}{
            \begin{minipage}[t]{14cm}
                Tecnica in cui un attaccante utilizza livelli (layer) trasparenti o opachi per ingannare l'utente, inducendolo a cliccare su un elemento diverso da quello visibile. In pratica, l'attaccante 'dirotta' i clic destinati alla pagina visibile verso un'altra pagina sottostante, spesso appartenente a un'applicazione o dominio sensibile.
            \vspace{0.5em}\end{minipage}
			}\\
            \hline
            
            \begin{minipage}[t]{3cm}\textbf{Response and Postconditions}\vspace{0.5em}\end{minipage} &
            \multicolumn{2}{|l|}{
                \begin{minipage}[t]{14cm}
                    Il clickjacking, noto anche come "attacco di riparazione dell'interfaccia utente", è quando un utente malintenzionato utilizza più livelli trasparenti o opachi per indurre un utente a fare clic su un pulsante o un link su un'altra pagina quando intendeva fare clic sulla pagina di primo livello. Pertanto, l'attaccante sta "dirottando" i clic destinati alla loro pagina e li indirizza a un'altra pagina, molto probabilmente di proprietà di un'altra applicazione, dominio o entrambi. Utilizzando una tecnica simile, le sequenze di tasti possono anche essere dirottate. Con una combinazione accuratamente realizzata di fogli di stile, iframe e caselle di testo, un utente può essere portato a credere che stia digitando la password della propria email o conto bancario, ma sta invece digitando in una cornice invisibile controllata dall'attaccante.
                \vspace{0.5em}\end{minipage}
            } \\
            \hline
            
            \begin{minipage}[t]{3cm}\textbf{Non Functional Requirements}\vspace{0.5em}\end{minipage} &
            \multicolumn{2}{|l|}{
                \begin{minipage}[t]{14cm}
                    In primo luogo, il sistema deve implementare restrizioni rigorose sul framing. Ogni risposta HTTP generata dall'applicazione che contiene codice HTML deve includere l'header di sicurezza Content-Security-Policy (CSP) configurato con la direttiva frame-ancestors. Questa direttiva deve esplicitamente elencare le origini autorizzate a incorporare l'applicazione (tipicamente impostata su 'self' per consentire il framing solo dallo stesso dominio o 'none' per bloccarlo del tutto). Per garantire la compatibilità con browser meno recenti che non supportano pienamente le CSP, il sistema deve includere ridondantemente anche l'header X-Frame-Options impostato su SAMEORIGIN o DENY.

                    In secondo luogo, deve essere applicata una politica di isolamento della sessione. Tutti i cookie utilizzati per la gestione della sessione e l'autenticazione devono essere contrassegnati con l'attributo SameSite, impostato preferibilmente su Strict (o quantomeno su Lax). Questo requisito assicura che il browser non invii i cookie di sessione se l'applicazione viene caricata all'interno di un contesto di terze parti (come un iframe nascosto), rendendo inefficace l'attacco anche qualora il framing non venisse bloccato.

                    Infine, per le funzionalità critiche (quali transazioni finanziarie, modifiche delle credenziali o eliminazione di dati), il sistema deve adottare un approccio di difesa in profondità nell'interfaccia utente. Tali operazioni non devono essere eseguibili tramite un singolo clic o interazione immediata; il sistema deve richiedere una conferma esplicita (ad esempio tramite una finestra modale, l'inserimento della password o un token OTP) che interrompa il flusso automatico e richieda la consapevolezza attiva dell'utente.
                \vspace{0.5em}\end{minipage}
            } \\
            \hline
            
            \begin{minipage}[t]{3cm}\textbf{Mitigations}\vspace{0.5em}\end{minipage} &
            \multicolumn{2}{|l|}{
                \begin{minipage}[t]{14cm}
                    \begin{itemize}
                        \item Se usi il browser Firefox, usa il plug-in NoScript che aiuterà a vietare iFrames;
                        \item Disattiva JavaScript, Flash e disabilita CSS;
                        \item Quando si mantiene una sessione autenticata con un sistema di destinazione privilegiato, non utilizzare lo stesso browser per navigare su siti sconosciuti per eseguire altre attività. Termina di lavorare con il sistema di destinazione e disconnetti prima di procedere ad altre attività.
                    \end{itemize}
                \vspace{0.5em}\end{minipage}
            } \\
            \hline
            
            \begin{minipage}[t]{3cm}\textbf{Comments}\vspace{0.5em}\end{minipage} &
            \multicolumn{2}{|l|}{
                \begin{minipage}[t]{14cm}
                    /
                \vspace{0.5em}\end{minipage}
            } \\
            \hline
            
        \end{tabular}
        }}
\end{table} \clearpage

%%%%%%%%%%%%%%%%%%%%%%%%%%%%%%%%%%%%%%%%%%%%%%%%%%%%%%%%%%%%%%%%%%%%%%%%%%%%%%%%%%%%%%%%%%%%
\section{Misuse Case}

%%%%% Invio dati su canali non autorizzati %%%%%%%%

In questa sezione, vengono riportati gli schemi di Jacobson relativi ai misuse case.


\begin{table}[ht!]
	    \resizebox{\textwidth}{!}{%
		\centering
	{\footnotesize
		\begin{tabular}{|l |l| l|}
			\hline
			\begin{minipage}[t]{3.6cm}\textbf{Case Type}\end{minipage} &
			\begin{minipage}[t]{7.2cm}\textbf{Misuse Case}\end{minipage} &
			\begin{minipage}[t]{8.4cm}\textbf{Case ID} MC-01 \end{minipage} \\ \hline
			
			\begin{minipage}[t]{3.6cm}\textbf{Case Name}\vspace{0.5em}\end{minipage} &
			\multicolumn{2}{|l|}{
				\begin{minipage}[t]{15.6cm}
					Invio dati su canali non autorizzati
				\vspace{0.5em}\end{minipage}
			} \\
			\hline
			
			\begin{minipage}[t]{3.6cm}\textbf{Actors}\vspace{0.5em}\end{minipage} &
			\multicolumn{2}{|l|}{
				\begin{minipage}[t]{15.6cm}
					 Sistema, Utente maldestro
				\vspace{0.5em}\end{minipage}
			} \\
			\hline
			\begin{minipage}[t]{3.6cm}\textbf{Description}\vspace{0.5em}\end{minipage} &
			\multicolumn{2}{|l|}{
				\begin{minipage}[t]{21cm}
					Un utente malintenzionato o un attaccante interno invia dati sensibili o riservati attraverso canali di comunicazione non autorizzati o insicuri, bypassando le politiche di sicurezza dell'organizzazione. Questo può includere l'invio di informazioni tramite email personali e dispositivi di archiviazione esterni.
		\vspace{0.5em}\end{minipage}
			} \\
			\hline
			
			\begin{minipage}[t]{3.6cm}\textbf{Data}\vspace{0.5em}\end{minipage} &
			\multicolumn{2}{|l|}{
				\begin{minipage}[t]{15.6cm}
					Dati utente di sistema, Dati cliente, Dati pagamento, Dati corriere.
				\vspace{0.5em}\end{minipage}
			} \\
			\hline
			
			\begin{minipage}[t]{3.6cm}\textbf{Stimulus and Preconditions}\vspace{0.5em}\end{minipage} &
			\multicolumn{2}{|l|}{
				\begin{minipage}[t]{21cm}
					L'utente maldestro ha accesso ai dati sensibili.
				\vspace{0.5em}\end{minipage}
			} \\
			\hline
			\begin{minipage}[t]{3.6cm}\textbf{Attack Flow 1}\vspace{0.5em}\end{minipage} & 
			\multicolumn{2}{|l|}{
				\begin{minipage}[t]{21cm} 
						L'utente maldestro invia dati sensibili attraverso  email personali.
				\vspace{0.5em}\end{minipage}
			}\\
			\hline
			
			\begin{minipage}[t]{3.6cm}\textbf{Attack Flow 2}\vspace{0.5em}\end{minipage} &
			\multicolumn{2}{|l|}{
				\begin{minipage}[t]{17.5cm}
					L'utente maldestro utilizza dispositivi di archiviazione esterni per conservare i dati in locale.
				\vspace{0.5em}\end{minipage}
			}\\
			\hline


			\begin{minipage}[t]{3.6cm}\textbf{Attack Flow 3}\vspace{0.5em}\end{minipage} &
			\multicolumn{2}{|l|}{
				\begin{minipage}[t]{17.5cm}
					/
				\vspace{0.5em}\end{minipage}
			}\\
				\hline

			
			\begin{minipage}[t]{3.6cm}\textbf{Response and Postconditions}\vspace{0.5em}\end{minipage} &
			\multicolumn{2}{|l|}{
				\begin{minipage}[t]{18.6cm} 
					 L'utente maldestro riesce a inviare dati sensibili attraverso canali non autorizzati, compromettendo la riservatezza delle informazioni.
				\vspace{0.5em}\end{minipage}	
					} \\
			\hline
			
			\begin{minipage}[t]{3.6cm}\textbf{Non Functional Requirements}\vspace{0.5em}\end{minipage} &
		\multicolumn{2}{|l|}{
			\begin{minipage}[t]{20.5cm}
				Il sistema deve implementare controlli di accesso rigorosi, monitorare le attività degli utenti e utilizzare la crittografia per proteggere i dati sensibili durante la trasmissione.
			\vspace{0.5em}\end{minipage}
				} \\
			\hline
			
			\begin{minipage}[t]{3.6cm}\textbf{Mitigations}\vspace{0.5em}\end{minipage} &
			\multicolumn{2}{|l|}{
				\begin{minipage}[t]{15.6cm}
					\begin{itemize}
						\item Implementare politiche di sicurezza rigorose per l'uso dei canali di comunicazione;
						\item Monitorare e registrare le attività degli utenti per rilevare comportamenti sospetti;
						\item Utilizzare la crittografia per proteggere i dati sensibili durante la trasmissione.
					\end{itemize}
				\vspace{0.1em}\end{minipage}
			} \\
			\hline
			
			\begin{minipage}[t]{3.6cm}\textbf{Comments}\vspace{0.5em}\end{minipage} &
			\multicolumn{2}{|l|}{
				\begin{minipage}[t]{15.6cm}
					/
				\vspace{0.5em}\end{minipage}
			} \\
			\hline
			
		\end{tabular}
		}
		}
\end{table} \clearpage



%%%%%%%%%%%%%%%%%%%%%%%%   Attività CRUD accidentali     %%%%%%%%%%%%%%%%%%%%%%%%%%%%%%%%%%%%%%%%%%%%%%%%%%%%%%%%%%%%%%%


\begin{table}[ht!]
	    \resizebox{\textwidth}{!}{%
		\centering
	{\footnotesize
		\begin{tabular}{|l |l| l|}
			\hline
			\begin{minipage}[t]{3.6cm}\textbf{Case Type}\end{minipage} &
			\begin{minipage}[t]{7.2cm}\textbf{Misuse Case}\end{minipage} &
			\begin{minipage}[t]{8.4cm}\textbf{Case ID} MC-02 \end{minipage} \\ \hline
			
			\begin{minipage}[t]{3.6cm}\textbf{Case Name}\vspace{0.5em}\end{minipage} &
			\multicolumn{2}{|l|}{
				\begin{minipage}[t]{15.6cm}
					Attività CRUD accidentali  
				\vspace{0.5em}\end{minipage}
			} \\
			\hline
			
			\begin{minipage}[t]{3.6cm}\textbf{Actors}\vspace{0.5em}\end{minipage} &
			\multicolumn{2}{|l|}{
				\begin{minipage}[t]{15.6cm}
					 Sistema, Utente maldestro
				\vspace{0.5em}\end{minipage}
			} \\
			\hline
			\begin{minipage}[t]{3.6cm}\textbf{Description}\vspace{0.5em}\end{minipage} &
			\multicolumn{2}{|l|}{
				\begin{minipage}[t]{21cm}
					Un utente maldestro effettua operazioni di CRUD (Create, Read, Update, Delete) in modo accidentale o non intenzionale, causando modifiche indesiderate ai dati o alle risorse del sistema. Queste azioni possono portare a perdita di dati, corruzione delle informazioni o interruzione del servizio.
		\vspace{0.5em}\end{minipage}
			} \\
			\hline
			
			\begin{minipage}[t]{3.6cm}\textbf{Data}\vspace{0.5em}\end{minipage} &
			\multicolumn{2}{|l|}{
				\begin{minipage}[t]{15.6cm}
					Dati utente di sistema, Dati cliente, Dati pagamento, Dati corriere.
				\vspace{0.5em}\end{minipage}
			} \\
			\hline
			
			\begin{minipage}[t]{3.6cm}\textbf{Stimulus and Preconditions}\vspace{0.5em}\end{minipage} &
			\multicolumn{2}{|l|}{
				\begin{minipage}[t]{21cm}
					L'utente maldestro ha accesso ai dati sensibili.
				\vspace{0.5em}\end{minipage}
			} \\
			\hline
			\begin{minipage}[t]{3.6cm}\textbf{Attack Flow 1}\vspace{0.5em}\end{minipage} & 
			\multicolumn{2}{|l|}{
				\begin{minipage}[t]{21cm} 
						L'utente maldestro effettua operazioni di Create, Read, Update o Delete in modo accidentale, violando l'integrità dei dati.
				\vspace{0.5em}\end{minipage}
			}\\
			\hline
			
			\begin{minipage}[t]{3.6cm}\textbf{Attack Flow 2}\vspace{0.5em}\end{minipage} &
			\multicolumn{2}{|l|}{
				\begin{minipage}[t]{17.5cm}
					L'utente maldestro effettua operazioni di Create, Read, Update o Delete in modo accidentale, violando la disponibilità dei dati.
				\vspace{0.5em}\end{minipage}
			}\\
			\hline


			\begin{minipage}[t]{3.6cm}\textbf{Attack Flow 3}\vspace{0.5em}\end{minipage} &
			\multicolumn{2}{|l|}{
				\begin{minipage}[t]{17.5cm}
					/
				\vspace{0.5em}\end{minipage}
			}\\
				\hline

			
			\begin{minipage}[t]{3.6cm}\textbf{Response and Postconditions}\vspace{0.5em}\end{minipage} &
			\multicolumn{2}{|l|}{
				\begin{minipage}[t]{18.6cm} 
					 L'utente maldestro riesce a effettuare operazioni di CRUD in modo accidentale, compromettendo l'integrità e la disponibilità dei dati.
				\vspace{0.5em}\end{minipage}	
					} \\
			\hline
			
			\begin{minipage}[t]{3.6cm}\textbf{Non Functional Requirements}\vspace{0.5em}\end{minipage} &
		\multicolumn{2}{|l|}{
			\begin{minipage}[t]{20.5cm}
				Il sistema deve implementare sistemi di conferma per le operazioni critiche, fornire messaggi di avviso chiari e implementare meccanismi di rollback per mitigare gli effetti delle operazioni accidentali. Inoltre, il sistema deve effettuare continui backup per garantire il recupero dei dati in caso di perdita o corruzione.
			\vspace{0.5em}\end{minipage}
				} \\
			\hline
			
			\begin{minipage}[t]{3.6cm}\textbf{Mitigations}\vspace{0.5em}\end{minipage} &
			\multicolumn{2}{|l|}{
				\begin{minipage}[t]{15.6cm}
					\begin{itemize}
						\item Implementare sistemi di conferma per le operazioni critiche;
						\item Effettuare backup regolari dei dati per garantire il recupero in caso di perdita o corruzione;
						\item Implementare sistemi di rollback per il recupero di dati modificati accidentalmente;
					\end{itemize}
				\vspace{0.1em}\end{minipage}
			} \\
			\hline
			
			\begin{minipage}[t]{3.6cm}\textbf{Comments}\vspace{0.5em}\end{minipage} &
			\multicolumn{2}{|l|}{
				\begin{minipage}[t]{15.6cm}
					/
				\vspace{0.5em}\end{minipage}
			} \\
			\hline
			
		\end{tabular}
		}
		}
\end{table} \clearpage