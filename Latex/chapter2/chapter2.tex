\chapter{Valutazione del rischio}


\section{Diagrammi i*}

\subsection{SD/SR ruoli senza sistema}
% Diagramma i* attori senza sistema
\begin{figure}[h!]
	\centering
	\includegraphics[width=1\columnwidth]{chapter2/immagini/2\_model\_solo\_attori\_principali\_no\_sistema.png}
\end{figure}

\clearpage

\subsection{SD/SR ruoli con sistema}

% Diagramma i* attori con sistema
\begin{figure}[h!]
	\centering
	\includegraphics[width=0.8\columnwidth]{chapter2/immagini/3\_model\_con\_sistema.png}
\end{figure}

\subsection{SD/SR sistema e attaccanti con alberi di attacco}

% Diagramma i* attori con alberi di attacco
\begin{figure}[h!]	\centering
	\includegraphics[width=0.9\columnwidth]{chapter2/immagini/4\_model\_con\_alberi\_di\_attacco.png}
\end{figure}
\clearpage
\subsection{SD/SR sistema e oracolo bayesiano}

% Diagramma i* attori con oracolo
\begin{figure}[h!]
	\centering
	\includegraphics[width=1\columnwidth]{chapter2/immagini/5\_model\_con\_oracolo.png}
\end{figure}
% Prima pagina: sezione + PDF in landscape

\newgeometry{paperwidth=297mm,paperheight=210mm,margin=2cm}
\includepdf[
  pages=1,
  fitpaper=true,
  scale=0.85,        
  pagecommand={\section{Tabella Dual Stride} %qui si può scrivere
  }
]{chapter2/dual-stride.pdf}
\restoregeometry

\newgeometry{paperwidth=297mm,paperheight=210mm,margin=2cm} % A4 landscape
\includepdf[pages=2-,fitpaper=true,pagecommand={}]{chapter2/dual-stride.pdf}
\restoregeometry % torna al layout originale portrait


\section{Abuse Case}
In questa sezione, vengono riportati gli schemi di Jacobson relativi agli abuse case. In particolare si analizzerà ogni tipologia di attacco individuato nella tabella Dual Stride.
%%%%%%%%%% Authentication Abuse (CAPEC 114) %%%%%%%%%%%%%%%
\renewcommand{\arraystretch}{1.5} % aumenta l'altezza delle righe del 50%
\begin{table}[ht!]
	\centering
	{\footnotesize
		\begin{tabular}{|l |l| l|}
			\hline
			\begin{minipage}[t]{3cm}\textbf{Case Type}\end{minipage} &
			\begin{minipage}[t]{6cm}\textbf{Abuse Case}\end{minipage} &
			\begin{minipage}[t]{5cm}\textbf{Case ID} AC-01 \end{minipage} \\ \hline
			
			\begin{minipage}[t]{3cm}\textbf{Case Name}\vspace{0.5em}\end{minipage} &
			\multicolumn{2}{|l|}{
				\begin{minipage}[t]{11cm}\raggedright
					Authentication Abuse (CAPEC 114) 
				\vspace{0.5em}\end{minipage}
			} \\
			\hline
			
			\begin{minipage}[t]{3cm}\textbf{Actors}\vspace{0.5em}\end{minipage} &
			\multicolumn{2}{|l|}{
				\begin{minipage}[t]{11cm}\raggedright
					Sistema, Attaccante
				\vspace{0.5em}\end{minipage}
			} \\
			\hline
			
			\begin{minipage}[t]{3cm}\textbf{Description}\vspace{0.5em}\end{minipage} &
			\multicolumn{2}{|l|}{
				\begin{minipage}[t]{11cm}\raggedright
					Un aggressore ottiene l'accesso non autorizzato a un'applicazione, un servizio o un dispositivo conoscendo le debolezze intrinseche di un meccanismo di autenticazione o sfruttando una falla nell'implementazione dello schema di autenticazione. In un attacco di questo tipo, un meccanismo di autenticazione funziona, ma una sequenza di eventi attentamente controllata fa sì che il meccanismo conceda l'accesso all'aggressore.
				\vspace{0.5em}\end{minipage}
			} \\
			\hline
			
			\begin{minipage}[t]{3cm}\textbf{Data}\vspace{0.5em}\end{minipage} &
			\multicolumn{2}{|l|}{
				\begin{minipage}[t]{11cm}\raggedright
					Dati utente di sistema
				\vspace{0.5em}\end{minipage}
			} \\
			\hline
			
			\begin{minipage}[t]{3cm}\raggedright\textbf{Stimulus and Preconditions}\vspace{0.5em}\end{minipage} &
			\multicolumn{2}{|l|}{
				\begin{minipage}[t]{11cm}\raggedright
					\begin{itemize}
						\item Un meccanismo o sottosistema di autenticazione che implementa una qualche forma di autenticazione, come password, certificati di sicurezza, ecc., che presenta qualche difetto/vulnerabilità;
						\item Un'applicazione client, un accesso da riga di comando a un file binario o un linguaggio di programmazione in grado di interagire con il meccanismo di autenticazione.
					\end{itemize}
				\vspace{0.5em}\end{minipage}
			} \\
			\hline
			
			\begin{minipage}[t]{3cm}\textbf{Attack Flow 1}\vspace{0.5em}\end{minipage} &
			\multicolumn{2}{|l|}{
				\begin{minipage}[t]{11cm}\raggedright 
					Un attaccante sfrutta una debolezza nel meccanismo di autenticazione, permettendogli di eseguire azioni illegalmente. Per esempio, dopo aver bypassato l'autenticazione, potrebbe effettuare azioni come visionare i dati contenuti nel sistema o azioni di modifica dei dati.
				\vspace{0.5em}\end{minipage}
			} \\
			\hline
			
			\begin{minipage}[t]{3cm}\textbf{Attack Flow 2}\vspace{0.5em}\end{minipage} &
			\multicolumn{2}{|l|}{
				\begin{minipage}[t]{11cm}\raggedright 
					/
				\vspace{0.5em}\end{minipage}
			} \\
			\hline
			
			\begin{minipage}[t]{3cm}\textbf{Attack Flow 3}\vspace{0.5em}\end{minipage} &
			\multicolumn{2}{|l|}{
				\begin{minipage}[t]{11cm} 
					/
				\vspace{0.5em}\end{minipage}
			} \\
			\hline
			
			\begin{minipage}[t]{3cm}\raggedright\textbf{Response and Postconditions}\vspace{0.5em}\end{minipage} &
			\multicolumn{2}{|l|}{
				\begin{minipage}[t]{11cm}\raggedright 
					L’aggressore ottiene l’accesso non autorizzato ed effettua azioni malevoli.
				\vspace{0.5em}\end{minipage}
			} \\
			\hline
			
			\begin{minipage}[t]{3cm}\raggedright\textbf{Non Functional Requirements}\vspace{0.5em}\end{minipage} &
		\multicolumn{2}{|l|}{
			\begin{minipage}[t]{11cm}\raggedright
				Garantire che tutti i meccanismi di autenticazione siano sicuri e a più fattori, includendo protezioni contro l’elevazione dei privilegi non autorizzata. Effettuare audit regolari per rilevare configurazioni errate o credenziali
				compromesse.
			\vspace{0.5em}\end{minipage}
		} \\
			\hline
			
			\begin{minipage}[t]{3cm}\textbf{Mitigations}\vspace{0.5em}\end{minipage} &
			\multicolumn{2}{|l|}{
				\begin{minipage}[t]{11cm}\raggedright
					\begin{itemize}
						\item Utilizzare autenticazione MFA;
					\end{itemize}
				\vspace{0.5em}\end{minipage}
			} \\
			\hline
			
			\begin{minipage}[t]{3cm}\textbf{Comments}\vspace{0.5em}\end{minipage} &
			\multicolumn{2}{|l|}{
				\begin{minipage}[t]{11cm}\raggedright
					Gli attacchi sui sistemi di autenticazione sono una delle cause principali di compromissioni. La combinazione di MFA, auditing continuo e l'uso di password robuste può ridurre significativamente il rischio.
				\vspace{0.5em}\end{minipage}
			} \\
			\hline
			
		\end{tabular}
	}

\end{table}

%%%%%%%%%%%%%%%%%%% Privilage Escalation (CAPEC 233) %%%%%%%%%%%%%%%%%%%%%%%%%

\begin{table}[ht!]
	\centering
	{\footnotesize
		\begin{tabular}{|l |l| l|}
			\hline
			\begin{minipage}[t]{3cm}\textbf{Case Type}\end{minipage} &
			\begin{minipage}[t]{6cm}\textbf{Abuse Case}\end{minipage} &
			\begin{minipage}[t]{5cm}\textbf{Case ID} AC-02 \end{minipage} \\ \hline
			
			\begin{minipage}[t]{3cm}\textbf{Case Name}\vspace{0.5em}\end{minipage} &
			\multicolumn{2}{|l|}{
				\begin{minipage}[t]{11cm}\raggedright
					Privilage Escalation (CAPEC 233) 
				\vspace{0.5em}\end{minipage}
			} \\
			\hline
			
			\begin{minipage}[t]{3cm}\textbf{Actors}\vspace{0.5em}\end{minipage} &
			\multicolumn{2}{|l|}{
				\begin{minipage}[t]{11cm}\raggedright
					Sistema, Attaccante
				\vspace{0.5em}\end{minipage}
			} \\
			\hline
			
			\begin{minipage}[t]{3cm}\textbf{Description}\vspace{0.5em}\end{minipage} &
			\multicolumn{2}{|l|}{
				\begin{minipage}[t]{11cm}\raggedright
					Un avversario sfrutta una debolezza che gli consente di elevare i propri privilegi ed eseguire azioni che non dovrebbe essere autorizzato a eseguire.				\vspace{0.5em}\end{minipage}
			} \\
			\hline
			
			\begin{minipage}[t]{3cm}\textbf{Data}\vspace{0.5em}\end{minipage} &
			\multicolumn{2}{|l|}{
				\begin{minipage}[t]{11cm}\raggedright
					Sistema, Attaccante
				\vspace{0.5em}\end{minipage}
			} \\
			\hline
			
			\begin{minipage}[t]{3cm}\raggedright\textbf{Stimulus and Preconditions}\vspace{0.5em}\end{minipage} &
			\multicolumn{2}{|l|}{
				\begin{minipage}[t]{11cm}\raggedright
					\begin{itemize}
        				\item Il sistema contiene un meccanismo di controllo dei privilegi mal configurato o vulnerabile
        				\item L'attaccante ha un accesso limitato al sistema.
        				\item L'attaccante conosce o riesce a dedurre una vulnerabilità sfruttabile nel controllo dei privilegi.
      				\end{itemize}
				\vspace{0.5em}\end{minipage}
			} \\
			\hline
			
			\begin{minipage}[t]{3cm}\textbf{Attack Flow 1}\vspace{0.5em}\end{minipage} &
			\multicolumn{2}{|l|}{
				\begin{minipage}[t]{11cm}\raggedright 
					Gli attaccanti possono eseguire il caching sudo e/o utilizzare il file "sudoers" per elevare i propri privilegi. Ciò permette di eseguire comandi al posto di altri utenti o per generare processi con privilegi più elevati.
				\vspace{0.5em}\end{minipage}
			} \\
			\hline
			
			\begin{minipage}[t]{3cm}\textbf{Attack Flow 2}\vspace{0.5em}\end{minipage} &
			\multicolumn{2}{|l|}{
				\begin{minipage}[t]{11cm}\raggedright 
					L’attaccante sfrutta la configurazione errata del meccanismo di controllo degli account utente, consentendo l’esecuzione di codice arbitrario con privilegi elevati senza richiedere autorizzazioni
				\vspace{0.5em}\end{minipage}
			} \\
			\hline
			
			\begin{minipage}[t]{3cm}\textbf{Attack Flow 3}\vspace{0.5em}\end{minipage} &
			\multicolumn{2}{|l|}{
				\begin{minipage}[t]{11cm} 
					/
				\vspace{0.5em}\end{minipage}
			} \\
			\hline
			
			\begin{minipage}[t]{3cm}\raggedright\textbf{Response and Postconditions}\vspace{0.5em}\end{minipage} &
			\multicolumn{2}{|l|}{
				\begin{minipage}[t]{11cm}\raggedright 
					I privilegi aumentati vengono utilizzati per eseguire operazioni non autorizzate.
				\vspace{0.5em}\end{minipage}
			} \\
			\hline
			
			\begin{minipage}[t]{3cm}\raggedright\textbf{Non Functional Requirements}\vspace{0.5em}\end{minipage} &
		\multicolumn{2}{|l|}{
			\begin{minipage}[t]{11cm}\raggedright
				Deve essere garantita una gestione dei privilegi sicura, proteggendo i meccanismi di escalation e prevenendo configurazioni usate. Monitoraggio continuo e auditing dei privilegi devono essere implementati per rilevare tentativi di escalation non autorizzati
			\vspace{0.5em}\end{minipage}
		} \\
			\hline
			
			\begin{minipage}[t]{3cm}\textbf{Mitigations}\vspace{0.5em}\end{minipage} &
			\multicolumn{2}{|l|}{
				\begin{minipage}[t]{11cm}\raggedright
					\begin{itemize}
        				\item Applicare MFA per azioni sensibili;
        				\item Controllare periodicamente file di configurazione;
        				\item Effettuare audit periodici del sistema;
        				\item Mantenere aggiornati i programmi, le librerie e framework in uso;
      				\end{itemize}
				\vspace{0.5em}\end{minipage}
			} \\
			\hline
			
			\begin{minipage}[t]{3cm}\textbf{Comments}\vspace{0.5em}\end{minipage} &
			\multicolumn{2}{|l|}{
				\begin{minipage}[t]{11cm}\raggedright
					È fondamentale combinare l'autenticazione MFA, monitorare continuamente il sistema, mantenere aggiornate le varei tecnologie in uso e gestire rigorosamente i privilegi e i file di configurazione per prevenire vulnerabilità di questo tipo.
				\vspace{0.5em}\end{minipage}
			} \\
			\hline
			
		\end{tabular}
	}
\end{table}

%%%%%%%%%%%%%%%%%% Targeted Malware (CAPEC 542) %%%%%%%%%%%%%%%%%%%%%%

\begin{table}[ht!]
	\centering
	{\footnotesize
		\begin{tabular}{|l |l| l|}
			\hline
			\begin{minipage}[t]{3cm}\textbf{Case Type}\end{minipage} &
			\begin{minipage}[t]{6cm}\textbf{Abuse Case}\end{minipage} &
			\begin{minipage}[t]{5cm}\textbf{Case ID} AC-03 \end{minipage} \\ \hline
			
			\begin{minipage}[t]{3cm}\textbf{Case Name}\vspace{0.5em}\end{minipage} &
			\multicolumn{2}{|l|}{
				\begin{minipage}[t]{11cm}\raggedright
					Targeted Malware (CAPEC 542) 
				\vspace{0.5em}\end{minipage}
			} \\
			\hline
			
			\begin{minipage}[t]{3cm}\textbf{Actors}\vspace{0.5em}\end{minipage} &
			\multicolumn{2}{|l|}{
				\begin{minipage}[t]{11cm}\raggedright
					Cliente, Corriere, Sistema, Attaccante
				\vspace{0.5em}\end{minipage}
			} \\
			\hline
			
			\begin{minipage}[t]{3cm}\textbf{Description}\vspace{0.5em}\end{minipage} &
			\multicolumn{2}{|l|}{
				\begin{minipage}[t]{11cm}\raggedright
					Un avversario sviluppa un malware mirato che sfrutta una vulnerabilità nota in un ambiente informatico organizzativo. Il malware creato per questi attacchi si basa specificamente sulle informazioni raccolte sull'ambiente tecnologico. L'esecuzione con successo del malware consente a un avversario di ottenere un'ampia varietà di impatti tecnici negativi	
				\vspace{0.5em}\end{minipage}
			} \\
			\hline
			
			\begin{minipage}[t]{3cm}\textbf{Data}\vspace{0.5em}\end{minipage} &
			\multicolumn{2}{|l|}{
				\begin{minipage}[t]{11cm}\raggedright
					Dati utente di sistema, Dati cliente, Dati pagamento, Dati corriere
				\vspace{0.5em}\end{minipage}
			} \\
			\hline
			
			\begin{minipage}[t]{3cm}\raggedright\textbf{Stimulus and Preconditions}\vspace{0.5em}\end{minipage} &
			\multicolumn{2}{|l|}{
				\begin{minipage}[t]{11cm}\raggedright
					\begin{itemize}
        				\item L'attaccante deve raccogliere informazioni sull'ambiente target.
        				\item L'attaccante deve provare modi di social engineering come il phising per far eseguire il malware sul sistema. 
        				\item Identificare vulnerabilità e creare malware ad hoc per sfruttarle.
      				\end{itemize}
				\vspace{0.5em}\end{minipage}
			} \\
			\hline
			
			\begin{minipage}[t]{3cm}\textbf{Attack Flow 1}\vspace{0.5em}\end{minipage} &
			\multicolumn{2}{|l|}{
				\begin{minipage}[t]{11cm}\raggedright 
					L'attaccante analizza il sistema cercando di ottenere informazioni utili per lo sviluppo di un malware ed il suo deploy. Successivamente, dopo lo sviluppo del malware, l'attaccante tenta il deploy di quest'ultimo o tramite tecniche di social engineering come il phising, andando ad inviare email o messaggi fasulli agli utenti interessati.
				\vspace{0.5em}\end{minipage}
			} \\
			\hline
			
			\begin{minipage}[t]{3cm}\textbf{Attack Flow 2}\vspace{0.5em}\end{minipage} &
			\multicolumn{2}{|l|}{
				\begin{minipage}[t]{11cm}\raggedright 
					L'attaccante, sempre dopo aver analizzato il sistema e sviluppato un malware, sfrutta una o più vulnerabilità riscontrate nel sistema per effettuare il deploy del malware.
				\vspace{0.5em}\end{minipage}
			} \\
			\hline
			
			\begin{minipage}[t]{3cm}\textbf{Attack Flow 3}\vspace{0.5em}\end{minipage} &
			\multicolumn{2}{|l|}{
				\begin{minipage}[t]{11cm} 
					/
				\vspace{0.5em}\end{minipage}
			} \\
			\hline
			
			\begin{minipage}[t]{3cm}\raggedright\textbf{Response and Postconditions}\vspace{0.5em}\end{minipage} &
			\multicolumn{2}{|l|}{
				\begin{minipage}[t]{11cm}\raggedright 
					L'aggressore riesce ad eseguire malware mirati contro il sistema.
				\vspace{0.5em}\end{minipage}
			} \\
			\hline
			
			\begin{minipage}[t]{3cm}\raggedright\textbf{Non Functional Requirements}\vspace{0.5em}\end{minipage} &
		\multicolumn{2}{|l|}{
			\begin{minipage}[t]{11cm}\raggedright
				Garantire che il sistema effetui un monitoraggio avanzato per rilevare le minacce.
			\vspace{0.5em}\end{minipage}
				} \\
			\hline
			
			\begin{minipage}[t]{3cm}\textbf{Mitigations}\vspace{0.5em}\end{minipage} &
			\multicolumn{2}{|l|}{
				\begin{minipage}[t]{11cm}\raggedright
					\begin{itemize}
        				\item Mantenere aggiornati sistemi e software;
        				\item Implementare strumenti di rilevamento come antivirus e firewall;
        				\item Effettuare degli audit del sistema andando ad analizzare manualmente processi e servizi in esecuzione.
      				\end{itemize}
				\vspace{0.5em}\end{minipage}
			} \\
			\hline
			
			\begin{minipage}[t]{3cm}\textbf{Comments}\vspace{0.5em}\end{minipage} &
			\multicolumn{2}{|l|}{
				\begin{minipage}[t]{11cm}\raggedright
					Gli avversari spesso utilizzano tecniche di offuscamento quando sviluppano malware allo scopo di evitare il rilevamento o impedire al bersaglio di decodificare e comprendere un campione di malware catturato. Alcune di queste tecniche includono, ma non sono limitate a, il riempimento binario, l'imballaggio del software, la rimozione di simboli e stringhe da un payload e l'utilizzo di una risoluzione API dinamica.
				\vspace{0.5em}\end{minipage}
			} \\
			\hline
			
		\end{tabular}
	}
\end{table}

%%%%%%%%%%%%%%%% Phishing (CAPEC 98) %%%%%%%%%%%%%%%%%%%


\begin{table}[ht!]
	\centering
	{\footnotesize
		\begin{tabular}{|l |l| l|}
			\hline
			\begin{minipage}[t]{3cm}\textbf{Case Type}\end{minipage} &
			\begin{minipage}[t]{6cm}\textbf{Abuse Case}\end{minipage} &
			\begin{minipage}[t]{5cm}\textbf{Case ID} AC-04 \end{minipage} \\ \hline
			
			\begin{minipage}[t]{3cm}\textbf{Case Name}\vspace{0.5em}\end{minipage} &
			\multicolumn{2}{|l|}{
				\begin{minipage}[t]{11cm}\raggedright
					Phishing (CAPEC 98) 
				\vspace{0.5em}\end{minipage}
			} \\
			\hline
			
			\begin{minipage}[t]{3cm}\textbf{Actors}\vspace{0.5em}\end{minipage} &
			\multicolumn{2}{|l|}{
				\begin{minipage}[t]{11cm}\raggedright
					Cliente, Sistema, Attaccante
				\vspace{0.5em}\end{minipage}
			} \\
			\hline
			
			\begin{minipage}[t]{3cm}\textbf{Description}\vspace{0.5em}\end{minipage} &
			\multicolumn{2}{|l|}{
				\begin{minipage}[t]{11cm}\raggedright
					Il phishing è una tecnica di ingegneria sociale in cui un utente malintenzionato si maschera da entità legittima con la quale la vittima potrebbe fare affari al fine di indurre l'utente a rivelare alcune informazioni riservate (molto spesso credenziali di autenticazione) che possono essere successivamente utilizzate da un malintenzionato. Il phishing è essenzialmente una forma di raccolta di informazioni o "pesca" per informazioni	
				\vspace{0.5em}\end{minipage}
			} \\
			\hline
			
			\begin{minipage}[t]{3cm}\textbf{Data}\vspace{0.5em}\end{minipage} &
			\multicolumn{2}{|l|}{
				\begin{minipage}[t]{11cm}\raggedright
					Dati utente di sistema, Dati cliente, Dati pagamento, Dati corriere
				\vspace{0.5em}\end{minipage}
			} \\
			\hline
			
			\begin{minipage}[t]{3cm}\raggedright\textbf{Stimulus and Preconditions}\vspace{0.5em}\end{minipage} &
			\multicolumn{2}{|l|}{
				\begin{minipage}[t]{11cm}\raggedright
					\begin{itemize}
        				\item Presenza di una vulnerabilità nota nell’ambiente target, non ancora corretta o mitigata. 
        				\item Un aggressore deve avere un modo per entrare in contatto con la vittima (per esempio tramite e-mail);
        				\item Capacità dell’attaccante di raccogliere informazioni tramite ricognizione (OSINT, scansioni passive o altre tecniche non invasive)
        				\item L'aggressore deve ottenere la fiducia della vittima, inducendola con l'inganno a compiere determinate azioni.
        				\item Il servizio ingennevole deve assomigliare il più possibile a quello reale.
      				\end{itemize}
				\vspace{0.5em}\end{minipage}
			} \\
			\hline
			
			\begin{minipage}[t]{3cm}\textbf{Attack Flow 1}\vspace{0.5em}\end{minipage} &
			\multicolumn{2}{|l|}{
				\begin{minipage}[t]{11cm}\raggedright 
					Un aggressore invia un'e-mail alla vittima malevola per indurre l'utente a cliccare sul link incluso nell'e-mail (che indirizza la vittima al sito web dell'aggressore) e ad accedere. La chiave è far credere alla vittima che l'e-mail provenga da un'entità legittima e che il sito web a cui rimanda l'URL nell'e-mail sia il sito web legittimo. Un invito all'azione deve solitamente suonare legittimo e sufficientemente urgente da indurre l'utente ad agire.
				\vspace{0.5em}\end{minipage}
			} \\
			\hline
			
			\begin{minipage}[t]{3cm}\textbf{Attack Flow 2}\vspace{0.5em}\end{minipage} &
			\multicolumn{2}{|l|}{
				\begin{minipage}[t]{11cm}\raggedright 
					Una volta che l'aggressore ottiene alcune informazioni sensibili tramite phishing (credenziali di accesso, dati della carta di credito, ecc.), può sfruttare queste informazioni. Ad esempio, può utilizzare le credenziali di accesso della vittima per accedere al suo conto bancario e trasferire denaro su un conto a sua scelta.
				\vspace{0.5em}\end{minipage}
			} \\
			\hline
			
			\begin{minipage}[t]{3cm}\textbf{Attack Flow 3}\vspace{0.5em}\end{minipage} &
			\multicolumn{2}{|l|}{
				\begin{minipage}[t]{11cm} 
					Un aggressore crea un sito web che assomiglia molto al sito web che sta cercando di impersonare. Tale sito web in genere include un modulo di accesso in cui la vittima deve inserire le proprie credenziali di autenticazione.
				\vspace{0.5em}\end{minipage}
			} \\
			\hline
			
			\begin{minipage}[t]{3cm}\raggedright\textbf{Response and Postconditions}\vspace{0.5em}\end{minipage} &
			\multicolumn{2}{|l|}{
				\begin{minipage}[t]{11cm}\raggedright 
					L'aggressore riesce ad ottenere le informazioni riservate.
				\vspace{0.5em}\end{minipage}
			} \\
			\hline
			
			\begin{minipage}[t]{3cm}\raggedright\textbf{Non Functional Requirements}\vspace{0.5em}\end{minipage} &
		\multicolumn{2}{|l|}{
			\begin{minipage}[t]{11cm}\raggedright
				Garantire l'implementazione di filtri avanzati, il monitoraggio e l'analisi delle attività anomale.
			\vspace{0.5em}\end{minipage}
				} \\
			\hline
			
			\begin{minipage}[t]{3cm}\textbf{Mitigations}\vspace{0.5em}\end{minipage} &
			\multicolumn{2}{|l|}{
				\begin{minipage}[t]{11cm}\raggedright
					Non seguire alcun link che si riceve all'interno delle e-mail e non inserire credenziali di accesso su alcun sito web proveniente da e-mail sospette.
				\vspace{0.5em}\end{minipage}
			} \\
			\hline
			
			\begin{minipage}[t]{3cm}\textbf{Comments}\vspace{0.5em}\end{minipage} &
			\multicolumn{2}{|l|}{
				\begin{minipage}[t]{11cm}\raggedright
					Questo CAPEC descrive un attacco in cui un avversario sviluppa malware su misura per sfruttare debolezze note dell’ambiente target.
				\vspace{0.5em}\end{minipage}
			} \\
			\hline
			
		\end{tabular}
	}
\end{table}

%%%%%%%%%%%%%% Adversary in the middle(AiTM) (CAPEC 94) %%%%%%%%%%%%%%%

% per tabelle troppo "lunghe" che strabordano in basso, allargare dimensione totale della
% tabella e usare hspace per ricentrarlo

\begin{table}[ht!]
\hspace*{-1.8cm} % sposta la tabella verso sinistra
\centering
{\footnotesize
\begin{tabular}{|l|l|l|}
\hline
\begin{minipage}[t]{4.5cm}\textbf{Case Type}\end{minipage} &
\begin{minipage}[t]{7cm}\textbf{Abuse Case}\end{minipage} &
\begin{minipage}[t]{4.5cm}\textbf{Case ID} AC-05\end{minipage} \\ \hline

\begin{minipage}[t]{4.5cm}\textbf{Case Name}\vspace{0.5em}\end{minipage} &
\multicolumn{2}{|l|}{
\begin{minipage}[t]{11.5cm}\raggedright
Adversary in the middle (AiTM) (CAPEC 94)
\vspace{0.5em}\end{minipage}} \\ \hline

\begin{minipage}[t]{4.5cm}\textbf{Actors}\vspace{0.5em}\end{minipage} &
\multicolumn{2}{|l|}{
\begin{minipage}[t]{11.5cm}\raggedright
Cliente, Sistema, Attaccante
\vspace{0.5em}\end{minipage}} \\ \hline

\begin{minipage}[t]{4.5cm}\textbf{Description}\vspace{0.5em}\end{minipage} &
\multicolumn{2}{|l|}{
\begin{minipage}[t]{11.5cm}\raggedright
Ogni volta che un componente tenta di comunicare con l'altro, i dati fluiscono prima attraverso l'avversario, che può osservarli o alterarli, prima di essere trasmessi al destinatario previsto come se non fossero mai stati osservati. Questa interposizione è trasparente lasciando i due componenti compromessi inconsapevoli della potenziale corruzione o perdita delle loro comunicazioni. Il potenziale di questi attacchi produce un'implicita mancanza di fiducia nella comunicazione o nell'identificazione tra due componenti.
\vspace{0.5em}\end{minipage}} \\ \hline

\begin{minipage}[t]{4.5cm}\textbf{Data}\vspace{0.5em}\end{minipage} &
\multicolumn{2}{|l|}{
\begin{minipage}[t]{11.5cm}\raggedright
Dati utente di sistema, Dati cliente, Dati pagamento, Dati corriere
\vspace{0.5em}\end{minipage}} \\ \hline

\begin{minipage}[t]{4.5cm}\textbf{Stimulus and Preconditions}\vspace{0.5em}\end{minipage} &
\multicolumn{2}{|l|}{
\begin{minipage}[t]{11.5cm}\raggedright
\begin{itemize}
\item Ci sono due componenti che comunicano tra loro;
\item Un utente malintenzionato è in grado di identificare la natura e il meccanismo di comunicazione tra i due componenti bersaglio;
\item Un utente malintenzionato può origliare la comunicazione tra i componenti bersaglio;
\item Una forte autenticazione reciproca non viene utilizzata tra i due componenti bersaglio;
\item La comunicazione avviene in chiaro (non crittografato) o con crittografia insufficiente e falsificabile.
\end{itemize}
\vspace{0.5em}\end{minipage}} \\ \hline

\begin{minipage}[t]{4.5cm}\textbf{Attack Flow 1}\vspace{0.5em}\end{minipage} &
\multicolumn{2}{|l|}{
\begin{minipage}[t]{11.5cm}\raggedright
L'attaccante si interpone nella rete con uno spyware installato sul sistema. Ciò permette di registrare tutto il traffico che transita nella rete. Se il traffico non è criptato, l'attaccante può leggere tutte le comunicazioni in chiaro, altrimenti deve decriptarle.
\vspace{0.5em}\end{minipage}} \\ \hline

\begin{minipage}[t]{4.5cm}\textbf{Attack Flow 2}\vspace{0.5em}\end{minipage} &
\multicolumn{2}{|l|}{
\begin{minipage}[t]{11.5cm}\raggedright
L'attaccante può "avvelenare" la cache ARP (Address Resolution Protocol) per posizionarsi tra le comunicazioni di due o più dispositivi in rete.
\vspace{0.5em}\end{minipage}} \\ \hline

\begin{minipage}[t]{4.5cm}\textbf{Attack Flow 3}\vspace{0.5em}\end{minipage} &
\multicolumn{2}{|l|}{
\begin{minipage}[t]{11.5cm}\raggedright
L'attaccante può reindirizzare il traffico di rete verso sistemi di sua proprietà falsificando il traffico DHCP e comportandosi come server DHCP dannoso. Raggiungendo la posizione di "avversario nel mezzo" (AiTM), gli aggressori possono raccogliere le comunicazioni di rete, comprese le credenziali in transito tramite protocolli non sicuri.
\vspace{0.5em}\end{minipage}} \\ \hline

\begin{minipage}[t]{4.5cm}\textbf{Response and Postconditions}\vspace{0.5em}\end{minipage} &
\multicolumn{2}{|l|}{
\begin{minipage}[t]{11.5cm}\raggedright
L’aggressore riesce ad inserirsi nel canale di comunicazione tra due o più componenti.
\vspace{0.5em}\end{minipage}} \\ \hline

\begin{minipage}[t]{4.5cm}\textbf{Non Functional Requirements}\vspace{0.5em}\end{minipage} &
\multicolumn{2}{|l|}{
\begin{minipage}[t]{11.5cm}\raggedright
\begin{itemize}
\item Comunicazioni sicure tramite TLS 1.3, SSH, SSL.
\item Gestione sicura di chiavi, certificati, handshake.
\item Integrità dei messaggi tramite MAC o firme digitali.
\end{itemize}
\vspace{0.5em}\end{minipage}} \\ \hline

\begin{minipage}[t]{4.5cm}\textbf{Mitigations}\vspace{0.5em}\end{minipage} &
\multicolumn{2}{|l|}{
\begin{minipage}[t]{11.5cm}\raggedright
\begin{itemize}
\item Chiavi pubbliche firmate da CA attendibili.
\item Crittografia del traffico (SSL/TLS/SSH).
\item Autenticazione reciproca forte.
\item Scambio sicuro delle chiavi pubbliche.
\end{itemize}
\vspace{0.5em}\end{minipage}} \\ \hline

\begin{minipage}[t]{4.5cm}\textbf{Comments}\vspace{0.5em}\end{minipage} &
\multicolumn{2}{|l|}{
\begin{minipage}[t]{11.5cm}\raggedright
Il rischio di manipolazione delle transazioni tramite API è elevato quando la protezione non è adeguata, mediante attacchi AiTM. Questi attacchi differiscono dagli attacchi di sniffing perché modificano i messaggi prima che raggiungano il destinatario previsto.
\vspace{0.5em}\end{minipage}} \\ \hline

\end{tabular}
}
\end{table}



%%%%%%%Reusing session ID (CAPEC 60)%%%%%%%%%%%%%%%%%%%%%

\begin{table}[ht!]
	\resizebox{\textwidth}{!}{
	\centering
	{\footnotesize
		\begin{tabular}{|l |l| l|}
			\hline
			\begin{minipage}[t]{3cm}\textbf{Case Type}\end{minipage} &
			\begin{minipage}[t]{6cm}\textbf{Abuse Case}\end{minipage} &
			\begin{minipage}[t]{5cm}\textbf{Case ID} AC-06 \end{minipage} \\ \hline
			
			\begin{minipage}[t]{3cm}\textbf{Case Name}\vspace{0.5em}\end{minipage} &
			\multicolumn{2}{|l|}{
				\begin{minipage}[t]{11cm}\raggedright
					Reusing session ID (CAPEC 60) 
				\vspace{0.5em}\end{minipage}
			} \\
			\hline
			
			\begin{minipage}[t]{3cm}\textbf{Actors}\vspace{0.5em}\end{minipage} &
			\multicolumn{2}{|l|}{
				\begin{minipage}[t]{11cm}\raggedright
					Utente, Sistema, Attaccante
				\vspace{0.5em}\end{minipage}
			} \\
			\hline
			
			\begin{minipage}[t]{3cm}\textbf{Description}\vspace{0.5em}\end{minipage} &
			\multicolumn{2}{|l|}{
				\begin{minipage}[t]{11cm}\raggedright
					Questo attacco mira al riutilizzo di un ID di sessione valido per falsificare il sistema di destinazione al fine di ottenere privilegi. L'attaccante cerca di riutilizzare un ID di sessione rubato utilizzato in precedenza durante una transazione per eseguire lo spoofing e il dirottamento della sessione.				
				\vspace{0.5em}\end{minipage}
			} \\
			\hline
			
			\begin{minipage}[t]{3cm}\textbf{Data}\vspace{0.5em}\end{minipage} &
			\multicolumn{2}{|l|}{
				\begin{minipage}[t]{11cm}\raggedright
					Dati utente di sistema
				\vspace{0.5em}\end{minipage}
			} \\
			\hline
			
			\begin{minipage}[t]{3cm}\raggedright\textbf{Stimulus and Preconditions}\vspace{0.5em}\end{minipage} &
			\multicolumn{2}{|l|}{
				\begin{minipage}[t]{11cm}\raggedright
					\begin{itemize}
        				\item L'host di destinazione utilizza gli ID di sessione/session token per tenere traccia degli utenti;
        				\item Gli ID di sessione/session token vengono utilizzati per controllare l'accesso alle risorse;
        				\item Gli ID di sessione/session token utilizzati dall'host di destinazione non sono ben protetti dal furto di sessione.
      				\end{itemize}
				\vspace{0.5em}\end{minipage}
			} \\
			\hline
			
			\begin{minipage}[t]{3cm}\textbf{Attack Flow 1}\vspace{0.5em}\end{minipage} &
			\multicolumn{2}{|l|}{
				\begin{minipage}[t]{11cm}\raggedright 
					L'aggressore interagisce con l'host di destinazione e scopre che gli ID di sessione/token di sessione vengono utilizzati per autenticare gli utenti. Successivamente ruba un ID di sessione/token di sessione da un utente valido e lo usa per eseguire azioni per suo conto.
				\vspace{0.5em}\end{minipage}
			} \\
			\hline
			
			\begin{minipage}[t]{3cm}\textbf{Attack Flow 2}\vspace{0.5em}\end{minipage} &
			\multicolumn{2}{|l|}{
				\begin{minipage}[t]{11cm}\raggedright 
				L'aggressore tenta di utilizzare l'ID di sessione rubato per ottenere l'accesso al sistema con i privilegi del proprietario originale dell'ID di sessione.	
				\vspace{0.5em}\end{minipage}
			} \\
			\hline
			
			\begin{minipage}[t]{3cm}\textbf{Attack Flow 3}\vspace{0.5em}\end{minipage} &
			\multicolumn{2}{|l|}{
				\begin{minipage}[t]{11cm} 
						/
					\vspace{0.5em}\end{minipage}
				} \\
			\hline
			
			\begin{minipage}[t]{3cm}\raggedright\textbf{Response and Postconditions}\vspace{0.5em}\end{minipage} &
			\multicolumn{2}{|l|}{
				\begin{minipage}[t]{11cm}\raggedright 
					L'aggressore riesce ad utilizzare lo stesso ID di sessione di un altro utente loggato nel sistema.				\vspace{0.5em}\end{minipage}
					} \\
			\hline
			
			\begin{minipage}[t]{3cm}\raggedright\textbf{Non Functional Requirements}\vspace{0.5em}\end{minipage} &
		\multicolumn{2}{|l|}{
			\begin{minipage}[t]{11cm}\raggedright
				Garantire una gestione sicura delle sessioni, implementando tecniche specifiche per evitare il furto di sessione.			
			\vspace{0.5em}\end{minipage}
				} \\
			\hline
			
			\begin{minipage}[t]{3cm}\textbf{Mitigations}\vspace{0.5em}\end{minipage} &
			\multicolumn{2}{|l|}{
				\begin{minipage}[t]{11cm}\raggedright
					\begin{itemize}
            			\item Invalidare sempre un ID di sessione dopo il logout dell'utente.

            			\item Impostare un timeout di sessione per gli ID di sessione.

            			\item Non codificare l'ID di invio della sessione con il metodo GET, altrimenti l'ID della sessione verrà copiato nell'URL. In generale, evitare di scrivere gli ID di sessione negli URL. Gli URL possono accedere ai file di registro, che sono vulnerabili a un utente malintenzionato.

            			\item Crittografare i dati della sessione associati all'ID della sessione.

            			\item Usare l'autenticazione a più fattori.
        			\end{itemize}				
				\vspace{0.5em}\end{minipage}
			} \\
			\hline
			
			\begin{minipage}[t]{3cm}\textbf{Comments}\vspace{0.5em}\end{minipage} &
			\multicolumn{2}{|l|}{
				\begin{minipage}[t]{11cm}\raggedright
					Questo attacco descrive un attacco in cui l’avversario riutilizza o intercetta un ID di sessione valido per assumere l’identità dell’utente legittimo, sfruttando debolezze nei meccanismi di gestione e protezione delle sessioni.
				\vspace{0.5em}\end{minipage}
			} \\
			\hline
			
		\end{tabular}
		}
	}
\end{table}

%%%%%%%%%%%%%Password brute forcing (CAPEC 49)%%%%%%%%%%%%%%%%%%%%%%%%%%%%%%%

\begin{table}[ht!]
	\centering
	{\footnotesize
			\begin{tabular}{|l |l| l|}
			\hline
			\begin{minipage}[t]{3cm}\textbf{Case Type}\end{minipage} &
			\begin{minipage}[t]{6cm}\textbf{Abuse Case}\end{minipage} &
			\begin{minipage}[t]{5cm}\textbf{Case ID} AC-07 \end{minipage} \\ \hline
			
			\begin{minipage}[t]{3cm}\textbf{Case Name}\vspace{0.5em}\end{minipage} &
			\multicolumn{2}{|l|}{
				\begin{minipage}[t]{11cm}\raggedright
					Password brute forcing (CAPEC 49) 
				\vspace{0.5em}\end{minipage}
			} \\
			\hline
			
			\begin{minipage}[t]{3cm}\textbf{Actors}\vspace{0.5em}\end{minipage} &
			\multicolumn{2}{|l|}{
				\begin{minipage}[t]{11cm}\raggedright
					Sistema, Attaccante
				\vspace{0.5em}\end{minipage}
			} \\
			\hline
			
			\begin{minipage}[t]{3cm}\textbf{Description}\vspace{0.5em}\end{minipage} &
			\multicolumn{2}{|l|}{
				\begin{minipage}[t]{11cm}\raggedright
					Un avversario prova ogni possibile valore per una password finché non ci riesce. Un attacco brute force passerà in rassegna tutte le password possibili, dato l'alfabeto utilizzato (lettere minuscole, lettere maiuscole, numeri, simboli, ecc.) e la lunghezza massima della password.				
				\vspace{0.5em}\end{minipage}
			} \\
			\hline
			
			\begin{minipage}[t]{3cm}\textbf{Data}\vspace{0.5em}\end{minipage} &
			\multicolumn{2}{|l|}{
				\begin{minipage}[t]{11cm}\raggedright
					Dati utente di sistema
				\vspace{0.5em}\end{minipage}
			} \\
			\hline
			
			\begin{minipage}[t]{3cm}\raggedright\textbf{Stimulus and Preconditions}\vspace{0.5em}\end{minipage} &
			\multicolumn{2}{|l|}{
				\begin{minipage}[t]{11cm}\raggedright
					\begin{itemize}
        				\item L'host di destinazione utilizza gli ID di sessione/session token per tenere traccia degli utenti;
        				\item Gli ID di sessione/session token vengono utilizzati per controllare l'accesso alle risorse;
        				\item Gli ID di sessione/session token utilizzati dall'host di destinazione non sono ben protetti dal furto di sessione.
      				\end{itemize}
				\vspace{0.5em}\end{minipage}
			} \\
			\hline
			
			\begin{minipage}[t]{3cm}\textbf{Attack Flow 1}\vspace{0.5em}\end{minipage} &
			\multicolumn{2}{|l|}{
				\begin{minipage}[t]{11cm}\raggedright 
					L'aggressore interagisce con l'host di destinazione e scopre che gli ID di sessione/token di sessione vengono utilizzati per autenticare gli utenti. Successivamente ruba un ID di sessione/token di sessione da un utente valido e lo usa per eseguire azioni per suo conto.
				\vspace{0.5em}\end{minipage}
			} \\
			\hline
			
			\begin{minipage}[t]{3cm}\textbf{Attack Flow 2}\vspace{0.5em}\end{minipage} &
			\multicolumn{2}{|l|}{
				\begin{minipage}[t]{11cm}\raggedright 
					L'aggressore tenta di utilizzare l'ID di sessione rubato per ottenere l'accesso al sistema con i privilegi del proprietario originale dell'ID di sessione.				
				\vspace{0.5em}\end{minipage}
			} \\
			\hline
			
			\begin{minipage}[t]{3cm}\textbf{Attack Flow 3}\vspace{0.5em}\end{minipage} &
			\multicolumn{2}{|l|}{
				\begin{minipage}[t]{11cm} 
						/
					\vspace{0.5em}\end{minipage}
				} \\
			\hline
			
			\begin{minipage}[t]{3cm}\raggedright\textbf{Response and Postconditions}\vspace{0.5em}\end{minipage} &
			\multicolumn{2}{|l|}{
				\begin{minipage}[t]{11cm}\raggedright 
					L'aggressore riesce ad utilizzare lo stesso ID di sessione di un altro utente loggato nel sistema.				
				\vspace{0.5em}\end{minipage}
					} \\
			\hline
			
			\begin{minipage}[t]{3cm}\raggedright\textbf{Non Functional Requirements}\vspace{0.5em}\end{minipage} &
		\multicolumn{2}{|l|}{
			\begin{minipage}[t]{11cm}\raggedright
				Garantire una gestione sicura delle sessioni, implementando tecniche specifiche per evitare il furto di sessione.			
			\vspace{0.5em}\end{minipage}
				} \\
			\hline
			
			\begin{minipage}[t]{3cm}\textbf{Mitigations}\vspace{0.5em}\end{minipage} &
			\multicolumn{2}{|l|}{
				\begin{minipage}[t]{11cm}\raggedright
					\begin{itemize}
            			\item Invalidare sempre un ID di sessione dopo il logout dell'utente.

            			\item Impostare un timeout di sessione per gli ID di sessione.

            			\item Non codificare l'ID di invio della sessione con il metodo GET, altrimenti l'ID della sessione verrà copiato nell'URL. In generale, evitare di scrivere gli ID di sessione negli URL. Gli URL possono accedere ai file di registro, che sono vulnerabili a un utente malintenzionato.

            			\item Crittografare i dati della sessione associati all'ID della sessione.

            			\item Usare l'autenticazione a più fattori.
        			\end{itemize}				
				\vspace{0.5em}\end{minipage}
			} \\
			\hline
			
			\begin{minipage}[t]{3cm}\textbf{Comments}\vspace{0.5em}\end{minipage} &
			\multicolumn{2}{|l|}{
				\begin{minipage}[t]{11cm}\raggedright
					Questo attacco descrive un attacco in cui l’avversario riutilizza o intercetta un ID di sessione valido per assumere l’identità dell’utente legittimo, sfruttando debolezze nei meccanismi di gestione e protezione delle sessioni.
				\vspace{0.5em}\end{minipage}
			} \\
			\hline
			
		\end{tabular}
	}
\end{table}
%%%%%%%%%%%%%%%%%% Try common or default usernames and passwords (CAPEC 70) %%%%%%%%%%%%%%%%%%%%%%%%%%

\begin{table}[ht!]
	\centering
	{\footnotesize
		\begin{tabular}{|l |l| l|}
			\hline
			\begin{minipage}[t]{3cm}\textbf{Case Type}\end{minipage} &
			\begin{minipage}[t]{6cm}\textbf{Abuse Case}\end{minipage} &
			\begin{minipage}[t]{5cm}\textbf{Case ID} AC-08 \end{minipage} \\ \hline
			
			\begin{minipage}[t]{3cm}\textbf{Case Name}\vspace{0.5em}\end{minipage} &
			\multicolumn{2}{|l|}{
				\begin{minipage}[t]{11cm}\raggedright
					Try common or default usernames and passwords (CAPEC 70) 
				\vspace{0.5em}\end{minipage}
			} \\
			\hline
			
			\begin{minipage}[t]{3cm}\textbf{Actors}\vspace{0.5em}\end{minipage} &
			\multicolumn{2}{|l|}{
				\begin{minipage}[t]{11cm}\raggedright
					Sistema, Attaccante
				\vspace{0.5em}\end{minipage}
			} \\
			\hline
			
			\begin{minipage}[t]{3cm}\textbf{Description}\vspace{0.5em}\end{minipage} &
			\multicolumn{2}{|l|}{
				\begin{minipage}[t]{11cm}\raggedright
					Un avversario può provare alcuni nomi utente e password comuni o predefiniti per ottenere l'accesso al sistema ed eseguire azioni non autorizzate. Un avversario può provare una forza bruta intelligente usando password vuote, credenziali predefinite del fornitore note, nonché un dizionario di nomi utente e password comuni. Molti prodotti del fornitore sono preconfigurati con nomi utente e password predefiniti (e quindi ben noti) che dovrebbero essere eliminati prima dell'utilizzo in un ambiente di produzione. È un errore comune dimenticare di rimuovere queste credenziali di accesso predefinite. Un altro problema è che gli utenti sceglierebbero password molto semplici (comuni) (ad esempio "12345" o "password") che rendono più facile per l'attaccante ottenere l'accesso al sistema rispetto all'utilizzo di un attacco di forza bruta o anche di un attacco dizionario utilizzando un dizionario completo.				
				\vspace{0.5em}\end{minipage}
			} \\
			\hline
			
			\begin{minipage}[t]{3cm}\textbf{Data}\vspace{0.5em}\end{minipage} &
			\multicolumn{2}{|l|}{
				\begin{minipage}[t]{11cm}\raggedright
					Dati utente di sistema
				\vspace{0.5em}\end{minipage}
			} \\
			\hline
			
			\begin{minipage}[t]{3cm}\raggedright\textbf{Stimulus and Preconditions}\vspace{0.5em}\end{minipage} &
			\multicolumn{2}{|l|}{
				\begin{minipage}[t]{11cm}\raggedright
					\begin{itemize}
        				\item Il sistema utilizza l'autenticazione basata su password a un fattore. L'avversario ha i mezzi per interagire con il sistema;
        				\item Elenco specifico della tecnologia o del fornitore di nomi utente e password predefiniti.
      				\end{itemize}
				\vspace{0.5em}\end{minipage}
			} \\
			\hline
			
			\begin{minipage}[t]{3cm}\textbf{Attack Flow 1}\vspace{0.5em}\end{minipage} &
			\multicolumn{2}{|l|}{
				\begin{minipage}[t]{11cm}\raggedright 
					Un utente imposta la propria password su "password" o la lascia intenzionalmente vuota. Se il sistema non dispone di un'applicazione della forza della password rispetto a una politica di password valida, questa password potrebbe essere ammessa. L'attaccante tenta di utilizzare una serie di nomi utente e password comuni, tra cui "password" ed ottiene l'accesso al sistema.
				\vspace{0.5em}\end{minipage}
			} \\
			\hline
			
			\begin{minipage}[t]{3cm}\textbf{Attack Flow 2}\vspace{0.5em}\end{minipage} &
			\multicolumn{2}{|l|}{
				\begin{minipage}[t]{11cm}\raggedright 
					L'attaccante analizza il sistema per identificare le credenziali predefinite del fornitore. Successivamente, l'attaccante tenta di utilizzare queste credenziali per ottenere l'accesso al sistema.				
				\vspace{0.5em}\end{minipage}
			} \\
			\hline
			
			\begin{minipage}[t]{3cm}\textbf{Attack Flow 3}\vspace{0.5em}\end{minipage} &
			\multicolumn{2}{|l|}{
				\begin{minipage}[t]{11cm} 
						/
					\vspace{0.5em}\end{minipage}
				} \\
			\hline
			
			\begin{minipage}[t]{3cm}\raggedright\textbf{Response and Postconditions}\vspace{0.5em}\end{minipage} &
			\multicolumn{2}{|l|}{
				\begin{minipage}[t]{11cm}\raggedright 
					L'aggressore riesce ad ottenere mediante la prova di password comuni entrando nel sistema.				
				\vspace{0.5em}\end{minipage}
					} \\
			\hline
			
			\begin{minipage}[t]{3cm}\raggedright\textbf{Non Functional Requirements}\vspace{0.5em}\end{minipage} &
		\multicolumn{2}{|l|}{
			\begin{minipage}[t]{11cm}\raggedright
				Controllare l'input della password e dello username nella registrazione di un utente, obbligandolo a non utilizzare password o username comuni.			
			\vspace{0.5em}\end{minipage}
				} \\
			\hline
			
			\begin{minipage}[t]{3cm}\textbf{Mitigations}\vspace{0.5em}\end{minipage} &
			\multicolumn{2}{|l|}{
				\begin{minipage}[t]{11cm}\raggedright
					\begin{itemize}
            			 \item Elimina tutte le credenziali predefinite dell'account che possono essere inserite dal fornitore del prodotto.

        \item Implementare un meccanismo di limitazione delle password. Questo meccanismo dovrebbe prendere in considerazione sia l'indirizzo IP che il nome di accesso dell'utente.

        \item Metti insieme una politica di password forte e assicurati che tutte le password create dagli utenti siano conformi. In alternativa, genera automaticamente password complesse per gli utenti.

        \item Le password devono essere riciclate per prevenire l'invecchiamento, cioè ogni tanto deve essere scelta una nuova password.
        			\end{itemize}				
				\vspace{0.5em}\end{minipage}
			} \\
			\hline
			
			\begin{minipage}[t]{3cm}\textbf{Comments}\vspace{0.5em}\end{minipage} &
			\multicolumn{2}{|l|}{
				\begin{minipage}[t]{11cm}\raggedright
					Un utente imposta la propria password su "123" o lascia intenzionalmente la sua password vuota. Se il sistema non ha il controllo della password rispetto a una solida politica di password, questa password può essere ammessa. Password come questi due esempi sono due password semplici e comuni che possono essere facilmente indovinate dall'avversario.
				\vspace{0.5em}\end{minipage}
			} \\
			\hline
			
		\end{tabular}
	}
\end{table}

%%%%%%%%%%%%%%%%%% Credential stuffing (CAPEC 600)  %%%%%%%%%%%%%%%%%%%%%%%%%%
\begin{table}[ht!]
	\resizebox{\textwidth}{!}{
	\centering
	{\footnotesize
		\begin{tabular}{|l |l| l|}
			\hline
			\begin{minipage}[t]{3cm}\textbf{Case Type}\end{minipage} &
			\begin{minipage}[t]{6cm}\textbf{Abuse Case}\end{minipage} &
			\begin{minipage}[t]{5cm}\textbf{Case ID} AC-10 \end{minipage} \\ \hline
			
			\begin{minipage}[t]{3cm}\textbf{Case Name}\vspace{0.5em}\end{minipage} &
			\multicolumn{2}{|l|}{
				\begin{minipage}[t]{11cm}\raggedright
					Credential stuffing (CAPEC 600)
				\vspace{0.5em}\end{minipage}
			} \\
			\hline
			
			\begin{minipage}[t]{3cm}\textbf{Actors}\vspace{0.5em}\end{minipage} &
			\multicolumn{2}{|l|}{
				\begin{minipage}[t]{11cm}\raggedright
					 Sistema, Attaccante
				\vspace{0.5em}\end{minipage}
			} \\
			\hline
			
			\begin{minipage}[t]{3cm}\textbf{Description}\vspace{0.5em}\end{minipage} &
			\multicolumn{2}{|l|}{
				\begin{minipage}[t]{11cm}\raggedright
	Attacchi di questo tipo spesso prendono di mira i servizi di gestione su porte comunemente utilizzate come SSH, FTP, Telnet, LDAP, Kerberos, MySQL e altro ancora. Ulteriori obiettivi includono Single Sign-On (SSO) o applicazioni/servizi basati su cloud che utilizzano protocolli di autenticazione federati e applicazioni rivolte verso l'esterno.

    L'obiettivo principale di Credential Stuffing è ottenere il movimento laterale e ottenere l'accesso autenticato a sistemi, applicazioni e/o servizi aggiuntivi. Un attacco Credential Stuffing eseguito con successo potrebbe far sì che l'avversario impersoni la vittima o esegua qualsiasi azione che la vittima è autorizzata a eseguire.

    Sebbene non sia tecnicamente un attacco di forza bruta, gli attacchi Credential Stuffing possono funzionare come tali se un avversario possiede più password note per lo stesso account utente. Ciò può verificarsi nel caso in cui un avversario ottenga le credenziali utente da più fonti o se l'avversario ottiene la cronologia delle password di un utente per un account.

    Gli attacchi di riempimento delle credenziali sono simili agli attacchi di spruzzatura di password (CAPEC-565) per quanto riguarda i loro obiettivi e i loro obiettivi complessivi. Tuttavia, gli attacchi di spruzzatura di password non hanno alcuna intuizione sulle combinazioni note di nome utente/password e sfruttano invece password comuni o previste. Ciò significa anche che gli attacchi di Spruzzamento delle password devono evitare di indurre blocchi dell'account, che generalmente non è una preoccupazione per gli attacchi di riempimento delle credenziali. Gli attacchi di spruzzatura della password possono inoltre portare ad attacchi di riempimento delle credenziali, una volta scoperta una combinazione di successo di nome utente/password.				\vspace{0.5em}\end{minipage}
			} \\
			\hline
			
			\begin{minipage}[t]{3cm}\textbf{Data}\vspace{0.5em}\end{minipage} &
			\multicolumn{2}{|l|}{
				\begin{minipage}[t]{11cm}\raggedright
					Dati utente di sistema
				\vspace{0.5em}\end{minipage}
			} \\
			\hline
			
			\begin{minipage}[t]{3cm}\raggedright\textbf{Stimulus and Preconditions}\vspace{0.5em}\end{minipage} &
			\multicolumn{2}{|l|}{
				\begin{minipage}[t]{11cm}\raggedright
					\begin{itemize}
        				\item Il sistema/applicazione utilizza l'autenticazione basata su password a un fattore, SSO e/o l'autenticazione basata su cloud;

        \item Il sistema/applicazione non ha una solida politica di password che viene applicata;

        \item Il sistema/applicazione non implementa un meccanismo di limitazione della password efficace;

        \item L'avversario possiede un elenco di account utente noti e password corrispondenti che potrebbero esistere sul bersaglio;
        \item Un attacco Credential Stuffing è molto semplice;

        \item Una macchina con risorse sufficienti per il lavoro (ad es. CPU, RAM, HD);

        \item Un elenco noto di combinazioni nome utente/password;

        \item Uno script personalizzato che sfrutta l'elenco delle credenziali per avviare l'attacco.
      				\end{itemize}
				\vspace{0.5em}\end{minipage}
			} \\
			\hline
			
			\begin{minipage}[t]{3cm}\textbf{Attack Flow 1}\vspace{0.5em}\end{minipage} &
				\begin{minipage}[t]{3cm}\raggedright 
				Brute Force: Credential Stuffing	(T1110.004)	
				\vspace{0.5em}\end{minipage} &
				\begin{minipage}[t]{8cm}\raggedright 
				Gli aggressori possono utilizzare le credenziali ottenute dai dump di violazioni di account non correlati per accedere agli account target tramite la sovrapposizione delle credenziali. Occasionalmente, un gran numero di coppie di nome utente e password viene diffuso online quando un sito web o un servizio viene compromesso e le credenziali dell'account utente vengono utilizzate. Le informazioni possono essere utili a un aggressore che tenta di compromettere gli account, sfruttando la tendenza degli utenti a utilizzare le stesse password per account personali e aziendali.

				Il credential stuffing è un'opzione rischiosa perché potrebbe causare numerosi errori di autenticazione e blocchi degli account, a seconda delle policy di errore di accesso dell'organizzazione.				\vspace{0.5em}\end{minipage}
			\\
			\hline
			
			\begin{minipage}[t]{3cm}\textbf{Attack Flow 2}\vspace{0.5em}\end{minipage} &
			\multicolumn{2}{|l|}{
				\begin{minipage}[t]{11cm}\raggedright 
					/				
					\vspace{0.5em}\end{minipage}
			} \\
			\hline
			
			\begin{minipage}[t]{3cm}\textbf{Attack Flow 3}\vspace{0.5em}\end{minipage} &
			\multicolumn{2}{|l|}{
				\begin{minipage}[t]{11cm} 
						/
					\vspace{0.5em}\end{minipage}
				} \\
			\hline
			
			\begin{minipage}[t]{3cm}\raggedright\textbf{Response and Postconditions}\vspace{0.5em}\end{minipage} &
			\multicolumn{2}{|l|}{
				\begin{minipage}[t]{11cm}\raggedright 
					L’aggressore ottiene l’accesso non autorizzato ed effettua azioni malevole.				\vspace{0.5em}\end{minipage}
					} \\
			\hline
			
			\begin{minipage}[t]{3cm}\raggedright\textbf{Non Functional Requirements}\vspace{0.5em}\end{minipage} &
		\multicolumn{2}{|l|}{
			\begin{minipage}[t]{11cm}\raggedright
				Garantire il monitoraggio dei registri di sistema e l'implementazione dell'autentificazione a più fattori.			
			\vspace{0.5em}\end{minipage}
				} \\
			\hline
			
			\begin{minipage}[t]{3cm}\textbf{Mitigations}\vspace{0.5em}\end{minipage} &
			\multicolumn{2}{|l|}{
				\begin{minipage}[t]{11cm}\raggedright
					\begin{itemize}
            			\item Sfrutta l'autenticazione a più fattori per tutti i servizi di autenticazione e prima di concedere a un'entità l'accesso alla rete di domini;

        \item Crea una politica di password complessa e assicurati che il tuo sistema applichi questa politica;

        \item Assicurati che gli utenti non riutilizzino le combinazioni di nome utente/password per più sistemi, applicazioni o servizi;

        \item Non riutilizzare le credenziali dell'account amministratore locale su tutti i sistemi;

        \item Nega l'uso remoto delle credenziali di amministratore locali per accedere ai sistemi di dominio;

        \item Non consentire agli account di essere un amministratore locale su più di un sistema;

        \item Implementare un meccanismo intelligente di limitazione delle password. È necessario fare attenzione a garantire che questi meccanismi non consentano eccessivamente gli attacchi di blocco dell'account come CAPEC-2;

        \item Monitorare i registri di sistema e di dominio per l'accesso anomalo alle credenziali.
        			\end{itemize}				
				\vspace{0.5em}\end{minipage}
			} \\
			\hline
			
			\begin{minipage}[t]{3cm}\textbf{Comments}\vspace{0.5em}\end{minipage} &
			\multicolumn{2}{|l|}{
				\begin{minipage}[t]{11cm}\raggedright
					Un utente sfrutta la password "Password123" per una manciata di accessi alle applicazioni. Un avversario ottiene la combinazione nome utente/password di una vittima da una violazione di un'applicazione di social media ed esegue un attacco Credential Stuffing contro più applicazioni bancarie e di carte di credito. Poiché l'utente sfrutta le stesse credenziali per l'accesso al proprio conto bancario, l'avversario si autentica con successo sul conto bancario dell'utente e trasferisce denaro su un conto offshore.
				\vspace{0.5em}\end{minipage}
			} \\
			\hline
			
		\end{tabular}
		}
	}
\end{table}

%%%%%%%%%%%%%%%%%% Traffic Injection (CAPEC 594) %%%%%%%%%%%%%%%%%%%%%%%%%%
\begin{table}[ht!]
	\resizebox{\textwidth}{!}{
	\centering
	{\footnotesize
		\begin{tabular}{|l |l| l|}
			\hline
			\begin{minipage}[t]{3cm}\textbf{Case Type}\end{minipage} &
			\begin{minipage}[t]{6cm}\textbf{Abuse Case}\end{minipage} &
			\begin{minipage}[t]{5cm}\textbf{Case ID} AC-11 \end{minipage} \\ \hline
			
			\begin{minipage}[t]{3cm}\textbf{Case Name}\vspace{0.5em}\end{minipage} &
			\multicolumn{2}{|l|}{
				\begin{minipage}[t]{11cm}\raggedright
					Traffic Injection (CAPEC 594)
				\vspace{0.5em}\end{minipage}
			} \\
			\hline
			
			\begin{minipage}[t]{3cm}\textbf{Actors}\vspace{0.5em}\end{minipage} &
			\multicolumn{2}{|l|}{
				\begin{minipage}[t]{11cm}\raggedright
					 Sistema, Attaccante
				\vspace{0.5em}\end{minipage}
			} \\
			\hline
			\begin{minipage}[t]{3cm}\textbf{Description}\vspace{0.5em}\end{minipage} &
			\multicolumn{2}{|l|}{
				\begin{minipage}[t]{11cm}\raggedright
	Un avversario inietta traffico nella connessione di rete del bersaglio. L'avversario è quindi in grado di degradare o interrompere la connessione e potenzialmente modificarne il contenuto. Non si tratta di un attacco flooding, poiché l'avversario non si concentra sull'esaurimento delle risorse. Piuttosto, sta elaborando un input specifico per influenzare il sistema in un modo particolare.		\vspace{0.5em}\end{minipage}
			} \\
			\hline
			
			\begin{minipage}[t]{3cm}\textbf{Data}\vspace{0.5em}\end{minipage} &
			\multicolumn{2}{|l|}{
				\begin{minipage}[t]{11cm}\raggedright
					DATA
				\vspace{0.5em}\end{minipage}
			} \\
			\hline
			
			\begin{minipage}[t]{3cm}\raggedright\textbf{Stimulus and Preconditions}\vspace{0.5em}\end{minipage} &
			\multicolumn{2}{|l|}{
				\begin{minipage}[t]{11cm}\raggedright
					\begin{itemize}
        				\item L'applicazione di destinazione deve sfruttare un canale di comunicazione aperto.
        \item Il canale su cui comunica il bersaglio deve essere vulnerabile all'intercettazione.
      				\end{itemize}
				\vspace{0.5em}\end{minipage}
			} \\
			\hline
			
			\begin{minipage}[t]{3cm}\textbf{Attack Flow 1}\vspace{0.5em}\end{minipage} &
				\multicolumn{2}{|l|}{
				\begin{minipage}[t]{11cm}\raggedright
					L'attaccante intercetta il traffico di rete tra due endpoint comunicanti. Utilizzando tecniche come la spoofing degli indirizzi IP o ARP, l'attaccante si posiziona come intermediario nella comunicazione. Successivamente, l'attaccante inietta pacchetti di dati malevoli o manipolati nel flusso di traffico esistente, alterando il contenuto delle comunicazioni tra gli endpoint.
				\vspace{0.5em}\end{minipage}
			} 
			\\
			\hline
			
			\begin{minipage}[t]{3cm}\textbf{Attack Flow 2}\vspace{0.5em}\end{minipage} &
			\multicolumn{2}{|l|}{
				\begin{minipage}[t]{11cm}\raggedright 
					L'attaccante monitora il traffico di rete per identificare pacchetti specifici da iniettare. Utilizzando strumenti di analisi del traffico, l'attaccante seleziona i momenti opportuni per inserire i pacchetti malevoli, cercando di evitare rilevamenti e garantire che i pacchetti iniettati vengano accettati dagli endpoint di destinazione.		
					\vspace{0.5em}\end{minipage}
			} \\
			\hline
			
			\begin{minipage}[t]{3cm}\textbf{Attack Flow 3}\vspace{0.5em}\end{minipage} &
			\multicolumn{2}{|l|}{
				\begin{minipage}[t]{11cm} 
						/
					\vspace{0.5em}\end{minipage}
				} \\
			\hline
			
			\begin{minipage}[t]{3cm}\raggedright\textbf{Response and Postconditions}\vspace{0.5em}\end{minipage} &
			\multicolumn{2}{|l|}{
				\begin{minipage}[t]{11cm}\raggedright 
					 \begin{itemize}
        \item L’attacco altera il traffico di rete, causando risposte errate o blocchi che rendono il servizio non affidabile.
        \item L’attaccante modifica o inserisce dati nel traffico, compromettendo l’accuratezza e la coerenza delle informazioni scambiate. 
      \end{itemize}				\vspace{0.5em}\end{minipage}
					} \\
			\hline
			
			\begin{minipage}[t]{3cm}\raggedright\textbf{Non Functional Requirements}\vspace{0.5em}\end{minipage} &
		\multicolumn{2}{|l|}{
			\begin{minipage}[t]{11cm}\raggedright
				Garantire la protezione dell’integrità del traffico tramite autenticazione delle comunicazioni e il rilevamento di pacchetti anomali o non autorizzati.		
			\vspace{0.5em}\end{minipage}
				} \\
			\hline
			
			\begin{minipage}[t]{3cm}\textbf{Mitigations}\vspace{0.5em}\end{minipage} &
			\multicolumn{2}{|l|}{
				\begin{minipage}[t]{11cm}\raggedright
					\begin{itemize}
            			 \item Usare TLS per tutte le comunicazioni di rete, così da proteggere confidenzialità e integrità, rendendo più difficile l’iniezione di pacchetti malevoli.
        \item Verificare l’identità degli endpoint di rete tramite protocolli sicuri (TLS, IPsec) per assicurarsi che i dati provengano da fonti legittime.
        \item Implementare ingress filtering per bloccare pacchetti con indirizzi IP falsificati, riducendo la possibilità di spoofing e injection.
        \item Deployare sistemi di rilevamento/prevenzione delle intrusioni per analizzare il traffico di rete e bloccare pattern sospetti o iniettati.
        \item Configurare policy di rete che negano per default il traffico non autorizzato, consentendo solo le comunicazioni esplicitamente permesse.
        \item In ambienti aziendali, usare la decrittazione controllata del TLS presso punti di sicurezza per permettere l’analisi del traffico cifrato senza compromettere la sicurezza.
        \item Usare firewall e limiti di traffico per evitare sovraccarichi o pacchetti in eccesso.
        			\end{itemize}				
				\vspace{0.5em}\end{minipage}
			} \\
			\hline
			
			\begin{minipage}[t]{3cm}\textbf{Comments}\vspace{0.5em}\end{minipage} &
			\multicolumn{2}{|l|}{
				\begin{minipage}[t]{11cm}\raggedright
					/
				\vspace{0.5em}\end{minipage}
			} \\
			\hline
			
		\end{tabular}
		}
	}
\end{table}

%%%%%%%%%%%%%%%%%% Traffic Injection (CAPEC 594) %%%%%%%%%%%%%%%%%%%%%%%%%%
\begin{table}[ht!]
	\resizebox{\textwidth}{!}{
	\centering
	{\footnotesize
		\begin{tabular}{|l |l| l|}
			\hline
			\begin{minipage}[t]{3cm}\textbf{Case Type}\end{minipage} &
			\begin{minipage}[t]{6cm}\textbf{Abuse Case}\end{minipage} &
			\begin{minipage}[t]{5cm}\textbf{Case ID} AC-13 \end{minipage} \\ \hline
			
			\begin{minipage}[t]{3cm}\textbf{Case Name}\vspace{0.5em}\end{minipage} &
			\multicolumn{2}{|l|}{
				\begin{minipage}[t]{11cm}\raggedright
					Counterfeit GPS Signals (CAPEC 627)
				\vspace{0.5em}\end{minipage}
			} \\
			\hline
			
			\begin{minipage}[t]{3cm}\textbf{Actors}\vspace{0.5em}\end{minipage} &
			\multicolumn{2}{|l|}{
				\begin{minipage}[t]{11cm}\raggedright
					 Sistema, Attaccante
				\vspace{0.5em}\end{minipage}
			} \\
			\hline
			\begin{minipage}[t]{3cm}\textbf{Description}\vspace{0.5em}\end{minipage} &
			\multicolumn{2}{|l|}{
				\begin{minipage}[t]{11cm}\raggedright
	Un avversario tenta di ingannare un ricevitore GPS trasmettendo segnali GPS contraffatti, strutturati in modo da assomigliare a un insieme di segnali GPS normali. Questi segnali falsificati possono essere strutturati in modo tale da indurre il ricevitore a stimare la propria posizione in un luogo diverso da quello in cui si trova effettivamente, oppure a considerarla localizzata in un luogo diverso ma in un momento diverso, come determinato dall'avversario.		\vspace{0.5em}\end{minipage}
			} \\
			\hline
			
			\begin{minipage}[t]{3cm}\textbf{Data}\vspace{0.5em}\end{minipage} &
			\multicolumn{2}{|l|}{
				\begin{minipage}[t]{11cm}\raggedright
					DATA
				\vspace{0.5em}\end{minipage}
			} \\
			\hline
			
			\begin{minipage}[t]{3cm}\raggedright\textbf{Stimulus and Preconditions}\vspace{0.5em}\end{minipage} &
			\multicolumn{2}{|l|}{
				\begin{minipage}[t]{11cm}\raggedright
					\begin{itemize}
        				\item Per eseguire operazioni critiche, il bersaglio deve poter contare su un segnale GPS valido.

        \item Capacità di creare segnali GPS falsificati.
        \item La capacità di falsificare i segnali GPS non è banale.
      				\end{itemize}
				\vspace{0.5em}\end{minipage}
			} \\
			\hline
			
			\begin{minipage}[t]{3cm}\textbf{Attack Flow 1}\vspace{0.5em}\end{minipage} &
				\begin{minipage}[t]{3cm}\raggedright 
				Defacement: Internal Defacement	(T1491.001)	
				\vspace{0.5em}\end{minipage} &
				\begin{minipage}[t]{8cm}\raggedright 
				Un avversario può deturpare i sistemi interni di un'organizzazione nel tentativo di intimidire o fuorviare gli utenti, screditando così l'integrità dei sistemi. Ciò può assumere la forma di modifiche ai siti web interni o ai messaggi di accesso al server, o direttamente ai sistemi degli utenti con la sostituzione dello sfondo del desktop. Immagini inquietanti o offensive possono essere utilizzate come parte di un deturpamento interno per causare disagio all'utente o per esercitare pressioni sull'osservanza dei messaggi di accompagnamento. Poiché il deturpamento interno dei sistemi espone la presenza di un avversario, spesso avviene dopo che altri obiettivi di intrusione sono stati raggiunti. 			\vspace{0.5em}\end{minipage}
			\\
			\hline
			
			\begin{minipage}[t]{3cm}\textbf{Attack Flow 2}\vspace{0.5em}\end{minipage} &
				\begin{minipage}[t]{3cm}\raggedright 
				Defacement: External Defacement	(T1491.002)	
				\vspace{0.5em}\end{minipage} &
				\begin{minipage}[t]{8cm}\raggedright 
				Un avversario può deturpare sistemi esterni a un'organizzazione nel tentativo di inviare messaggi, intimidire o in altro modo fuorviare un'organizzazione o gli utenti. Il defacement esterno può in ultima analisi indurre gli utenti a diffidare dei sistemi e a mettere in dubbio/screditare l'integrità del sistema. I siti web rivolti all'esterno sono una vittima comune del defacement; spesso presi di mira da avversari e gruppi di hacktivisti al fine di diffondere un messaggio politico o propaganda. Il defacement esterno può essere utilizzato come catalizzatore per innescare eventi o come risposta ad azioni intraprese da un'organizzazione o da un governo. Allo stesso modo, il defacement di un sito web può anche essere utilizzato come impostazione o precursore per attacchi futuri come il Drive-by Compromise.			\vspace{0.5em}\end{minipage}
			\\
			\hline
			\begin{minipage}[t]{3cm}\textbf{Attack Flow 3}\vspace{0.5em}\end{minipage} &
			\multicolumn{2}{|l|}{
				\begin{minipage}[t]{11cm} 
						/
					\vspace{0.5em}\end{minipage}
				} \\
			\hline
			
			\begin{minipage}[t]{3cm}\raggedright\textbf{Response and Postconditions}\vspace{0.5em}\end{minipage} &
			\multicolumn{2}{|l|}{
				\begin{minipage}[t]{11cm}\raggedright 
					 \begin{itemize}
        \item Quando il sistema riconosce segnali GPS contraffatti, deve informare immediatamente l’utente o i componenti interni che il dato non è affidabile.
		\item La posizione deve smettere di influenzare funzioni critiche e deve essere usata una modalità alternativa basata su sensori interni o logiche di stima del movimento.
		\item Una volta che il segnale torna a essere normale, il sistema deve verificare che i valori siano coerenti prima di ripristinare il funzionamento completo.
		\item Dopo l’evento, devono essere conservate informazioni utili per capire cosa è successo e per adattare i controlli futuri, così da evitare che un attacco simile abbia successo di nuovo.
      \end{itemize}				\vspace{0.5em}\end{minipage}	
					} \\
			\hline
			
			\begin{minipage}[t]{3cm}\raggedright\textbf{Non Functional Requirements}\vspace{0.5em}\end{minipage} &
		\multicolumn{2}{|l|}{
			\begin{minipage}[t]{11cm}\raggedright
				Il sistema deve essere in grado di mantenere la propria affidabilità anche se il GPS diventa non attendibile, reagendo rapidamente a variazioni sospette della posizione e deve continuare a funzionare grazie a fonti alternative, evitando improvvisi malfunzionamenti.
			\vspace{0.5em}\end{minipage}
				} \\
			\hline
			
			\begin{minipage}[t]{3cm}\textbf{Mitigations}\vspace{0.5em}\end{minipage} &
			\multicolumn{2}{|l|}{
				\begin{minipage}[t]{11cm}\raggedright
					\begin{itemize}
            			 \item Le principali mitigazioni consistono nel riconoscere quando un segnale GPS non è autentico e nel ridurre la dipendenza dal GPS stesso.
						\item È utile confrontare continuamente la posizione ricevuta con altre fonti, come sensori interni, mappe, altri sistemi satellitari o dati provenienti da reti terrestri.
						\item La coerenza del movimento deve essere monitorata: cambiamenti troppo rapidi o improvvisi sono indizi di spoofing.
						\item Anche l’analisi della potenza del segnale permette di rilevare comportamenti anomali, perché i segnali falsi spesso risultano più forti dei segnali dei satelliti reali.
						\item A livello hardware si possono usare ricevitori avanzati e antenne progettate per filtrare segnali sospetti.
        			\end{itemize}				
				\vspace{0.5em}\end{minipage}
			} \\
			\hline
			
			\begin{minipage}[t]{3cm}\textbf{Comments}\vspace{0.5em}\end{minipage} &
			\multicolumn{2}{|l|}{
				\begin{minipage}[t]{11cm}\raggedright
					/
				\vspace{0.5em}\end{minipage}
			} \\
			\hline
			
		\end{tabular}
		}
	}
\end{table}