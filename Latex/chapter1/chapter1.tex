\chapter{Introduzione} 

\section{Descrizione del progetto}

Il progetto in esame propone lo sviluppo di un sistema software per la gestione di un'azienda di spedizioni. Essa si appoggerà a delle aziende di corrieri per effettuare le spedizioni. Il software gestitrà vari aspetti, come la gestione dei clienti, degli ordini, dei corrieri, del tracciamento dei pagamenti e del tracciamento delle spedizioni. Per far fronte a queste esigenze, si useranno tecnologie innovative come la rete blockchain, in particolare essa è basata sull'algoritmo QBFT (Quorum Byzantine Fault Tolerance). Il tipo di algoritmo è PoA, Proof of Authority, il cui funzionamento si basa sull'identità e la reputazione di un gruppo di nodi fidati e autorizzati per convalidare transazioni e creare nuovi blocchi, quindi è basato sul consenso. I suoi vantaggi principali sono che è un meccanismo efficiente, particolarmente adatto alle blockchain private o permissioned, basso consumo di energia, elevate performance e scalabilità, però sacrifica un certo grado di decentralizzazione. I dati esterni delle transazioni sono collegati alla blockchain tramite un servizio chiamato oracolo (on-chain). Inoltre, essi verificano l'autenticità e la validità delle transazioni grazie ad una rete bayesiana e dei dati che fanno da prove. Infine, si usano gli smart contract per interagire con le transazioni, la blockchain e gli oracoli. Essi sono dei programmi che controllano eventi o transazioni che accadono ed intraprendono delle azioni, anche i modo automatizzato. Grazie alla immutabilità delle transazioni registrate, è il codice stesso registrato sulla blockchain a garantire la corretta esecuzione delle interazioni, ma anche il funzionamento generale del sistema (senza un intermediario).
%%
%\begin{figure}[h]
%	\centering
%	\includegraphics [width=.35\columnwidth, angle=0]{chapter1/immagini/}
%	\caption{}
%	\label{}
%\end{figure}
%%
