\chapter{Programmazione Sicura} 

In questo capitolo, saranno presentate le pratiche e le tecniche per garantire la sicurezza nella programmazione. La sicurezza del software è diventata una preoccupazione fondamentale nello sviluppo di applicazioni moderne, poiché le vulnerabilità possono essere sfruttate da attori malintenzionati per compromettere i dati e le funzionalità del sistema. Sarà presentato un \textit{Runtime enforcement Monitor} per assicurare che le proprietà modellate nel capitolo precedente siano soddisfatte a tempo di esecuzione. Inoltre si tratterà di aspetti dell'applicazione sia a livello \textit{off-chain}, sia a livello \textit{on-chain}, con particolare attenzione alle best practice per la scrittura di codice sicuro.

%%%%%%%%%%%%%%%%%%%%%%%%%%%%%%%%%%%%%%%%%%%%%%%%%%%%%
\section{Modellazione di un monitor Runtime Enforcement}
Per garantire che le proprietà di sicurezza identificate nel capitolo precedente siano rispettate durante l'esecuzione del sistema, è stato implementato un monitor di \textit{Runtime Enforcement}. Questo monitor agisce come un supervisore che osserva le azioni del sistema in tempo reale e interviene quando rileva una violazione delle proprietà di sicurezza.

\begin{figure}[ht!]
	\centering
	\includegraphics[width=0.6\columnwidth]{chapter4/immagini/Monitor_RE.png}
    \caption{Modello del monitor di Runtime Enforcement.}
\end{figure}


%%%%%%%%%%%%%%%%%%%%%%%%%%%%%%%%%%%%%%%%%%%%%%%%%%%%%
\section{Programmazione off-chain e best practices}

In questa sezione, verranno discusse le migliori pratiche per la programmazione sicura in ambienti off-chain, che, nel nostro caso, includono lo sviluppo di applicazioni web. Si affronteranno argomenti come l'uso di librerie sicure, la validazione degli input e la protezione dei dati sensibili. Per l'analisi statica del codice è stato deciso di usare \textit{SonarQube}, una piattaforma open-source per l'integrazione nei flussi di lavoro di sviluppo software che garantisce la qualità e la sicurezza del codice, andando ad esaminarlo per trovare vulnerabilità, problemi di sicurezza ed altro. Inoltre, per una buona progettazione del sistema, sono state seguite alcune linee guida:

\begin{itemize}
    \item \textbf{Validare tutti gli input del programma}: ogni informazione in ingresso al programma è stata validato. In particolare, sono stati controlli per prevenire attacchi di tipo \textit{SQL Injection} e \textit{Cross-Site Scripting (XSS)}.
    \item \textbf{Creare un handler per ogni eccezione}: ogni possibile eccezione che può essere generata dal programma è stata gestita in modo appropriato con degli handler, cioè dei costrutti del tipo \textit{try/except}, evitando che il programma termini in modo anomalo.
    \item \textbf{Controllo sulla lunghezza delle stringhe ed altre strutture dati}: sono stati implementati dei controlli sulla lunghezza delle stringhe e di altre strutture dati per prevenire attacchi di tipo \textit{buffer overflow} o \textit{buffer underflow}.
    \item \textbf{Utilizzo di librerie sicure e aggiornate}: sono state utilizzate solo librerie e framework che sono stati regolarmente aggiornati e che hanno una buona reputazione in termini di sicurezza.
    \item \textbf{Limitare la visibilità delle informazioni}: sono stati implementati meccanismi per limitare la visibilità delle informazioni del programma esternamente, cioè l'utente può interagire con il sistema solo tramite le interfacce messe a disposizione.
    \item \textbf{Uso di algoritmi di hashing per informazioni sensibili}: per proteggere le informazioni sensibili, come le password degli utenti, sono stati utilizzati algoritmi di hashing sicuri (ad esempio \textit{bcrypt} o \textit{PBKDF2} con SHA-256).
\end{itemize}

Inoltre, per una maggiore sicurezza e resilienza, l'applicazione web del sistema viene eseguita in un ambiente virtuale di python (venv) per disaccoppiare le versioni di Python e le varie librerie presenti nelle macchine.

%%%%%%%%%%%%%%%%%%%%%%%%%%%%%%%%%%%%%%%%%%%%%%%%%%%%%
\section{Programmazione on-chain e best practices}

Per eseguire programmi sulla blockchain di Ethereum, sono stati utilizzati gli smart contract, scritti nel linguaggio di programmazione \textit{Solidity}. I principali vantaggi degli smart contract sono la trasparenza, l'automazione, l'immutabilità e la sicurezza. Tuttavia, è fondamentale seguire delle best practice per garantire che gli smart contract siano sicuri e privi di vulnerabilità. Per esempio, è stata utilizzata la libreria \textit{OpenZeppelin}, che fornisce implementazioni sicure e testate di smart contract, come token ERC20, gestione degli accessi e altro. I token sono delle rappresentazioni digitali di un qualcosa in una blockchain; potrebbe essere valute, asset, diritti di voto, eccetera.